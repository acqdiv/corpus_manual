\documentclass[a4paper, 11pt]{book}

\usepackage{array}
\usepackage{booktabs}
\usepackage{color}					% more color
	\definecolor{gray}{rgb}{0.4,0.4,0.4}
	\definecolor{darkblue}{rgb}{0.0,0.0,0.6}
	\definecolor{cyan}{rgb}{0.0,0.6,0.6}
\usepackage[bottom=2.75cm,left=2.75cm,right=2.75cm,top=2.75cm]{geometry}
\usepackage{fancyhdr}
	\pagestyle{fancy}
	\fancyhead[LE]{\leftmark}
	\fancyhead[LO,RE]{}
	\fancyhead[RO]{\rightmark}
	\fancyfoot[L]{\small{UZH, ACQDIV}}
	\fancyfoot[C]{}
	\fancyfoot[R]{\small\thepage}
\usepackage{fontspec}
	\setmainfont{Linux Libertine O}
	\newfontface \courier{Courier New}
	\newfontface \foreign{Arial Unicode MS}
\usepackage{graphicx}
\usepackage{hyperref}
	\hypersetup{pdfborder={0 0 0}} 	% no boxes (if colorlinks=false)
	\hypersetup{colorlinks=true, linkcolor=blue, citecolor=blue, urlcolor=blue} 	% color link text, do not box it
	\def\chapterautorefname{Chapter}
	\def\sectionautorefname{Section}
	\def\subsectionautorefname{Section}
	\def\subsubsectionautorefname{Section}
	\urlstyle{same}
\usepackage{linguex}			% linguistic examples with glosses
	\let\eachwordone=\it		% first line of gloss in italics
\usepackage{listings}				% embed code with syntax highlighting
\usepackage{longtable}
\usepackage{lscape}
\usepackage{mdwlist}
\usepackage{multicol}		% format text in multiple columns as \begin{multicols}{number}{text}
\usepackage{natbib}			% for flexible bibliography; multiple citations separated by
	\bibpunct[:]{(}{)}{,}{a}{}{,} % bibliography style with opening bracket, closing bracket, punctuation between multiple citations, citation style, 
	\def\newblock{\hskip .11em plus .33em minus .07em} % punctuation between author and year, punctuation between several years with a common author
	\setcitestyle{authoryear, round, citesep={,}, aysep={}, yysep={,}, notesep={:}}

\newcommand{\bks}{\textbackslash}	% backslash with \bks
\newcommand{\filename}[1]{“#1”} % file names and paths
\newcommand{\source}[1]{\hfill (#1)\\[-0.2cm]}	% sources for examples
\newcommand*\rot{\rotatebox{90}}
\newcommand{\term}[1]{\textbf{#1}} % terminology
\newcommand{\til}{\textasciitilde}	% nice tilde with \til
\newcommand{\und}{\underline{{ }}\hspace{0.2mm}}	% nice underscores with \und
\renewcommand{\firstrefdash}{}			% change dash when referring to embedded examples
\renewcommand{\labelitemii}{$\circ$}	% change symbol used in lists
\renewcommand{\labelitemi}{$\bullet$}	% change symbol used in lists


%%%%%%%%%%%%%%%%%%%%%%% TODOS %%%%%%%%%%%%%%%%%%%%%%%%

% TODO discuss with Steve, Sabine, Robert:
% location, release routine. Publish fully anonymized version? 
% TODO info plots

%%%%%%%%%%%%%%%%%%%%%%% TODOS %%%%%%%%%%%%%%%%%%%%%%%%


\begin{document}
% ‘’ “”

% title page
% \maketitle

\begin{center}
	
	\begin{figure}[h!]
		\vspace{2.4cm}
		\includegraphics[scale=0.8]{pics/acqdivgraph.pdf}
		\vspace{0.5cm}
	\end{figure}

	\Huge{\textbf{Manual for the ACQDIV Corpus}\\[0.5cm]}
	\LARGE{Robert Schikowski, Steven Moran and Sabine Stoll}

\end{center}

\tableofcontents
\listoffigures
\listoftables

\chapter{Introduction}

\section{Purpose and structure of this document}

This manual describes the corpora used and compiled in the ERC project “Acquisition processes in maximally diverse languages: min(d)ing the ambient language” (ACQDIV, grant no.\ 615988, 01/09/2014 - 31/08/2019, PI Sabine Stoll) \textendash\ in short, the “ACQDIV Corpus”. Also see \url{http://www.acqdiv.uzh.ch} and \url{http://www.psycholinguistics.uzh.ch} for the latest information. 

The remainder of the manual is divided into five chapters. \autoref{cha:Overview of the dataset} gives an overview of the data contained in the ACQDIV Corpus, and \autoref{cha:Details of the corpus} describes the format and content of the corpus in greater detail. Since the corpus is dynamically generated from several subcorpora, the following \autoref{cha:Data sources} describes the original data and how they are recast into the target structure. Readers with a technical interest may consult \autoref{cha:Generating the corpus} to learn more about the individual steps involved in this procedure. Finally, \autoref{cha:Information for developers} provides information for developers interested in extending the described architecture and methods to other resources.

\autoref{cha:Overview of the dataset} starts with a brief introduction to the \hyperref[sec:The language sample]{ACQDIV languages}, including examples for their diversity, shows an overview of the \hyperref[sec:Amount of data]{size} of the subcorpora (given as the number of utterances, words, and morphemes in each), and summarizes differences between the subcorpora regarding the \hyperref[sec:Sampling for speakers and periods]{sampling} of speakers and recording periods. This chapter also sketches the available \hyperref[sec:Annotation layers and data gaps]{annotation layers} as well as notable data gaps in individual subcorpora. 

\autoref{cha:Details of the corpus} starts with the conditions of access and extension of the corpus in \autoref{sec:Access to the corpus}. It introduces the \hyperref[sec:Architecture]{conceptual architecture} of the corpus and the \hyperref[sec:Format]{formats} in which this is implemented. This is followed by detailed information on the tables and fields of the corpus database in \autoref{sec:Structure of the corpus} and lists of standardized values (e.g.\ for glosses and parts of speech) in \autoref{sec:Conventions and standardization}. 

\autoref{cha:Data sources} first gives an overview of the subcorpora's original \hyperref[sec:Corpus formats]{corpus formats}, i.e.\ CHAT, TalkBank XML, and Toolbox. It then deals with the subcorpora in alphabetical order: \hyperref[sec:Chintang]{Chintang}, \hyperref[sec:Cree]{Cree}, \hyperref[sec:Indonesian]{Indonesian}, \hyperref[sec:Inuktitut]{Inuktitut}, \hyperref[sec:Japanese MiiPro]{Japanese MiiPro}, \hyperref[sec:Japanese Miyata]{Japanese Miyata}, \hyperref[sec:Russian]{Russian}, \hyperref[sec:Sesotho]{Sesotho}, \hyperref[sec:Turkish]{Turkish}, and \hyperref[sec:Yucatec]{Yucatec}. Each section deals with the same recurring aspects: accessibility of the data, recording schemes, file systems and formats, and corpus formats. The subsection on corpus formats also describes how source structures are mapped to target structures in the ACQDIV Corpus. 

\autoref{cha:Generating the corpus} deals with the steps involved in building the ACQDIV Corpus from the subcorpora in roughly chronological order. The first two steps mainly apply to corpora which were initially not available in an accepted input format (TalkBank XML or Toolbox). These corpora required automatic and manual cleaning of \hyperref[sec:Cleaning of file formats]{files} (including file systems, file names, and encodings) and \hyperref[sec:Cleaning of corpus formats]{corpus formats} (typically broken CHAT). The following steps apply to all corpora: they are \hyperref[sec:Parsing the corpus data]{parsed} and read into the \hyperref[sec:Building the database and postprocessing]{dynamically generated database}. The data are then \hyperref[sec:Building the database and postprocessing]{postprocessed} for the last finish. 

\autoref{cha:Information for developers} contains a very brief overview of the Python architecture behind the cleaning and the generation of the database and provides contacts for further information and access to the ACQDIV repository on GitHub. 


% Once the corpus is loaded, you are ready to query it. Below are some examples for queries phrased as SQL (the query language for the database, first line in each block) and R (second line).
%
% \begin{itemize}
% 	\item \textbf{Show all utterances} \\[0.2cm]
% 		\texttt{SELECT * FROM utterances} \\[0.19cm]
% 		\texttt{utterances}
%
% 	\item \textbf{Show all utterances from the Turkish corpus} \\[0.2cm]
% 		\texttt{SELECT * FROM utterances WHERE corpus="Turkish"} \\[0.19cm]
% 		\texttt{utterances[utterances\$corpus=="Turkish", ]}
%
% 	\item \textbf{Show only transcriptions and translations for Turkish utterances} \\[0.2cm]
% 		\texttt{SELECT utterance, translation FROM utterances WHERE corpus="Turkish"} \\[0.19cm]
% 		\texttt{utterances[utterances\$corpus=="Turkish", c("utterance",\\ "translation")]}
%
% 	\item \textbf{Get the types of unified part-of-speech tags used in Sesotho} \\[0.2cm]
% 		\texttt{SELECT DISTINCT pos FROM morphemes WHERE corpus="Sesotho"} \\[0.19cm]
% 		\texttt{unique(morphemes[morphemes\$corpus=="Sesotho", "pos"])} % unique(droplevels(morphemes[morphemes$corpus=="Chintang","pos_raw"]))
%
% 	\item \textbf{Count how often every part-of-speech type occurs in the individual corpora} \\[0.2cm]
% 		\texttt{SELECT corpus, pos, COUNT(*) AS `frequency` FROM morphemes GROUP BY corpus, pos} \\[0.19cm]
% 		\texttt{table(morphemes[ , c("corpus", "pos")])} % table(droplevels(morphemes[,c("corpus","pos_raw")]))
%
% 	\item \textbf{Get all utterances consisting of a negation marker} \\[0.2cm]
% 		\texttt{SELECT utterances.* FROM utterances \\
% 				INNER JOIN morphemes ON utterances.corpus=morphemes.corpus AND utterances.\\utterance\und id=morphemes.utterance\und id\und fk \\
% 				WHERE morphemes.gloss="NEG"} \\[0.19cm]
% 		\texttt{merge(utterances, morphemes, by.x=c("corpus", "utterance\und id"), by.y=\\c("corpus", "utterance\und id\und fk")) -> all; all[all\$gloss=="NEG", ]}
%
% 	\item \textbf{Get all utterances containing a negation marker, group by corpus} \\[0.2cm]
% 		\texttt{SELECT utterances.* FROM utterances \\
% 				INNER JOIN morphemes ON utterances.corpus=morphemes.corpus AND utterances.\\utterance\und id=morphemes.utterance\und id\und fk \\
% 				WHERE morphemes.gloss LIKE "\%NEG\%" ORDER BY utterances.corpus} \\[0.19cm]
% 		\texttt{merge(utterances, morphemes, by.x=c("corpus", "utterance\und id"), by.y=\\c("corpus", "utterance\und id\und fk")) -> all \\
% 				all[grep("NEG", all\$gloss), ]}
%
% 	\item \textbf{Get all utterances whose speaker is younger than 4 years} \\[0.2cm]
% 		\texttt{SELECT utterances.corpus, utterances.session\und id\und fk, utterances.utterance, utterances.speaker\und label, speakers.age, utterances.translation \\
% 				FROM utterances \\
% 				INNER JOIN speakers \\
% 				ON utterances.corpus=speakers.corpus AND utterances.session\und id\und fk=speakers.\\session\und id\und fk AND utterances.speaker\und\ label=speakers.speaker\und label \\
% 				WHERE speakers.age\und in\und days<1460} \\[0.19cm]
% 		\texttt{merge(utterances, speakers, by=c("corpus", "session\und id\und fk", "speaker\und label")) -> all \\
% 				all <- all[complete.cases(all\$age\und in\und days), ] \\
% 				all[all\$age\und in\und days<1460, c("corpus", "session\und id\und fk", "utterance",\\"speaker\und label", "age", "translation")]}
% \end{itemize}


\section{Contributions}
\label{sec:Contributions}

The ACQDIV Corpus is the result of one and a half years of collaborative work. The main contributors are: 

\begin{itemize*}
	\item \textbf{Sabine Stoll} provided the idea, vision, and concept for the project
	\item \textbf{Robert Schikowski} devised the conceptual architecture of the corpus and supervised its realization 
	\item \textbf{Steven Moran} designed and built the IT infrastructure for the database and was responsible for its implementation
	\item \textbf{Cazim Hysi} wrote the metadata parsers and refactored the data parsers
	\item \textbf{Danica Pajović} helped to clean the corpora and to write the parsers
\end{itemize*}

\noindent We would also like to thank the following people: 

\begin{itemize*}
	\item \textbf{Laura Canedo} helped to clean the Yucatec corpus
	\item \textbf{John Gamboa} helped to clean the Inuktitut corpus
	\item \textbf{Andreas Gerster} helped to clean the CHAT corpora and to test the data parsers and worked on gloss and POS unification
	\item \textbf{Irene Ma} helped with role unification
	\item \textbf{Jekaterina Mažara} created the graphics in \autoref{sec:The language sample} and provided expertise on Russian
	\item \textbf{Süleyman Sabri Taşçı} helped to clean the Turkish corpus
	\item \textbf{Melanie Trüssel} helped with gloss and POS unification
	% \item Taras Zakharko
\end{itemize*}

Further, this project would not have been possible without the data provided by our external collaborators: 

\begin{itemize*}
	\item Shanley Allen for Inuktitut
	\item Julie Brittain for Cree
	\item Katherine Demuth for Sesotho
	\item Gaby Hermon for Indonesian
	\item Aylin Küntay for Turkish
	\item Barbara Pfeiler for Yucatec
	\item Yvan Rose for Cree
	\item Hannah Sarvasy for Nungon
\end{itemize*}

Even more people were involved in the creation of the original corpora. See \autoref{cha:Data sources} (subsection “Publication, accessibility, documentation” in each corpus-specific section) for detailed information on corpus authors and citation. 


\chapter{The dataset}
\label{cha:Overview of the dataset}

\section{The language sample}
\label{sec:The language sample}

The ACQDIV Corpus is a longitudinal language acquisition corpus that currently features ten diverse languages. The languages and the eleven corpora by which they are represented are shown below. 

\begin{table}[ht!]
	\centering
	\begin{tabular}{llll}
		\toprule
			\textbf{Language} 	& \textbf{ISO} 		& \textbf{Corpora}  & \textbf{Acronym} \\
								& \textbf{639-2}	&					& \\
		\midrule
			Cree 		& \texttt{crl} 		 & Corpus of the Chisasibi Child Language Acquisition Study 	& CCLAS \\
			Chintang 	& \texttt{ctn}  	 & Chintang Language Corpus (Language Acquisition subcorpus) 	& CLC \\
			% Dënë Sųłıné (\texttt{chp}) & Corpus of the Dënë Sųłıné Language Acquisition Study		& DESLAS \\
			Indonesian 	& \texttt{ind}  	 & MPI-EVA Jakarta Child Language Database 						& JCLD \\
			Inuktitut 	& \texttt{iku}  	 & Allen Inuktitut Child Language Corpus 						& AIC \\
			Japanese 	& \texttt{jpn}  	 & MiiPro Japanese Corpus 										& MPJC \\
						&					 & Miyata Japanese Corpus 										& MYJC \\
			Nungon	 	& \texttt{yuw}  	 & Sarvasy Nungon Corpus 										& SNC \\
			Russian 	& \texttt{rus}  	 & Stoll Russian Corpus 										& StRuC \\
			Sesotho 	& \texttt{sot}  	 & Demuth Sesotho Corpus 										& DSC \\
			Turkish 	& \texttt{tur}  	 & Koç University Longitudinal Language Development Database 	& KULLDD \\
			Yucatec 	& \texttt{yua}  	 & Pfeiler Yucatec Child Language Corpus 						& PYC \\
		\bottomrule
	\end{tabular}
	\caption{ACQDIV languages and corpora}
	\label{tab:ACQDIV languages and corpora}
\end{table}

The initial set of languages was selected from five clusters calculated via maximum diversity sampling \citep{Stoll_etal2013a} on the \href{http://www.autotyp.uzh.ch}{AUTOTYP database} and from the \href{http://wals.info}{World Atlas of Language Structures}. This guarantees maximal diversity with respect to a number of central typological parameters: % TODO SS ask Sabine which parameters precisely and if these were really identical to those in the paper. Research proposal says “a dozen variables”

\begin{itemize*}
	\item presence and nature of agreement and case marking
	\item word order
	\item degree of synthesis
	\item polyexponence and inflectional compactness of categories
	\item syncretism
	\item inflectional classes
\end{itemize*}

Below some examples are given to illustrate the diversity of the ACQDIV languages with respect to these parameters. 

Verbs in Japanese \Next[a] do not agree with any arguments, whereas Russian verbs \Next[b] agree with an S/A argument and Sesotho verbs \Next[c] agree with S or both A and P:

\ex.
	\ag. Okaa-san ga ue kara kore o otos-u. \\
		mother-HON NOM above ABL PROX ACC drop-NPST \\
		‘Mummy drops this from above.’ \source{MPJC, tom20010518.u1806}
	\bg. Kak ty mam-u obnima-ešʼ? \\
		how 2SG.NOM mother-ACC embrace.IPFV-PRS.2SG.S/A \\
		‘How do you embrace mummy?’ \source{StRuC, A00410909\und 594}
	\cg. Mme o-e-hlatsw-its-e. \\
		mother(I) NC.I.S/A-NC.IX.P-wash-PRF-IND \\
		‘Mother washed it.’ \source{DSC, tiid.u143}
	% \cg. Teboho o-ntsa-o-batl-a. \\
	% 	T. 3SG.A-PST.CONT-2SG.P-seek-SBJV \\
	% 	‘Teboho was looking for you.’ \source{DSC, tviia.u137}

Sesotho \Next[a] does not have case marking for core arguments. By contrast, Inuktitut always marks at least one argument in a transitive scenario, be it the A as in \Next[b] or the P as in \Next[c]. 

\ex.
	\ag. Fusi a-s-a-di-kh-il-e di-perekisi. \\
		F. NC.I.S/A-still-NC.I.S/A-NC.X.P-pick-PRF-IND NC.X-peach \\
		‘Fusi has already picked the peaches.’ \source{DSC, tviid.u207}
	\bg. Anaana-ngata aarqi-rataa-kainna-tanga. \\
		mother-POSS.3SG>3SG.ERG repair-RES-PST.RECENT-IND.3SG>3SG \\
		‘His mother has just fixed it.’ \source{AIC, JUP92WM.u1427}
	\cg. Himmi-mi taku-lau-llu? \\
		dog-INS see-POL-IMP.1DL.S \\
		‘Shall we see the dog?’ \source{AIC, SUP51WM.u733}

Another aspect in which the ACQDIV languages is synthesis. Indonesian \Next[a] is an example of a language with a fairly low degree of synthesis, whereas Cree \Next[b] belongs to one of the most genuinely polysynthetic languages of the world, featuring noun incorporation and polypartite stems: 

\ex.
	\ag. O, Ei lagi minum susu. \\
		oh E. more drink milk  \\
		‘Oh, Ei is drinking more milk.’ \source{JCLD, HIZ-1999-05-20.0556}
	\bg. Chi-wâp-iht-â-n â kâ-pushch-ishk-iw-â-t. \\
		2-light-by.head-TR.INAN.NON3-2SG>0 Q PVB.CONJ-put.on-by.foot-STEM-TR.ANIM-3SG>4SG \\
		‘You see? She was putting it on.’ \source{CCLAS, 19-A1-2006-08-16ms.u289}
	% \cg. Nâshtâpwâh chi-nânich-ishk-uw-i-n. \\
	% 	very.much 2-block-by.foot-STEM-TR.ANIM.NON3-2SG>1SG \\
	% 	‘You’re really in my way.’ \source{CCLAS, 18-A1-2006-07-12.u755}


Word orders differ radically between the ACQDIV languages. The most common word order, SVO, is e.g.\ found in Russian \Next[a]. Another common word order, SOV, is found in Turkish \Next[b]. Yucatec features (among other orders) the much less common VOS \Next[c].

\ex.
	\ag. Ja ne xoč-u salat! \\
		1SG.NOM NEG want.IPFV-NPST.1SG.S/A salad \\
		‘I don’t want salad!’ \source{StRuC, A05021006.68}
	\bg. Abla çay-ın-ı iç-sin. \\
		sister tea-POSS.3SG-ACC drink-OPT.3SG.S/A \\
		‘Let sister have her tea.’ \source{KULLDD, irem32\und 02sep03\und 02-00-16.u1825}
	\cg. T-u-náach in-kʼab le Osita-o. \\
		PFV-3.A-bite POSS.1SG-hand DET O.-DIST \\
		‘That Osita bit my hand.’ \source{PYC, SAN-1996-06-14.u181}

Russian has inflectional classes both in the nominal and verbal domains and often expresses a large number of categories by a single morpheme. The examples in \Next[a] and \Next[b] show the same bundle of grammatical functions (PL.GEN) expressed by very different morphs due to nominal inflection classes. By contrast, Chintang does not feature any inflectional classes, has less compact grammatical morphemes, and may even express a single function several times within a single word, as shown by the complex verb form in \Next[c]. 

\ex.
	\ag. Skolʼko produkt-ov papa nam privez? \\
		How.many product-PL.GEN dad.NOM 1PL.DAT bring.PFV.PST.M.SG.S/A \\
		‘How many products has dad brought us?’ \source{StRuC, A06830304.1293}
	\bg. Im mnogo konfet-Ø togda ne da-ešʼ. \\
		3SG.DAT much sweet-PL.GEN then NEG give.IPFV-NPST.2SG.S/A \\
		‘Don’t give him too many sweets then.’ \source{StRuC, A06930318.523}
	\cg. Athom u-patt-a-ŋ-s-a-ŋ-nɨ-ŋ=kha. \\
		before 3A-call-PST-1sP-PRF-PST-1sP-3p=NMLZ \\
		‘They had called me before.’ \source{CLC, CLDLCh2R02S01b.415}

The ACQDIV languages also feature very different kinds of syncretism. For instance, even though both Chintang and Inuktitut have an ergative that is used to mark agents in \Next[a] and \NNext[a], the Chintang ergative also serves (among others) to mark causes \Next[b], whereas the Inuktitut ergative is also (again among others) used as a genitive \NNext[b]:

\ex.
	\ag. U-madum-ŋa=ta khur-u-gond-o-ko. \\
		POSS.3SG-aunt-ERG=FOC carry-3[s]P-around-3[s]P-IND.NPST[.3sA] \\
		‘His aunt carries her around.’ \source{CLC, CLDLCh3R03S04.0496}
	\bg. Kok-ŋa=ta meʔ-no=kha=lo na. \\
		rice-ERG=FOC be.big-IND.NPST=NMLZ=SURP TOP \\
		‘He’s so big because of the rice.’ \source{CLC, CLDLCh2R04S04.438}

\ex.
	\ag. Ii, nuka-pi-ppit atu-ruma-mmauk. \\
		no younger\und same\und sex\und sibling-DIM-POSS.2SG>3SG.ERG use-want-CAUS.3SG>3SG \\
		‘No, (it’s because) your sister wants to use it.’ \source{AIC, MAE14WM.u206}
	\bg. Ataata-ppit kami-alu-alu-ni sanarvat-ti-gia-lau-rit. \\
		father-POSS.2SG>3SG.ERG boot-big-big-INS put-CAUS-INCEP-POL-IMP.2SG.S \\
		‘Put your father’s big, big boots somewhere.’ \source{AIC, JUP51WM.0593}

% JUP92WM.cha-*JUP:	Una kinau tasijuanga  ?
% JUP92WM.cha-%eng:	Whose sock is this  ?
% JUP92WM.cha:%xmor:	DR|u^here&SG_ST+DI|na^ABS_SG WH|kina^who+NI|up^ERG_SG NR|tasijuaq^sock+NI|nga^ABS_3Ssg?
%
% JUP11WM.cha-*JUP:	Angaapiit tigujaalua  .
% JUP11WM.cha:%eng:	Your uncle took it  .
% JUP11WM.cha-%xmor:	NR|angak^maternal_uncle+NI|ppit^ERG_2Ssp VR|tigu^take+NZ|jaq^NZ_PASS+NN|AUG|aluk^EMPH+NI|nga^ABS_3Ssg.

\newpage

\section{Amount of data}
\label{sec:Amount of data}

The subcorpora of the ACQDIV Corpus vary considerably in size. \autoref{fig:Amount of data in the ACQDIV subcorpora} shows how much utterances, words, and morphemes there are in each. 

\begin{figure}[ht!]
	\centering
	\includegraphics[scale=0.83]{pics/tokens_per_corpus.pdf}
	\caption{Amount of data in the ACQDIV subcorpora}
	\label{fig:Amount of data in the ACQDIV subcorpora}
\end{figure}

\section{Sampling for speakers and periods}
\label{sec:Sampling for speakers and periods}

The ACQDIV corpus focuses on the acquisition period from the beginning of the 2nd to the end of the 3rd year, and this is the period where the most linguistically diverse data are available. However, some subcorpora start at a much younger age (the lower boundary being some Chintang and Turkish children where recordings startet around half a year) and end considerably later (the extreme here is Indonesian, where the recordings for one child start at around 4;6 and end around 8;8). 

The subcorpora also vary with regard to the number of target children that were recorded. The Cree subcorpus only features a single target child (and a single session for one other child), whereas the Indonesian and the Turkish corpus both feature eight target children. 

The differences between the corpora are shown in summary fashion in \autoref{fig:Recording periods}. 

\begin{figure}
	\centering
	\includegraphics[scale=0.8]{pics/age_spans_acqdiv_corpora.pdf}
	\caption{Children and recording periods in the ACQDIV Corpus}
	\label{fig:Recording periods}
\end{figure}

There is less variation in the intervals between recordings. In most corpora the recordings for one child took place every other week or once a month, and only two of the corpora have an even higher frequency rhythm with weekly recordings. The sessions vary in length both within and across corpora, ranging from half an hour to four hours. 
% Russian and Japanese Miyata have the densest rhythm with weekly recordings whereas sessions in Indonesian, Turkish and Yucatec took place every other week and Chintang, Inuktitut, Japanese MiiPro and Sesotho have monthly recordings. Cree has recordings every two to three weeks with some gaps.

More details on temporal sampling can be found in the corpus-specific sections of \autoref{cha:Data sources}. 

\section{Annotation layers and data gaps}
\label{sec:Annotation layers and data gaps}

The ACQDIV Corpus is richly annotated. Each of the three principal levels \textendash\ utterances, words, and morphemes \textendash\ has dedicated additional annotations in addition to a transcription. The list below only shows a few frequently used and widely implemented types of annotations; for details see the section on the \hyperref[sec:Structure of the corpus]{structure of the corpus}. 

\begin{itemize*}
	\item \textbf{utterances:} speaker, addressee, translation (usually into English), time stamps for start end end in associated media
	\item \textbf{words:} actual and target word, part of speech of the stem
	\item \textbf{morphemes:} gloss (original or unified across corpora), part of speech (original or unified)
\end{itemize*}

These data are associated with metadata, the two principal levels here being sessions and speakers: 

\begin{itemize*}
	\item \textbf{sessions:} recording date, media file
	\item \textbf{speakers:} label, name, age (as Y;M.D or in days), gender, role
\end{itemize*}

Note that the only thing that all subcorpora have in common is that all sessions have been transcribed and that morphological analyses (including glosses) are at least available for some sessions or utterances. All other annotation layers mentioned above are widespread but not always available. The most important gaps can be summarized as follows:

\begin{itemize}
	\item One corpus, Japanese Miyata, does not have systematic \textbf{transcriptions} for utterances by the mother, which present the overwhelming
		majority of non-target-child speech. This corpus is therefore not suitable for the study of child-surrounding speech. 
	\item Both Japanese corpora and the Russian corpus have not been \textbf{translated} into any language. For Yucatec only Spanish translations are available. 
	\item Almost half of the corpora do not specify addressees: this is the case for Cree, Indonesian, Sesotho, and Yucatec. Chintang features addressee coding
		only in a subset of the complete corpus. 
	\item Turkish and Yucatec do not have any \textbf{time stamps}. The Russian corpus only has time stamps in a few sessions (2\% of the Toolbox files which are incorporated
		into the ACQDIV Corpus; 14\% in a parallel set of ELAN files which is currently not part of the ACQDIV Corpus). The Japanese Miyata corpus also has considerable gaps
		\textendash\ the roughly 36\% of linked files all stem from a single target child (which they cover completely). Indonesian and Inuktitut 
		are comprehensively time-linked (with a few gaps in Inuktitut, around 87\% of linked sessions) but only mark the beginning and not the 
		end of utterances, so durations cannot be calculated. Only Chintang, Cree, Japanese MiiPro, and Sesotho have complete time stamps 
		for both utterance boundaries. 
	\item Some corpora contain considerable gaps with respect to \textbf{segmentation}, \textbf{glosses}, and \textbf{parts of speech}. For Cree, only the Ani subcorpus 
		has been morphologically analyzed, and even there analyses are mainly available for the child’s utterances. Likewise, Inuktitut completely lacks analyses for
		some sessions; moreover, many adult utterances in other sessions have not been analysed. The Turkish corpus has complete analyses for all participants 
		in the sessions of three children but almost nothing for the remaining five children. The corpus team is presently exploring the possibility
		of using an automatic parser. The situation is similar in Yucatec, although there are no plans for automatic analysis in this case.  
		In Chintang, a small part of the data (about 80 sessions) have been analyzed automatically and thus have lower overall glossing quality. The majority
		of the Chintang sessions; however, have been analyzed manually.
	\item While all corpora have glosses, some are of limited use because they comply with \textbf{CHAT glossing conventions} where stems are only given
		in their phonological form (without a functional label) and affixes are only given as glosses (without specifying the phonological form). 
		Thus, a word like German \emph{Tage} is not segmented to \emph{Tag -e} and then assigned two labels (“day -PL”) but is glossed as “Tag -PL”. 
		This makes it difficult to infer the meaning of a word form from the glosses and makes it impossible to distinguish automatically between 
		homophonous stems or affixes with the same label. Conventions of this kind are fully implemented in the two Japanese corpora and in Turkish. 
		In Yucatec, the phonological form is given for all types of morphemes but there are still no functional labels for stems. 
	\item The Russian corpus does not feature \textbf{segmentation}. Glosses cover all functional aspects of word forms but are concatenated into a
		single string. Accordingly, the \texttt{morphemes} table does not contain real morphemes but full word forms for Russian. 	 
	\item Indonesian does not contain \textbf{part-of-speech tags}. Dummy tags are inserted during parsing to differentiate between stems and prefixes/suffixes, 
		but more specific information is not available. 
\end{itemize}

For more details on which layers are available for which corpus, also see the tables in the sections on the database tables \hyperref[subsec:Table utterances]{\texttt{utterances}}, \hyperref[subsec:Table words]{\texttt{words}}, and \hyperref[subsec:Table morphemes]{\texttt{morphemes}}. 


\chapter{The corpus}
\label{cha:Details of the corpus}

\section{Getting access and adding data}
\label{sec:Access to the corpus}

The ACQDIV Corpus may be described as semi-open. Access may be gained by contributing data (for which see below) or by collaborating with the ACQDIV project. The detailed access regulations are described in the \href{http://www.acqdiv.uzh.ch/dam/jcr:c7318751-f531-43a8-9dbd-b48eee950a4c/terms_of_use_for_the_acqdiv_corpus.pdf}{Terms of Use}, which are available online at the \href{http://www.acqdiv.uzh.ch/en/resources.html}{ACQDIV website}. The core points can be summarized as follows: 

\begin{itemize}
	\item The ACQDIV Corpus is a resource to be kept separate from the original data it builds on since it incorporates extensive efforts to clean, unify, and enrich the original data.
	\item The ACQDIV PI (Sabine Stoll, UZH) decides about access to and distribution of the data in the ACQDIV Corpus. On the other hand, the owners of the original data keep all their rights to these data. 
	\item All resources used in a publication within the ACQDIV framework (including original data) must be properly cited.
	\item In addition, the developers of the ACQDIV Corpus as well as of any non-public corpora included therein must be asked if they want to become co-authors of publications in which these corpora are used. The contribution of each author (e.g.\ resource development vs.\ active contribution to research) must be specified. 
\end{itemize}

\noindent The ACQDIV Corpus should be cited as follows: 

\begin{quote}
Moran, Steven, Robert Schikowski, Danica Pajović, Cazim Hysi \& Sabine Stoll. 2015. \emph{The ACQDIV Corpus: a comparative longitudinal language acquisition corpus.} Version 1.0. % Moran_etal2016a
\end{quote}

The publication year should correspond to the cited release. New releases will be published after major changes at intervals yet to be fixed. 

Currently the master version of the corpus is stored on the server of the Department of Comparative Linguistics at UZH, where the ACQDIV project is based. New corpora can be added at any time by request to the PI. 

% TODO
% release rhythm
% link to documents online: terms of use etc.
% SM: how to add new corpora (and what that entails for licensing purposes)


\section{Conceptual architecture}
\label{sec:Architecture}

Conceptually, the ACQDIV Corpus is a tree with five levels below the root: 

\begin{itemize*}
	\item subcorpus
	\item session
	\item utterance
	\item word 
	\item morpheme
\end{itemize*}

A session is defined as a continuous stretch of time which contains spoken communication and whose boundaries are set by the applied recording scheme. Sessions may be instantiated by various types of files such as media, transcripts, or metadata files in the original subcorpora. While the original subcorpora consist of several sessions, where each in turn may or may not be instantiated by several files, all subcorpora and all their session-related data are contained in a single file in the ACQDIV Corpus. 

Each level has one or several properties that can be searched for. To name a few examples, subcorpora have a language, sessions have recording dates, utterances may have a phonetic transcription, words may have an actual and a target form, and morphemes may have a gloss. These properties will henceforth be called tiers. Each tier is described in detail in \autoref{sec:Structure of the corpus} below. 

In addition to the corpus tree, there are two metadata tables (one for session-level metadata, one for participant-level metadata). These tables are linked to the corpus via session IDs and participant codes, respectively. 

% The following rules apply when mapping tiers from different levels to this structure:
%
% \begin{itemize}
% 	\item When the tier is associated with the row unit, each cell in the relevant column contains the value for that unit.
% 		For instance, glosses are associated with the morpheme level, so in a table where each row is a morpheme, each cell in the gloss column contains one gloss.
% 	\item When the tier is associated with a higher unit, each cell in the relevant column repeats the higher unit’s value.
% 		For instance, utterance translations are associated with the utterance level, so in the word table each cell in the translation column
% 		contains a translation for the complete utterance, and this value is repeated for all words belonging to the same utterance.
% 	\item When the tier is associated with a lower unit, the corresponding column is not given at all. For instance, parts of speech are associated with
% 		morphemes, so in the clause table there is no part of speech column.
% \end{itemize}


\section{Format}
\label{sec:Format}

The abstract structure sketched above is currently implemented as an SQLite database. The database can be mapped to various output formats as required. Currently, the data are regularly exported as an R data object \citep{RCoreTeam2015}, whose dataframes largely mirror the tables of the database. 

There are many database GUIs that can be used to conveniently interact with the SQLite version. One that the ACQDIV team has made good experiences with and that is free to download is the DB Browser for SQLite, available from \url{http://sqlitebrowser.org/}. R is freely available from \url{https://www.r-project.org/}. Note that in either environment the corpus may take some time to load, depending on your system and computer. We recommend opening the database locally to save working memory. 

The data sources for the subcorpora are encoded in diverse formats \textendash\ see \autoref{cha:Data sources} for details. 

Note that the original subcorpora also contain media files (audio and/or video, mostly digitized). The ACQDIV Corpus does not include these files to protect the children’s privacy \textendash\ sensitive information is much harder to remove or anonymize in media files than in text files. However, the names of the original media files are provided in the \texttt{sessions} metadata table. 


\section{Structure of the corpus}
\label{sec:Structure of the corpus}

\subsection{Overview and ERD}

As a relational database, the ACQDIV Corpus is constituted by several tables and fields (also called columns below). The tables correspond roughly to the corpus levels described \hyperref[sec:Architecture]{above}: 

\begin{itemize*}
	\item \texttt{sessions}: session-level metadata
	\item \texttt{speakers}: speaker-level metadata as given in individual sessions (i.e.\ one row = one speaker-session tuple)
	\item \texttt{uniquespeakers}: speaker-level metadata that can be specified independently of sessions
	\item \texttt{utterances}: utterances with their annotations, linkable to \texttt{sessions} and \texttt{speakers}
	\item \texttt{words}: words with their annotations, linkable to \texttt{utterances}
	\item \texttt{morphemes}: morphemes with their annotations, linkable to \texttt{utterances} % and later also words?
%	\item \texttt{warnings}: warnings for various levels % obsolete
\end{itemize*}

Each table has several fields, which correspond to what would be called a tier in a format more oriented towards running text. The names and detailed contents of the fields are described in the sections below. Two naming conventions are used across tables: 

\begin{itemize*}
	\item Foreign keys have the suffix “\und fk”. 
	\item The database often contains both the original data and a postprocessed version in separate columns. In such cases, the field containing the original data
		is marked by the suffix “\und raw” (e.g.\ \texttt{gloss} vs.\ \texttt{gloss\und raw}). 
\end{itemize*}

Joining information from several tables requires a key field with shared values. Currently, composite keys have to be used in all relevant cases, i.e.\ the link between tables has to be established by several fields as follows: 

\begin{itemize}
	\item The tables \texttt{utterances} and \texttt{sessions} are linked via the combination of the fields \texttt{corpus} and \texttt{session\und id(\und fk)}.  
		The composite key is required because session IDs may not be unique across corpora. 
	\item The tables \texttt{utterances} and \texttt{speakers} are linked via the combination of the fields \texttt{corpus}, \texttt{session\und id(\und fk)}, 
		and \texttt{speaker\und label}. The composite key is required because speaker labels may not be unique across corpora and because age and roles 
		can change with the session. 
	\item The tables \texttt{words} (or \texttt{morphemes}) and \texttt{utterances} are linked via the combination of the fields \texttt{corpus} and 
		\texttt{utterance\und id(\und fk)}. The composite key is required because utterance IDs may not be unique across corpora. 
\end{itemize}

\autoref{fig:Entity-relationship diagram of the ACQDIV Corpus} shows an ERD of the database.\footnote{A few columns added later on in the project might be missing, but the basic picture and especially the relations between tables have remained the same.}

\begin{landscape}
\begin{figure}
	\centering
	\includegraphics[scale=0.6]{pics/ERD.png}
	\caption{Entity-relationship diagram of the ACQDIV Corpus}
	\label{fig:Entity-relationship diagram of the ACQDIV Corpus}
\end{figure}
\end{landscape}

\subsection{Table \texttt{sessions}}
\label{subsec:Table sessions}

\begin{longtable}{lp{.5\linewidth}p{.2\linewidth}}
	\toprule
		\textbf{Column} & \textbf{Content} 	& \textbf{Origin} \\
	\midrule
	\endhead
		
	\bottomrule\\[-0.15cm]
	\caption{Columns of the table \texttt{session}}
	\endfoot
	
	
		\texttt{id}				& an automatically generated numeric ID for the session & postprocessing \\
		\texttt{source\und id}	& the name of the transcript file associated with the session. Note that source IDs are sometimes not unique across all corpora, 
			 					  so they are of limited use for identifying sessions. & data \\
		\texttt{corpus}			& the name of the corpus the session belongs to & data \\
		\texttt{language}		& the language of the corpus. While normally one language corresponds to one corpus, some languages may be represented by several corpora & postprocessing \\
		\texttt{date} 			& the recording date for the session & data \\ 
		\texttt{target\und child\und fk}  & the unique ID of the target child of the session, linking to the table \texttt{uniquespeakers} & postprocessing \\ 
		% \texttt{situation} 		& informal notes on the situation at the time of recording & data \\
		% \texttt{genre} 			& the genre of the recording & data \\ it, right?
		% \texttt{media} 			& the file name of the associated primary media file & data \\
		% \texttt{media\und type} & the type of the associated primary media file (mostly video, sometimes audio) & data \\[-0.3cm]
		\label{tab:Table sessions}
\end{longtable}


\subsection{Table \texttt{speakers}}
\label{subsec:Table speakers}

\begin{longtable}{lp{.5\linewidth}p{.2\linewidth}}
	\toprule
		\textbf{Column} & \textbf{Content} 	& \textbf{Origin} \\
	\midrule
	\endhead
	
	\bottomrule\\[-0.15cm]
	\caption{Columns of the table \texttt{speakers}}
	\endfoot
	
		\texttt{id}				& an automatically generated numeric ID for the speaker-session combination & postprocessing \\
		\texttt{uniquespeaker\und id\und fk} & the unique ID of the speaker independently of sessions, linking to the table \texttt{uniquespeakers} & postprocessing \\
		\texttt{session\und id\und fk} & the ID of the session the speaker appeared in, linking to the table \texttt{session} & data \\
		\texttt{corpus}			& the name of the corpus the utterance belongs to & data \\
		\texttt{language}		& the language of the corpus & postprocessing \\
		\texttt{speaker\und label} & a code used to identify the speaker within the current corpus & data \\
		\texttt{name} 			& the full name of the speaker & data \\
		\texttt{age} 			& the age of the speaker at the time when the current session was recorded. The age may be given in years
			 					  or (especially for children) in the formats Y;M.D or Y;M & postprocessing \\
		\texttt{age\und in\und days} & the equivalent of the age in days & postprocessing \\
		\texttt{age\und raw} 	& the age as given in the original data. This may be formally slightly different from the standardized form
			 					  given in \texttt{age} & data \\
		\texttt{gender} 		& the gender of the speaker. The only allowed values are “female” and “male”. & postprocessing \\
		\texttt{gender\und raw} & the gender as given in the original data, sometimes slightly different from the processed form & data \\
		\texttt{role} 			& the role performed by the speaker in the present recording. Because of the diversity of the ACQ\-DIV corpora, 
			 					  this concept covers both kinship terms (given in relation to the target child) and roles related to the 
								  setting (e.g.\ speaker, recorder, assistant). See \hyperref[subsec:Roles and macroroles]{below} for 
								  a list of the possible values of this field. & postprocessing \\
		\texttt{macrorole} 		& this column only allows four values: Target\und Child, Child (any other children younger than or 12 years old), 
								  Adult (older than 12 years), Unknown. Its purpose is to make the most basic age- and role-related information
								  available for all speakers, even when the precise age and/or role are not known. & postprocessing \\
		\texttt{role\und raw} 	& the role as given in the original data. This is often slightly different from the standardized form in 
			  					  \texttt{role} because of terminological differences (“target child” vs.\ “focus child” etc.) & data \\
		\texttt{languages\und spoken} & a space-separated list of all languages the speaker is able speak, given in the form of ISO 639-2 codes & data \\
		\texttt{birthdate} 		& the birthdate of the speaker in the format YYYY-MM-DD & data \\[-0.3cm] 
	\label{tab:Table speakers}
\end{longtable}


\subsection{Table \texttt{uniquespeakers}}
\label{subsec:Table uniquespeakers}

The corpora themselves do not always make it clear which of the speaker labels they use are unique, so this table requires some additional explanation. 
In the CHAT-based corpora, different speakers with identical speaker labels occur regularly because (different) target children always have the code CHI and their mothers are always referred to as MOT. Thus, speaker labels alone are not sufficient for identifying unique speakers. The \texttt{uniquespeakers} table therefore uses unique combinations of speaker labels, full names, and birthdates (if available) to achieve this. % SM: is "corpus" actually included in the logic? Birthdate is often "Unknown", so we fall back to code+name for less important speakers; couldn't they be ambiguous across corpora? -> nope, not included

On the other hand, there is also the less frequent case of a single speaker being referred to by different labels (and/or names and birthdates) because of gaps or mistakes in the metadata. These case are currently ignored, i.e.\ these cases will appear as different speakers in the \texttt{uniquespeakers} table. 

\begin{longtable}{lp{.5\linewidth}p{.2\linewidth}}
	\toprule
		\textbf{Column} & \textbf{Content} 	& \textbf{Origin} \\
	\midrule
	\endhead
	
	\bottomrule\\[-0.15cm]
	\caption{Columns of the table \texttt{uniquespeaker}}
	\endfoot
	
		\texttt{id} 		& an automatically generated numeric ID for the speaker & postprocessing \\
		\texttt{speaker\und label} & a code used to identify the speaker within the associated corpus & data \\
		\texttt{corpus} 	& the corpus the speaker appears in & data \\ 
		\texttt{name}		& the full name of the speaker & data \\
		\texttt{birthdate} 	& the birthdate of the speaker in the format YYYY-MM-DD & data \\ 
		\texttt{gender} 	& the gender of the speaker & postprocessing \\[-0.3cm]
	\label{tab:Table uniquespeakers}
\end{longtable}


\subsection{Table \texttt{utterances}}
\label{subsec:Table utterances}

\begin{longtable}{lp{.5\linewidth}p{.2\linewidth}}
	\toprule
		\textbf{Column} & \textbf{Content} 	& \textbf{Origin} \\
	\midrule
	\endhead
	
	\bottomrule\\[-0.15cm]
	\caption{Columns of the table \texttt{utterance}}
	\endfoot

		\texttt{id}				& an automatically generated numeric ID for the utterance & postprocessing \\
		\texttt{session\und id\und fk} & the ID of the session the utterance belongs to, linking to the table \texttt{session} & data \\
		\texttt{source\und id} 	& the ID of the utterance in the original data & data \\
		\texttt{corpus}			& the name of the corpus the utterance belongs to & data \\
		\texttt{language}		& the language of the corpus & postprocessing \\
		\texttt{speaker\und id\und fk} & the ID of the speaker who produced the utterance, linking to the table \texttt{speakers} & postprocessing \\
		\texttt{uniquespeaker\und id\und fk} & the ID of the \hyperref[subsec:Table uniquespeakers]{unique speaker} who produced the utterance, linking to the table \texttt{uniquespeakers} & postprocessing \\
		\texttt{speaker\und label} & a code which uniquely identifies the speaker of an utterance within a corpus and presents a link to the tables \texttt{speaker} and \texttt{uniquespeaker} & data \\
		\texttt{addressee}		& a code for the participant adressed by the speaker of an utterance & data \\
		\texttt{utterance}		& an orthographic representation of the utterance (created by concatenating the single words if no separate representation is available; cleaned of punctuation marks) & postprocessing \\
		\texttt{utterance\und raw} & the original orthographic representation of an utterance (created by concatenating the single words if no separate representation is available &  \\ 
		% \texttt{phonetic}		& a phonetic transcription of the actual utterance, mostly in IPA & data \\ % only four corpora have this: Chintang, Cree, Indonesian, Yucatec
		\texttt{translation}	& a free translation of the utterance (mostly English but Spanish for Yucatec) & data \\
		\texttt{sentence\und type} &	broad sentence types, the most frequent values being \texttt{default}, \texttt{question} and \texttt{exclamation}. This may be taken directly from the data or inferred on the base of sentence delimiters. & data or postprocessing \\
		\texttt{childdirected} & \texttt{1} for utterances directed to a target child,
								 \texttt{0} for all others (including unknown addressees) & data or postprocessing \\
		\texttt{start}			& the point in time in an associated media file where the utterance starts; format HH:MM:SS. & postprocessing \\ 
		\texttt{start\und raw} 	& like \texttt{start} but not unified to HH:MM:SS & data \\
		\texttt{end}			& the point in time in an associated media file where the utterance ends; format HH:MM:SS. & postprocessing \\
		
		\texttt{end\und raw}	& like \texttt{end} but not unified to HH:MM:SS & data \\
		\texttt{morpheme}		& all morphemes contained in an utterance, separated by spaces & postprocessing (concatenated morphemes) \\
		\texttt{gloss\und raw}  & all glosses contained in an utterance, separated by spaces & postprocessing (concatenated morphemes) \\
		\texttt{pos\und raw}	& all part-of-speech tags contained in an utterance, separated by spaces & postprocessing (concatenated morphemes) \\
		\texttt{comment}		& any comments on the utterance. This tier merges several tiers that are separated in some of the subcorpora 
									but mostly overlap due to inconsistent usage: actions accompanying an utterance, background situation, ethnographic comments, comments on grammar, generic comments. & data \\
		\texttt{warning}	& warnings about formal problems on the utterance level & parsing \\[-0.3cm]
			
	\label{tab:Table utterances}
\end{longtable}

% List of obsolete tiers/columns

% \item \textbf{argument_coding:} various types of syntactic coding. This tier is presently not passed on from JSON to R because it is only present in a few corpora.
% \item \textbf{error\und coding:} free coding of errors made by children. This tier is presently not passed on from JSON to R because it is only present in a few corpora.
% \item \textbf{gestures:} free coding of gestures accompanying speech. This tier is presently not passed on from JSON to R because it is only present in a few corpora.
% \item \textbf{information_structure:} free coding of information structure. This tier is presently not passed on from JSON to R because it is only present in a few corpora.
% \item \textbf{nepali:} a free Nepali translation of the utterance
% \item \textbf{spanish:} a free Spanish translation of the utterance
% \item \textbf{syntax\und coding:} various types of syntactic coding. This tier is presently not passed on from JSON to R because it is only present in a few corpora.

The table below shows which of the utterance columns are regularly filled in which of the corpora. 

% \small
\begin{longtable}{lccccccccccc}	
	% \toprule
	\textbf{tier} 				& \rot{\textbf{CLC} (ctn)} & \rot{\textbf{CCLAS} (crl)} & \rot{\textbf{JCLD} (ind)} & \rot{\textbf{AIC} (ike)} & \rot{\textbf{MPJC} (jpn)} & \rot{\textbf{MYJC} (jpn)} & \rot{\textbf{StRuC} (rus)} & \rot{\textbf{DSC} (sot)} & \rot{\textbf{KULLDD} (tur)} & \rot{\textbf{PYC} (yua)} & \rot{\textbf{SNC} (yuw)}\\
	\midrule
	\endhead
	
	\bottomrule\\[-0.15cm]
	\caption{Presence of columns in the table \texttt{utterances}}
	\endfoot
	
	\small
	
	addressee 					&  (+) 		& - 	& - 		& + 		 & (+) 	  & (+) 	& -    	 & -  		& + 	  & - 	 & -  \\
%	argument\und coding 		&  - 		& - 	& - 		& +  		 & - 	  & -  		& - 	 & -  		& - 	  & - 	 & -  \\
	childdirected 				&  (+) 		& - 	& - 		& +  		 & + 	  & +  		& - 	 & -  		& + 	  & - 	 & -  \\
	comment 					&  + 		& + 	& + 		& +  		 & + 	  & +  		& + 	 & +  		& + 	  & + 	 & +  \\
	corpus	 					&  + 		& + 	& +  		& +  		 & + 	  & +  		& + 	 & +  		& + 	  & + 	 & +  \\
	end	 						&  + 		& + 	& - 		& -  		 & + 	  & (+)  	& (+) 	 & +  		& - 	  & - 	 & +  \\
	end\und raw	 				&  + 		& + 	& - 		& -  		 & + 	  & (+)  	& (+) 	 & +  		& - 	  & - 	 & +  \\
%	error\und coding 			&  - 		& - 	& - 		& +  		 & - 	  & +  		& - 	 & -  		& + 	  & - 	 & -  \\
	% gestures 					&  - 		& - 	& - 		& -  		 & - 	  & +  		& - 	 & -  		& + 	  & - 	 & -  \\
	id		 					&  + 		& + 	& +  		& +  		 & + 	  & +  		& + 	 & +  		& + 	  & + 	 & +  \\
%	information\und structure 	&  - 		& - 	& -  		& -  		 & + 	  & -  		& - 	 & -  		& - 	  & - 	 & -  \\
%	orthographic 				&  - 		& - 	& -  		& -  		 & + 	  & +  		& - 	 & -  		& - 	  & - 	 & -  \\
	language					&  + 		& + 	& +  		& +  		 & + 	  & +  		& + 	 & +  		& + 	  & + 	 & +  \\
%	phonetic 					&  - 		& + 	& +  		& -  		 & - 	  & -  		& + 	 & -  		& - 	  & + 	 & +  \\
	sentence\und type 			&  - 		& + 	& -  		& +  		 & + 	  & +  		& - 	 & +  		& + 	  & + 	 & +  \\
	speaker\und label 			&  + 		& + 	& +  		& +  		 & + 	  & +  		& + 	 & +  		& + 	  & + 	 & +  \\
	session\und id\und fk	 	&  + 		& + 	& +  		& +  		 & + 	  & +  		& + 	 & +  		& + 	  & + 	 & +  \\
	start		 				&  + 		& + 	& +  		& +  		 & + 	  & (+)  	& (+) 	 & +  		& - 	  & - 	 & +  \\
	start\und raw 				&  + 		& + 	& +  		& +  		 & + 	  & (+) 	& (+) 	 & +  		& - 	  & - 	 & +  \\
%	syntax\und coding 			&  - 		& - 	& -  		& +  		 & - 	  & -  		& - 	 & -  		& - 	  & - 	 & -  \\
	translation					&  + 		& + 	& + 		& +  		 & - 	  & -  		& - 	 & +  		& + 	  & + 	 & +  \\
	uniquespeaker\und id\und fk &  + 		& + 	& +  		& +  		 & + 	  & +  		& + 	 & +  		& + 	  & + 	 & +  \\
	utterance		 			&  + 		& + 	& +  		& +  		 & + 	  & +  		& + 	 & +  		& + 	  & + 	 & +  \\
	utterance\und id 			&  + 		& + 	& +  		& +  		 & + 	  & +  		& + 	 & +  		& + 	  & + 	 & +  \\
	utterance\und raw 			&  + 		& + 	& +  		& +  		 & + 	  & +  		& + 	 & +  		& + 	  & + 	 & +  \\
	warning 					&  + 		& + 	& + 		& + 		 & + 	  & + 		& + 	 & + 		& + 	  & + 	 & +  \\
\end{longtable}

\subsection{Table \texttt{words}}
\label{subsec:Table words}

\begin{longtable}{lp{.5\linewidth}p{.2\linewidth}}
	\toprule
		\textbf{Column} & \textbf{Content} 	& \textbf{Origin} \\
	\midrule
	\endhead
	
	\bottomrule\\[-0.15cm]
	\caption{Columns of the table \texttt{words}}
	\endfoot
	
		\texttt{id}	 				& an automatically generated numeric ID for the word & postprocessing \\
		\texttt{utterance\und id\und fk} & the ID of the utterance the word belongs to, linking to the table \texttt{utterances} & data \\
		\texttt{session\und id\und fk} & the ID of the session the word belongs to, linking to the table \texttt{sessions} & data \\
		\texttt{corpus} 			& the name of the corpus the word belongs to & data \\ 
		\texttt{language}			& the language of the corpus & postprocessing \\
		\texttt{word\und language}	& the language of the stem of a word; equals the corpus language by default & data \\							  
		\texttt{word}				& an orthographic representation of a word. When both actual and target forms are available 
									  (see \autoref{subsec:Actual and target fields}), this is the actual word; otherwise it is the only available form. 
									  See \texttt{word\und actual} and \texttt{word\und target} for more precisely specified (but often empty) word forms. & data \\
		\texttt{word\und actual}	& the word form the speaker actually produced; may be empty when only the target form is known & data \\
		\texttt{word\und target}	& the word form the speaker intended to produce; may be empty when only the actual form is known & data \\
		\texttt{pos}				& the standardized part-of-speech tag of the stem of the word & postprocessing \\
		\texttt{pos\und ud}       &  the universal part-of-speech tag\footnote{\url{http://universaldependencies.org/u/pos/}} of the stem of the word & postprocessing \\
		\texttt{warning}			& warnings about formal problems on the word level, e.g.\ missing or broken glosses & parsing \\

	\label{tab:Table words}
\end{longtable}

The table below shows which of these columns are regularly filled in which of the corpora. 

% \small
\begin{longtable}{lccccccccccc}	
	% \toprule
	\textbf{tier} 				& \rot{\textbf{CLC} (ctn)} & \rot{\textbf{CCLAS} (crl)} & \rot{\textbf{JCLD} (ind)} & \rot{\textbf{AIC} (ike)} & \rot{\textbf{MPJC} (jpn)} & \rot{\textbf{MYJC} (jpn)} & \rot{\textbf{StRuC} (rus)} & \rot{\textbf{DSC} (sot)} & \rot{\textbf{KULLDD} (tur)} & \rot{\textbf{PYC} (yua)} & \rot{\textbf{SNC} (yuw)}\\
	\midrule
	\endhead
	
	\bottomrule\\[-0.15cm]
	\caption{Presence of columns in the table \texttt{words}}
	\endfoot
	
	corpus						&  + 		& + 	& +  		& +  		 & + 	  & +  		& + 	 & +  		& + 	  & + 	 & + \\
	id		 					&  + 		& + 	& +  		& +  		 & + 	  & +  		& + 	 & +  		& + 	  & + 	 & + \\
	language					&  + 		& + 	& +  		& +  		 & + 	  & +  		& + 	 & +  		& + 	  & + 	 & + \\
	session\und id\und fk		&  + 		& + 	& +  		& +  		 & + 	  & +  		& + 	 & +  		& + 	  & + 	 & + \\
	utterance\und id\und fk		&  + 		& + 	& +  		& +  		 & + 	  & +  		& + 	 & +  		& + 	  & + 	 & + \\
	warning 					&  + 		& + 	& + 		& + 		 & + 	  & + 		& + 	 & + 		& + 	  & + 	 & + \\
	word	 					&  + 		& + 	& +  		& +  		 & + 	  & +  		& + 	 & +  		& + 	  & + 	 & + \\
	word\und actual 			&  + 		& + 	& + 		& +  		 & + 	  & +  		& + 	 & +  		& + 	  & (+)  & - \\
	word\und target 			&  - 		& + 	& + 		& +  		 & + 	  & +  		& - 	 & +  		& + 	  & + 	 & + \\
	pos 						&  + 		& + 	& (+)  		& +  		 & + 	  & +  		& + 	 & +  		& + 	  & +    & + \\
        pos\und ud 						&  + 		& + 	& (+)  		& +  		 & + 	  & +  		& + 	 & +  		& + 	  & +    & + \\

\end{longtable}
\normalsize

\subsection{Table \texttt{morphemes}}
\label{subsec:Table morphemes}


\begin{longtable}{lp{.5\linewidth}p{.2\linewidth}}
	\toprule
		\textbf{Column} & \textbf{Content} 	& \textbf{Origin} \\
	\midrule
	\endhead
	
	\bottomrule\\[-0.15cm]
	\caption{Columns of the table \texttt{morphemes}}
	\endfoot

		\texttt{id}	 			& an automatically generated numeric ID for the morpheme & postprocessing \\
		
		\texttt{word\und id\und fk} & the ID of the word the morpheme belongs to, linking to the table \texttt{words} & data \\ 
		\texttt{utterance\und id\und fk} & the ID of the utterance the morpheme belongs to, linking to the table \texttt{utterances} & data \\
		\texttt{session\und id\und fk} & the ID of the session the morpheme belongs to, linking to the table \texttt{sessions} & data \\
		\texttt{corpus} 		& the name of the corpus the morpheme belongs to & data \\
		\texttt{language}		& the dominant language of the corpus & data \\
		\texttt{morpheme\und language}	& the language of an individual morpheme; equals the corpus language by default & data \\
		\texttt{type} 			& the morpheme type (actual vs.\ target, (see \autoref{subsec:Actual and target fields}). Because most corpora only
								  specify either the actual or the target morpheme most of the time (differently from the word level, where contrasting 
								  forms are often given), only this one form is taken over and the type is specified in this column. & data \\
		\texttt{morpheme} 		& an orthographic representation of a morpheme (often in its underlying shape). Mostly this is the only form available, 
								  but in the rare case where both an actual and a target form are given only the actual form is taken over. \\
		\texttt{gloss} 			& a standardized label indicating the function of grammatical morphemes. The \href{http://www.eva.mpg.de/lingua/resources/glossing-rules.php}{Leipzig 
								  Glossing Rules} form the base for standardization and additional labels are drawn from a project-internal vocabulary given in 
								  \autoref{subsec:Grammatical glosses}. Morphemes whose original form cannot be assigned to a standard appear as \texttt{\textsc{null}}
								  in this column. This also includes all lexical morphemes \textendash\ there are too many different types in this partition
								  to create a standardized vocabulary, and there are no simple automatizable rules for distinguishing them from
								  grammatical morphemes. & data/postprocessing \\
		\texttt{gloss\und raw} 	& the original gloss (before standardization). Depending on the corpus, this column may contain glosses for both
			 					  grammatical and lexical morphemes (differently from \texttt{gloss}, where only standardized grammatical labels appear). & data \\
		\texttt{pos} 			& a part-of-speech tag. Parts of speech are also standardized. The project-internal set of tags is given in
			 					  \autoref{subsec:Part-of-speech tags} & data/postprocessing \\
		
		\texttt{pos\und raw} 	& the original part-of-speech tag (before standardization) & data \\
		\texttt{warning}		& warnings about formal problems on the morpheme level & parsing \\[-0.3cm]

	\label{tab:Table morphemes}
\end{longtable}

The table below shows which of these columns are regularly filled in which of the corpora. 

% \small
\begin{longtable}{lccccccccccc}	
	% \toprule
	\textbf{tier} 				& \rot{\textbf{CLC} (ctn)} & \rot{\textbf{crl} (CCLAS)} & \rot{\textbf{JCLD} (ind)} & \rot{\textbf{AIC} (ike)} & \rot{\textbf{MPJC} (jpn)} & \rot{\textbf{MYJC} (jpn)} & \rot{\textbf{StRuC} (rus)} & \rot{\textbf{DSC} (sot)} & \rot{\textbf{KULLDD} (tur)} & \rot{\textbf{PYC} (yua)} & \rot{\textbf{SNC} (yuw)}\\
	\midrule
	\endhead
	
	\bottomrule\\[-0.15cm]
	\caption{Presence of columns in the table \texttt{morphemes}}
	\endfoot
	
	corpus						&  + 		& + 	& +  		& +  		 & + 	  & +  		& + 	 & +  		& + 	  & + 	  & + 	 \\
	language					&  + 		& + 	& +  		& +  		 & + 	  & +  		& + 	 & +  		& + 	  & + 	  & + 	 \\
	gloss	 					&  + 		& + 	& +  		& +  		 & (+) 	  & (+)		& + 	 & +  		& (+) 	  & (+)   & +   \\
	gloss\und raw	 			&  + 		& + 	& +  		& +  		 & (+) 	  & (+)		& + 	 & +  		& + 	  & + 	  & + 	 \\
	id		 					&  + 		& + 	& +  		& +  		 & + 	  & +  		& + 	 & +  		& + 	  & + 	  & + 	 \\
	language					&  + 		& + 	& +  		& +  		 & + 	  & +  		& + 	 & +  		& + 	  & + 	  & + 	 \\
	morpheme\und language		&  + 		& + 	& -  		& -  		 & + 	  & +  		& + 	 & -  		& + 	  & - 	  & + 	 \\
	morpheme 					&  + 		& + 	& +  		& +  		 & (+) 	  & (+)		& + 	 & +  		& (+) 	  & +  	  & +  	 \\
	pos 						&  + 		& + 	& (+)  		& +  		 & + 	  & +  		& + 	 & +  		& + 	  & +  	  & +  	 \\
	pos\und raw 				&  + 		& + 	& -  		& +  		 & + 	  & +  		& + 	 & +  		& + 	  & +  	  & +  	 \\
	session\und id\und fk		&  + 		& + 	& +  		& +  		 & + 	  & +  		& + 	 & +  		& + 	  & + 	  & + 	 \\
	type						&  + 		& + 	& +  		& +  		 & + 	  & +  		& + 	 & +  		& + 	  & + 	  & + 	 \\
	utterance\und id\und fk		&  + 		& + 	& +  		& +  		 & + 	  & +  		& + 	 & +  		& + 	  & + 	  & + 	 \\
	warning 					&  + 		& + 	& + 		& + 		 & + 	  & + 		& + 	 & + 		& + 	  & + 	  & + 	 \\

\end{longtable}
\normalsize


\subsection{Table \texttt{all\und data}}
\label{subsec:Table all_data}

This table only exists in the R object and brings together information from all tables in one big flat object. IDs and foreign keys on which the merger is performed, duplicated columns, and a few less often used columns are omitted. Some columns are renamed in order to make clear which table they originate from. The table below shows the correspondences between original columns and columns in \texttt{all\und data}. 

\begin{longtable}[ht!]{lll}
	\toprule
		\textbf{original table} & \textbf{old column} & \textbf{new column} \\
	\midrule
	\endhead
	
	\bottomrule\\[-0.15cm]
	\caption{Columns in the merged table}
	\endfoot
	
		sessions 				& id					& session\und id \\
		sessions 				& source\und id			& session\und id\und source \\
		sessions 				& corpus				& corpus \\
		sessions 				& language				& language \\
		sessions 				& date					& date \\
		% sessions 				& media					& - \\
		% sessions 				& media\und type 		& - \\
		speakers				& id 					& speaker\und id \\
		speakers				& uniquespeaker\und id\und fk 	& - \\
		speakers				& session\und id 		& - \\
		speakers				& corpus				& corpus \\
		speakers				& language				& language \\
		speakers				& speaker\und label 	& speaker\und label \\
		speakers				& name					& name \\
		speakers				& age					& age \\
		speakers				& age\und in\und days 	& age\und in\und days \\
		speakers				& age\und raw			& age\und raw \\
		speakers				& gender 				& gender \\
		speakers				& gender\und raw		& gender\und raw \\
		speakers				& role					& role \\
		speakers				& macrorole				& macrorole \\
		speakers				& role\und raw			& role\und raw \\
		speakers				& languages\und spoken 	& - \\
		speakers				& birthdate				& birthdate \\
		uniquespeakers 			& id					& - \\
		uniquespeakers 			& speaker\und label		& - \\
		uniquespeakers 			& corpus				& - \\
		uniquespeakers 			& name					& - \\
		uniquespeakers 			& birthdate				& - \\
		uniquespeakers 			& gender				& - \\			
		utterances				& id					& utterance\und id \\
		utterances				& session\und id\und fk	& - \\
		utterances				& source\und id			& utterance\und id\und source \\
		utterances				& corpus				& corpus \\
		utterances				& language				& language \\
		utterances				& speaker\und label		& speaker\und label \\
		utterances				& addressee				& addressee \\
		utterances				& utterance				& utterance \\
		utterances				& utterance\und raw		& utterance\und raw \\
		utterances				& translation			& translation \\
		utterances				& sentence\und type		& sentence\und type \\
		utterances				& start					& start \\
		utterances				& start\und raw			& start\und raw \\
		utterances				& end					& end \\
		utterances				& end\und raw			& end\und raw \\
		utterances				& morpheme				& utterance\und morphemes \\
		utterances				& gloss\und raw			& utterance\und glosses\und raw \\
		utterances				& pos\und raw			& utterance\und poses\und raw \\
		utterances				& comment				& comment \\
		utterances				& warning				& - \\
		words					& id					& word\und id \\
		words					& utterance\und id\und fk & - \\
		words					& session\und id\und fk	& - \\
		words					& corpus				& corpus \\
		words					& language				& language \\
		words					& word					& word \\
		words					& word\und actual		& word\und actual \\
		words					& word\und target		& word\und target \\
		words					& pos					& pos\und word\und stem \\
		words					& warning				& - \\
		morphemes				& id					& morpheme\und id \\
		morphemes				& word\und id\und fk	& - \\
		morphemes				& utterance\und id\und fk & - \\
		morphemes				& session\und id\und fk & - \\
		morphemes				& corpus				& corpus \\
		morphemes				& language				& language \\
		morphemes				& type					& morpheme\und type \\
		morphemes				& morpheme				& morpheme \\
		morphemes				& gloss					& gloss \\
		morphemes				& gloss\und raw			& gloss\und raw \\
		morphemes				& pos					& pos \\
		morphemes				& pos\und raw			& pos\und raw \\
		morphemes				& warning				& - \\

		\label{tab:Columns in the merged table}
\end{longtable}


\subsection{Actual and target fields}
\label{subsec:Actual and target fields}

All of the original subcorpora make a distinction between what a child actually said and what the adult target form would have been. Although none of the corpora carries this distinction through on all tiers of all levels, all of them incorporate it at least implicitly and many have separate tiers for the actual and target versions of at least overarching tiers. The table below shows for each corpus if the main tiers of each level always belong to one type (“a(ctual)” or “t(arget)”), if the types are distinguished using separate tiers (“a vs.\ t”), or if both types are mixed on a single tier without making a clear distinction (“a/t”).

\begin{table}[ht!]
	\centering
	\begin{tabular}{lllll}
		\toprule
			\textbf{subcorpus} & \texttt{words} & \texttt{morphemes} 	& \texttt{morphemes} & \texttt{morphemes} \\
							   & \texttt{word}  & \texttt{morpheme} 	& \texttt{gloss} 	 & \texttt{pos} \\
		\midrule
			CLC	(Chintang) 		   & a			  	 & t					& t			 		& t     \\
			CCLAS (Cree)		   & a vs.\ t		 & a vs.\ t				& t			 		& t	  	\\
			JCLD (Indonesian)	 	   & a vs.\ t		 & t					& t			 		& - 	\\
			AIC	(Inuktitut)		   & a vs.\ t		 & t/a					& t			 		& t  	\\
			MPJC (Japanese)		   & a vs.\ t		 & t/a					& t/a			 	& t/a 	\\
			MYJC (Japanese)		   & a vs.\ t 	  	 & t		 			& t			 		& t	  	\\
			StRuC (Russian)		   & a			  	 & t					& t			 		& t	  	\\
			DSC	(Sesotho)		   & a vs.\ t		 & t	 				& t			 		& t 	\\
			KULLDD (Turkish)	   & a vs.\ t 	  	 & t/a					& t/a		 		& t/a	\\
			PYC	(Yucatec)		   & t/a			 & t					& t			 		& t	  	\\
			SNC	(Nungon)		   & t			  	 & t					& t			 		& t	  	\\
		\bottomrule
	\end{tabular}
	\caption{Actual and target tiers in the original subcorpora}
	\label{tab:Actual and target tiers in the original subcorpora}
\end{table}

Since there is not a single corpus which consistently codes the actual/target distinction on all tiers of all levels and the overall emerging picture is rather chaotic, the following rules for simplification were applied: 

\begin{itemize}
	\item The distinction is most relevant and most frequently coded on the word level. Therefore, the \texttt{words} table of the ACQDIV Corpus
		features three columns: \texttt{word\und actual}, \texttt{word\und target}, and \texttt{word}. The latter is intended for easy searches
		regardless of the actual/target distinction. It contains the actual word form by default but may contain the target word form in the rare
		case that the actual word form is not available.
	\item On the morpheme level, the actual/target distinction is less relevant and less consistently coded. The \texttt{morphemes} table
		therefore only gives three default columns (\texttt{morpheme}, \texttt{gloss}, \texttt{pos}) and an additional column \texttt{type}
		that specifies if the values normally correspond to actual or to target forms. This representation glosses over inconsistencies (many of the subcorpora 
		do not have a clear guideline for the distinction on the morpheme level so that both actual and target forms are found) and ignores any 
		differences that might exist between the three fields. 
	\item Finally, only two corpora (Cree and Indonesian) makes a distinction on the utterance level. This distinction is therefore completely ignored in the ACQDIV Corpus. 
\end{itemize}

% This is described in detail in issue #609.
The current implementation for \texttt{word\und actual} and \texttt{word\und target} is encoded with a binary value in the \textit{consistent\_actual\_target} field in the corpus-specific configuration files. The reasoning behind this decision is that if a corpus has a high congruence between actual and target in child speech, it is likely, that the transcribers/annotators did not make the distinction consistently. In other words, ``inconsistently coded" means there are some cases where an actual/target distinction has been coded but they are so few that it's very unlikely that the coding is consistent. There are two corpora that are currently tagged \textit{consistent\_actual\_target$==$no}. These are Yucatec and Nungon. In Yucatec, the transcription comes on three lines (Pfeiler, p.c.):

\begin{enumerate}
\item the expression according to the norm
\item phon: what the child actually said
\item eng/esp: Translation into Spanish.
\end{enumerate}

\noindent In our data processing, we did not include the CHAT \textit{\%(x)pho} tier, it being located on the utterance level (i.e.\ alignment to words is not guaranteed). All in all we had too few corpora with phonetic transcriptions on that level to introduce an column like \textit{utterances.phonetic}. Perhaps in Yucatec, however, the alignment is even good enough to pull at least the Yucatec tier to the word level -- evaluation pending. 

Regarding Nungon, the transcriber aims to write down the actual word spoken by the child (Sarvasy, p.c.). If the spoken form by the child diverges phonetically or phonologically from the adult enough to be noticed by the transcriber, this should be documented on the first tier. The second tier then reflects the target form, as understood by the transcriber. Large differences are noted by the transcriber, but minor ones may be missed (e.g.\ the child doesn't articulate a word medial syllable carefully). Hence Nungon given our criteria for inconsistent encoding is marked as ``no'', but this is due to the very few data points that we have and this issue should be revisited in the future when more data is put into the pipeline.

Finally, there is largely no difference in \textit{consistent\_actual\_target$==$yes} languages in the adult forms. That is, most transcriptions of words by adults are the same between actual spoken utterance and the target form. This seems a bit confusing from a database point-of-view, i.e.\ a lot of duplicated between two or three fields in the same table (word, word\_actual, word\_target). But the reasoning here is that each column can be taken separately depending on the analysis.

\section{Conventions}
\label{sec:Conventions and standardization}

\subsection{Transcription conventions}
\label{subsec:Transcription conventions}

The transcription conventions used in the ACQDIV Corpus have been greatly simplified compared to the original subcorpora, especially those that were initially coded as CHAT or TalkBank XML. This was necessary in order to ensure maximal comparability. 

While the conventions for representing segmental material have not been touched, the following changes were applied with respect to additional symbols: 

\begin{itemize}
	\item All punctuation has been removed. The information contained in utterance delimiters (including CHAT’s Special Utterance Terminators) was transferred to the newly introduced tier 
		\texttt{utterances.sentence\und type} (for instance, a utterance-final question mark now corresponds to the sentence type “question”). 
	\item All special CHAT codes such as postcodes, Satellite Markers, tone and prosody markers, quotation markers, Utterance Linkers, and overlap markers have been removed without replacement. 
	\item Likewise, CHAT’s Special Form Markers (codes starting with “@” attached to words) have been deleted. 
	\item CHAT’s Local Events have been transferred to the comment tier (concatenating them to any pre-existing material). Where the utterance only consists of an event, the sentence type has been set to “action”. 
		Pauses, which are also classified as a special type of Local Event by the CHAT manual, have been removed without traces. 
	\item All types of codes for untranscribed material have been replaced by \texttt{NULL/NA} in isolation and by “???” when embedded into a string. This includes the CHAT codes “xxx”, “yyy”, “www”, so the the difference 
		between unintelligible words, words with a clear phonetic shape but unclear phonology, and words not transcribed for other reasons is lost. Where more detailed comments are available on what could not be transcribed
		for which reason, they are transferred to the relevant \texttt{warnings} field. 
	\item Morpheme separators (mainly given on the morphology tiers, but sometimes also elsewhere) have been deleted. The information contained in them has been transferred to the field \texttt{morphemes.pos}, 
		where all prefixes and suffixes get the dummy tags “pfx” and “sfx”, respectively. 
	\item A few corpora have explicit coding for compounds. This has been simplified (see the description of the \hyperref[cha:Data sources]{original data}), leaving only “=” as 
		the separator between the compound elements (“apple=tree”). 
\end{itemize}

This leaves “???” (untranscribed element within string) and “=” (compound separator) as the only metalinguistic elements on the object language tiers. 

\subsection{Roles and macroroles}
\label{subsec:Roles and macroroles}

The ACQDIV Corpus currently allows the following values in the \texttt{speakers.role} field. This list is the result of a simplification of the values found in the original data, which are diverse both because of terminological differences (e.g.\ “target child” vs.\ “focus child”) and spelling mistakes (“Garndmother” vs.\ “Grandmother”). While some subcorpora distinguish between kinship terms (“mother”, “son”), age groups (“child”, “adult”), and other roles (“caretaker”, “playmate”) most of the corpora do not, so these categories also appear as one in the ACQDIV Corpus. 

\begin{multicols}{4}
	\noindent Adult \\
	Aunt \\
	Babysitter \\
	Boy \\
	Brother \\
	Caller \\
	Caretaker \\
	Child \\
	Cousin \\
	Daughter \\
	Family\und Friend \\
	Father \\
	Female \\
	Friend \\
	Girl \\
	Grandfather \\
	Grandmother \\
	Great-Grandmother \\
	Host \\
	Housekeeper \\
	Male \\
	Mother \\
	Neighbour \\
	Niece \\
	Playmate \\
	Research\und Team \\
	Sister \\
	Speaker \\
	Student \\
	Subject \\
	Target\und Child \\
	Teacher \\
	Teenager \\
	Toy \\
	Twin\und Brother \\
	Uncle \\
	Unknown \\
	Visitor \\
\end{multicols}

The field \texttt{speakers.macrorole}, which is created during postprocessing, is the result of mapping these roles to the four values “Child”, “Target\und Child”, “Adult”, and “Unknown”. Differently from the \texttt{role} field, macroroles also include inference based on age (younger than 12 years = “Child”) and IDs (e.g.\ CHI = “Child”; other ID-based mappings depend on the individual corpora). 


\subsection{Grammatical glosses}
\label{subsec:Grammatical glosses}

The ACQDIV Corpus uses a standardized set of grammatical glosses in the column \texttt{morphemes.gloss}. The value used in the original data is given in \texttt{morphemes.gloss\und raw}. The standardizet set consists of all glosses proposed in the \href{http://www.eva.mpg.de/lingua/resources/glossing-rules.php}{Leipzig Glossing Rules} plus additional values as needed (marked with an asterisk in the list below). Less frequent values were directly taken over from the original data in order to fill all rows but are not documented below. 

% original whitelist from integrity tests: 
% None, "0", "1", "1/2PL", "1DL", "1NSG", "1PL", "1SG", "2", "2DL", "NSG", "2SG", "2PL", "3", "3DL",
% "3NSG", "3SG", "3PL", "4", "4SYL", "A", "ABIL", "ABL", "ABS", "ACC", "ACROSS", "ACT", "ADESS", "ADJ",
% "ADJZ", "ADN", "ADV", "ADVZ", "AFF", "AGT", "AGR", "ALL", "ALT", "AMBUL", "ANIM", "ANTIP", "AOR",
% "APPL", "ART", "ASP", "ASS", "ASSOC", "ATTN", "AUTOBEN", "AUX", "AV", "BABBLE", "BEN", "CAUS", "CHOS",
% "CLF", "CLIT", "CM", "COM", "COMP", "COMPAR", "COMPL", "CONC", "COND", "CONJ", "CONJ", "CON", "CONT",
% "CONTEMP", "CONTING", "CONTR", "COP", "CVB", "DAT", "DECL", "DEF", "DEICT", "DEM", "DEP", "DEPR",
% "DESID", "DESTR", "DET", "DETR", "DIM", "DIR", "DIR", "DIST", "DISTR", "DOWN", "DU", "DUB", "DUR",
% "DYN", "ECHO", "EMPH", "EQU", "ERG", "EVID", "EXCL", "EXCLA", "EXIST", "EXT", "F", "FILLER", "FOC",
% "FUT", "FUT1", "GEN", "HAB", "HES", "HHON", "HON", "HORT", "I", "IDEOPH", "II", "III", "IMIT", "IMP",
% "IMPERS", "INAL", "INAN", "INCEP", "INCH", "INCL", "INCOMPL", "IND", "IND1", "IND2", "INDF", "INDIR",
% "INF", "INS", "INSIST", "INTJ", "INTR", "INTRG", "INV", "IPFV", "IRR", "IV", "IX", "LHON", "LNK",
% "LOC", "M", "MED", "MHON", "MIR", "MOD", "MOOD", "MV", "N", "N", "N", "NAG", "NAME", "NC", "NEG",
% "NICKNAMER", "NMLZ", "NOM", "NPST", "NSG", "NTVZ", "NUM", "OBJ", "OBJVZ", "OBL", "OBLIG", "OBV",
% "ONOM", "OPT", "ORD", "P", "PARTIT", "PASS", "PEJ", "PERL", "PERMIS", "PERSIST", "PFV", "PL", "POL",
% "POSS", "POT", "PRAG", "PRED", "PREDADJ", "PREP", "PREP", "PRF", "PRO", "PROB", "PROG", "PROH", "PROP",
% "PROX", "PRS", "PST", "PTCL", "PTCP", "PURP", "PV", "PVB", "Q", "QUANT", "QUOT", "RECENT", "RECNF",
% "RECP", "REF", "REFL", "REL", "REM", "REP", "RES", "REVERS", "S", "S/A", "S/P", "SBJ", "SBJV", "SEQ",
% "SG", "SIM", "SOC", "SPEC", "STAT", "STEM", "SUPERL", "SURP", "TEASER", "TEL", "TEMP", "TENSE", "TERM",
% "TOP", "TR", "UP", "V", "VI", "VII", "VIII", "V2", "V.AUX", "V.CAUS", "V.IMP", "V.ITR", "V.PASS",
% "V.POS", "V.TR", "VBZ", "VOICE", "VN", "VOC", "VOL", "WH", "X", "XI", "XII", "XII", "XIV", "???"
				 
\begin{multicols}{2}
	\begin{tabbing}
	*0 \hspace{2cm} \= non-person \\
	1 \> first person \\
	2 \> second person \\
	3 \> third person \\
	*4 \> fourth person (in switch refer-\\
		\> ence or direct/inverse systems) \\
	*4SYL \> tetrasyllabifier \\
	A \> agent-like argument of canon- \\
		\> ical transitive verb \\
	*ABIL \> abilitative \\
	ABL \> ablative \\
	ABS \> absolutive \\
	ACC \> accusative \\
	*ACROSS \> distal horizontal deixis \\
	*ACT \> active \\
	*ADESS \> adessive \\
	ADJ \> adjective \\
	*ADJZ \> adjectivizer \\
	*ADN \> adnominal \\
	ADV \> adverb(ial) \\
	*ADVZ \> adverbializer \\
	*AFF \> affirmative \\
	*AGT \> agentive \\
	AGR \> agreement \\
	ALL \> allative \\
	*ALT \> alternating tense \\
	*AMBUL \> ambulative \\
	*ANIM \> animate \\
	ANTIP \> antipassive \\
	*AOR \> aorist \\
	APPL \> applicative \\
	ART \> article \\
	*ASP \> unspecified aspect marker \\
	*ASS \> assertive \\
	*ASSOC \> associative \\
	*ATTN \> attention \\
	*AUTOBEN \> autobenefactive \\
	AUX \> auxiliary \\
	*AV \> actor voice \\
	*BABBLE \> babbling \\
	BEN \> benefactive \\
	CAUS \> causative \\
	*CHOS \> change of state \\
	CLF	\> classifier \\
	*CLIT \> clitic with unspecified function \\
	*CM \> compound marker \\
	*COLL \> collective \\
	COM \> comitative \\
	COMP \> complementizer \\
	*COMPAR \> comparative \\
	*COMPL \> completive \\
	*CONC \> concessive \\
	COND \> conditional \\
	*CONJ \> conjunction \\
	*CONJ \> conjugation marker \\
	*CON \> conative \\
	*CONT \> continuous \\
	*CONTEMP \> contemporative mood \\
	*CONTING \> contingent mood \\
	*CONTR \> contrastive \\
	COP \> copula \\
	CVB \> converb \\
	DAT \> dative \\
	DECL \> declarative \\
	DEF \> definite \\
	*DEICT \> deictics (other than\\
		\> demonstratives) \\
	DEM \> demonstrative \\
	*DEP \> dependent (mood or other form) \\
	*DEPR \> deprivative \\
	*DESID \> desiderative \\
	*DESTR \> destructive \\
	DET \> determiner \\
	*DETR \> detransitivization \\
	*DIFF.SBJ \> different subject \\
	*DIM \> diminutive \\
	*DIR \> directional case \\
	*DIR \> direction \\
		\>(in direct/inverse systems) \\
	DIST \> distal \\
	DISTR \> distributive \\
	*DOWN \> distal deixis pointing down \\
	DU \> dual \\
	*DUB \> dubitative \\
	DUR \> durative \\
	*DYN \> dynamic \\
	*ECHO \> echo word \\
	*EMPH \> emphatic \\
	*EQU \> equative \\
	ERG \> ergative \\
	*EVID \> evidential \\
	EXCL \> exclusive \\
	*EXCLA \> exclamation \\
	*EXIST \> existential copula \\
	*EXT \> extensional \\
	F \> feminine \\
	*FILLER \> filler \\
	*FOC \> focus \\
	FUT \> future \\
	GEN \> genitive \\
	*HAB \> habitual \\
	HES \> hesitative \\
	*HHON \> high honorific \\
	*HON \> honorific \\
	*HORT \> hortative \\
	*IDEOPH \> ideophone \\
	*IMIT \> imitative \\
	*IMNT \> imminent \\
	IMP \> imperative \\
	*IMPERS \> impersonal \\
	*INAL \> inalienable possession \\
	*INAN \> inanimate \\
	*INCEP \> inceptive \\
	*INCH \> inchoative \\
	INCL \> inclusive \\
	*INCOMPL \> incompletive \\
	IND \> indicative \\
	INDF \> indefinite \\
	*INDIR \> indirect \\
	INF \> infinitve \\
	INS \> instrumental \\
	*INSIST \> insistive \\
	*INSIST \> intensifier \\
	*INTJ \> interjection \\
	INTR \> intransitive \\
	*INTRG \> interrogative \\
	*INV \> inverse \\
	IPFV \> imperfective	\\
	IRR \> irrealis \\
	*LNK \> linker \\
	LOC \> locative \\
	M \> masculine \\
	*MED \> medial (deixis) \\
	*MHON \> mid honorific \\
	MIR \> mirative \\
	*MOD \> modal \\
	*MOOD \> unspecified mood marker \\
	*MV \> middle voice \\
	N \> neuter \\
	*N \> noun \\
	N \> non- (e.g. NSG, NPST...) \\
	*NAG \> nomen agentis \\
	*NAME \> person's name \\
	*NC \> noun classes, \\
		\> e.g.\ NC.I, NC.II, NC.III... \\
	NEG \> negative \\
	*NICKNAMER \> suffix for forming nicknames \\
	NMLZ \> nominalizer \\
	NOM \> nominative \\
	NPST \> nonpast \\
	NSG \> non-singular \\
	NSPEC \> non-specific \\
	*NTVZ \> nativizer \\
	*NUM \> numeral \\
	OBJ \> object \\
	*OBJVZ \> objectivizer \\
	OBL \> oblique \\
	*OBLIG \> obligative \\
	*OBV \> obviative \\
	*ONOM \> onomoatopoeia \\
	*OPT \> optative \\
	*ORD \> ordinal \\
	P \> patient-like argument of \\
		\> canonical transitive verb \\
	*PARTIT \> partitive \\
	PASS \> passive \\
	*PEJ \> pejorative \\
	*PERL \> perlative \\
	*PERMIS \> permissive \\
	*PERSIST \> persistive \\
	PFV \> perfective \\
	PL \> plural \\
	*POL \> polite \\
	POSS \> possessive \\
	*POT \> potential \\
	*PRAG \> pragmatic marker \\
	PRED \> predicate/predicative \\
	*PREDADJ \> predicative adjective \\
	*PREP \> preposition \\
	*PREP \> preopositional case \\
	PRF \> perfect \\
	*PRO \> pronoun \\
	*PROB \> probabilitive \\
	PROG \> progressive \\
	PROH \> prohibitive \\
	*PROP \> proper noun \\
	PROX \> proximal \\
	PRS \> present \\
	PST \> past \\
	*PTCL \> particle \\
	PTCP \> participle \\
	PURP \> purposive \\
	*PV \> patient voice \\
	*PVB \> preverb \\
	Q \> question \\
	*QUANT \> quantifier \\
	QUOT \> quotative \\
	*RECENT \> recent past tense \\
	*RECNF \> reconfirmative \\
	RECP \> reciprocal \\
	*REF \> referential \\
	REFL \> reflexive \\
	REL \> relative \\
	*REM \> remote (past/future) \\
	*REP \> reportative \\
	RES \> resultative \\
	*RESTR \> restrictive \\
	*REVERS \> reversive \\
	S \> single argument of canonical\\
		\> intransitive verb \\
	*SAME.SBJ \> same subject \\
	SBJ \> subject \\
	SBJV \> subjunctive \\
	*SEQ \> sequential \\
	SG \> singular \\
	*SIM \> simultaneous \\
	*SOC \> sociative \\
	*SPEC \> specific \\
	*STAT \> stative \\
	*STEM \> stem (esp. in languages\\
		\> with multipartite stems) \\
	*SUPERL \> superlative \\
	*SURP \> surprise \\
	*TEASER \> form for teasing people \\
	*TEL \> telic \\
	*TEMP \> temporal \\
	*TENSE \> unspecified tense marker \\
	*TERM \> terminative \\
	TOP \> topic \\
	*TR \> transitive \\
	*UP \> distal deixis pointing up \\
	*V \> verb \\
	*V2 \> vector verb with unspecified\\
		\> function \\
	*V.AUX \> verbal auxiliary \\
	*V.CAUS \> causative verb \\
	*V.IMP \> imperative verb \\
	*V.ITR \> intransitive verb \\
	*V.PASS \> passive verb \\
	*V.POS \> positional verb \\
	*V.TR \> transitive verb \\
	*VBZ \> verbalizer \\
	*VOICE \> voice marker with unspecified\\
		\> function \\
	*VN \> verbal noun \\
	VOC \> vocative \\
	*VOL \> volitional \\
	*WH \> wh-word \\

	\end{tabbing}
\end{multicols}

\noindent The following characters have special meanings: \\

\begin{tabular}{l l}
. 		& joins several functions expressed by a single morpheme, e.g.\ “IND.PST”\\
/ 		& joins alternative functions of a morpheme for which no common label is available, \\
 		& e.g.\ “1/2” (= 1st or 2nd person)\\
\und 	& joins several metalanguage words coding a single object language function,\\
 		& e.g.\ “put\und on”\\
> 		& agent acting on patient; possessor and possessum\\
\end{tabular}

% explanations for some of the more controversial replacements
% 
% Cree 		& 1.pl 		& 1PL.EXCL 	& more standard glosses
% Cree 		& 21.pl 	& 1PL.INCL 	& more standard glosses
% Indonesian 	& IN 		& VOICE 	& the controversial Indonesian voice markers were glossed as neutrally as possible (the original gloss is simply the phonological shape of the marker)
% Indonesian 	& KAN 		& VOICE 	& the controversial Indonesian voice markers were glossed as neutrally as possible (the original gloss is simply the phonological shape of the marker)
% Indonesian 	& BA 		& MV 		& the controversial Indonesian voice markers were glossed as neutrally as possible (the original gloss is simply the phonological shape of the marker)
% Indonesian 	& DIH 		& PV 		& the controversial Indonesian voice markers were glossed as neutrally as possible (the original gloss is simply the phonological shape of the marker)
% Indonesian 	& KE 		& PASS 		& the controversial Indonesian voice markers were glossed as neutrally as possible (the original gloss is simply the phonological shape of the marker)
% Indonesian 	& MEN 		& AV 		& the controversial Indonesian voice markers were glossed as neutrally as possible (the original gloss is simply the phonological shape of the marker)
% Indonesian 	& N 		& AV 		& the controversial Indonesian voice markers were glossed as neutrally as possible (the original gloss is simply the phonological shape of the marker)
% Indonesian 	& TA 		& PASS 		& the controversial Indonesian voice markers were glossed as neutrally as possible (the original gloss is simply the phonological shape of the marker)
% Indonesian 	& TER 		& PASS 		& the controversial Indonesian voice markers were glossed as neutrally as possible (the original gloss is simply the phonological shape of the marker)
% Inuktitut 	& IND 		& IND1 		& the participial mood is used in a similar way to the indicative in Tarramiut (the dialect of the corpus)
% Inuktitut 	& PAR 		& IND2 		& the participial mood is used in a similar way to the indicative in Tarramiut (the dialect of the corpus)
% Inuktitut 	& CSV 		& CONTING 	& this mood is traditionally called "causal" but rather has temporal semantics
% Inuktitut 	& MOD 		& INS 		& the "modal" case performs a variety of functions. According to Shanley its main function in Tarramiut is to mark objects in antipassive, but that's not what the corpus data show. The case is called "instrumental" here because this is one of its less hard to analyze functions and also the term used in other Inuit languages
% Inuktitut 	& ABS_ERG 	&  			& forms that functionally could be both absolutive or ergative are zero-marked
% Inuktitut 	& SGPL 		& NDU 		& some forms can be interpreted as singular or plural but not dual - this is given as "non-dual" here
% Inuktitut 	& Xx 		&  			& some verb forms do not index an argument. For instance, a form which indexes a third person singular object but no agent would be given as XxS_3sO in the original data but becomes 3SG.P in the ACQDIV Corpus.
% Inuktitut 	& ALL_3Ssg 	& POSS.3SG>SG.ALL & Inuktitut has portmanteau suffixes indicating both the person and number of the possessor and the number and case of the possessum
% Russian 	& FULL 		&  			& the full form is the default form of adjectives and therefore not specially marked
% Russian 	& SHORT 	& PREDADJ 	& the short form is only possible in predicative use, so this gloss is a little more telling
% Russian 	& IRREFL 	&  			& in the original data all non-reflexive forms are marked as "irreflexive". However, since this is clearly the default no special marking is necessary.
% Sesotho 	& m^s 		& SBJV1 	& the so-called "participial mood" is functionally similar to the subjunctive
% Sesotho 	& m^pt 		& SBJV2 	& the so-called "participial mood" is functionally similar to the subjunctive
% Sesotho 	& t^np 		& REM.PST 	& the narrative past is interpreted as remote past
% Sesotho 	& 1, 2, 3... & NC.I, NC.II, NC.III... & noun class numbers are converted to Roman numbers with a prefix NC ("noun class")
% Sesotho 	& come/m^i 	& come.IMP 	& TMA is often expressed by suppletive stems in Sesotho. The complete gloss is taken over even though the stem is lexical.
% Turkish 	& GER 		& CVB, INF, NMLZ, PTCP & so-called gerunds are relabeled as converb, nominalizer, or infinitive, depending on their function (which can be inferred from their phonological shape, e.g. GER:ACAK = PTCP, GER:IP = CVB etc.).
% Yucatec 	& 1PRON.SG 	& 1SG 		& free pronouns don't get a special gloss
% Yucatec 	& AG 		& PURP 		& Was noted as "agentive case", but this is unlikely because there is an ERG which is formally distinct from this form (AG|h). There aren't too many attestations in the corpus, but the form always occurs before verbs (often marked as a prefix) and correlates with S/A coreferential purposive clauses in the translations ('in order to do'). The meaning of the original abbreviation is unclear. 

\subsection{Part-of-speech tags}
\label{subsec:Part-of-speech tags}

The ACQDIV Corpus uses a standardized set of part-of-speech tags in the column \texttt{morphemes.pos}. The set was deliberately kept small in order to make broad comparisons across languages possible. The original tags are maintained in the column \texttt{morphemes.pos\und raw}. \texttt{NULL/NA} is inserted when the part of speech is unknown. Tags not contained in the Leipzig Glossing Rules are again marked by an asterisk in the list below.

\begin{multicols}{2}
	\begin{tabbing}
	ADJ \hspace{2cm} \= adjective \\
	ADV \> adverb \\
	ART \> article \\
	AUX \> auxiliary \\
	CLF \> numeral classifier \\
	CONJ* \> conjunction \\
	IDEOPH* \> ideophone \\
	INTJ* \> interjection \\
	N* \> noun \\
	NUM* \> numeral \\
	pfx* \> prefix \\
	POST* \> postposition \\
	PREP* \> preposition \\
	PRODEM* \> pronouns/demonstratives \\
	PTCL* \> particle \\
	PVB* \> preverb \\
	QUANT* \> non-numeral quantifier \\
	sfx* \> suffix \\
	stem* \> stem \\
	V* \> verb \\
	\end{tabbing}
\end{multicols}

The Universal Dependency (UD) part-of-speech tag is added to the column \texttt{words.pos\und ud}. It is derived from the raw POS label and not from the standardized ACQDIV tag, i.e. every corpus has a separate mapping which is defined in the corresponding configuration file. The reason for this is that the UD tags are more specific in some cases. For instance, the UD tag-set distinguishes between determiners (DET) and pronouns (PRON) whereas the ACQDIV tag-set conflates them to PRODEM. This would lead to arbitrary mappings like `PRODEM=PRON' which would bias the UD label distribution. There are also numerous cases where the raw tags are less specific than the UD tags. In these cases, we map them to the most common equivalent. All cases are listed in Table \ref{tab:mapping-raw-UD}.


\begin{longtable}{lp{.2\linewidth}p{.5\linewidth}}
	\toprule
		\textbf{Corpus} & \textbf{Mapping} 	& \textbf{Comment} \\
	\midrule
	\endhead
		
	\bottomrule\\[-0.15cm]
	\caption{Problematic mappings of raw to UD POS tags.}
	\endfoot
		Chintang                  & gm = PART                 & Some of these are ADP. In Nepali, CCONJ and SCONJ are also possible. \\
		Chintang                  & n = NOUN                  & PROPN are not marked. \\
		Chintang                  & pro = PRON                & There is no lexical distinction between referential and adnominal pronouns in Chintang, but in UD they would probably be tagged as DET even when simply used adnominally in syntax. \\
		Cree                      & p,conjn = CCONJ           & It seems like there is no difference between subordinating and coordinating conjunctions in Cree, and all conjunctions in our corpus have glosses that one would rather associate with a coordinating function. However, there are very clear (verbal inflectional) markers of subordination with which these conjunctions can co-occur. Thus, UD might require that they be tagged as SCONJ in such cases. \\
		Cree                      & pro,* = PRON              & This is a whole class of tags, some of which might also be DET in the UD framework. \\ 
		Inuktitut                 & DEM, DM, DR, LR = PRON    & These are demonstrative stems whose translation and classification depends a lot on case. In the ABS or ERG, they correspond to English pronouns, in the various LOC cases to (pronominal) adverbs such as `here', `there', which in UD would be tagged ADV. \\ 
		Japanese MiiPro/Miyata    & conj = CCONJ              & Some of these would probably be counted as SCONJ under the UD definition, but most are CCONJ. \\
		Japanese MiiPro/Miyata    & n:deic:pr(e)s \hspace{2cm} = PRON      & The personal pronouns behave like ordinary nouns in Japanese, but this classification is probably more in the comparative spirit of UD. \\
		Japanese MiiPro/Miyata    & ptl:conj = PART           & This is a heterogeneous class, some of whose members would rather correspond to ADP or SCONJ, depending on their use in syntax. \\
		Nungon                    & d, dem = PRON             & Some of these can probably be used adnominally, i.e. they would be DET depending on use. \\ 
		Nungon                    & n = NOUN                  & PROPN is not distinguished. \\
		Russian                   & CONJ = CCONJ              & Some of these are definitely SCONJ, but the two types that cover 70\% of all tokens (`a' and `i') are CCONJ. \\
		Russian                   & NA = PRON                 & This also includes some potential DET. \\
		Russian                   & PRO-DEF, PRO-DEM, \hspace{2cm} PRO-INTERROG, PRO-REFL = PRON & This could also be DET. Note, though, that in most cases Russian consistently distinguishes between adnominal and referential use, e.g. PRO-DEM-ADJ vs. PRO-DEM-NOUN. Thus, this is much less of a problem than in the other corpora. \\
		Sesotho                   & cj = CCONJ                & Most of these seem to be CCONJ, but it is not excluded that there are also SCONJ. The grammatical situation is similar to Cree, i.e. subordination is mainly expressed via verbal inflections. \\
		Sesotho                   & d = PRON                  & These words can also be used adnominally (DET). \\
		Sesotho                   & ps = DET                  & Possessives are adnominal by default, but they can also refer (e.g. `Whose is this?'). \\
		Turkish                   & CON* = CCONJ              & The two most frequent types, which cover 80\% of all tokens (`dA', `ama'), are clearly CCONJ, but others might be SCONJ. \\
		Yucatec                   & CONJ = CCONJ              & The two most frequent types are CCONJ and cover 75\% of all tokens (`kux', `pero'). Others might be SCONJ. \\
		Yucatec                   & DET = DET                 & The Yucatec tag also covers forms that can be used referentially. It is not clear what criteria the use of DET in the corpus was based on (it does not seem to be syntax). \\
		Yucatec                   & QUANT = PRON              & This includes many forms that can also be used adnominally (DET), but only syntactic annotations would help us. \\
		\label{tab:mapping-raw-UD}
\end{longtable}

\chapter{Data sources}
\label{cha:Data sources}
% TODO add documentation for sources of metadata (XPath for IMDIs, CHAT tiers otherwise)

This chapter describes the structure of the input data and how it is mapped to the target structure found in the ACQDIV Corpus. The \texttt{warnings} tiers found on the utterance, word, and morpheme levels are inserted during postprocessing and are therefore ignored below. 

Note that the input data used for the ACQDIV Corpus are a subset of the original data. Tiers that were not present in the majority of corpora were generally ignored, as were parts of the subcorpora whose target children were out of the core age range (2;0.0-3;12.31) during the whole recording period. For this reason this chapter cannot be a complete documentation of the original data and may often diverge from the original documentation (which is linked below, if existing).

There were two valid input formats, TalkBank XML and Toolbox. A third important format is CHAT: most corpora were originally formatted as CHAT and still contain traces of it. These were converted from CHAT to a valid input format either by the respective corpus teams or by the ACQDIV core team. \autoref{sec:Corpus formats} gives a brief introduction to the three formats. For details on conversion work done by the ACQDIV core team, see \autoref{cha:Generating the corpus}. The remaining sections in this chapter deal with the particularities of the individual subcorpora. 

The locations of tiers in XML corpora are specified using XPath. Toolbox corpora are flatter, so it is sufficient to give the tier name here. 


\section{Original corpus formats}
\label{sec:Corpus formats}

\subsection{CHAT}
\label{subsec:CHAT}

CHAT is the original format of most subcorpora: Indonesian, Inuktitut, Japanese MiiPro and Miyata, Russian, Sesotho, Turkish, and Yucatec. The Indonesian and
Russian corpora had already been converted to Toolbox by the corpus teams at the time they were added to the ACQDIV Corpus, so the input format was Toolbox.
Similarly, the two Japanese corpora and Sesotho had been converted to TalkBank XML using the existing parser Chatter, 
so only Inuktitut, Turkish, and Yucatec were still only available as CHAT at the beginning of the project. Nevertheless, traces of CHAT are omnipresent in all corpora listed above, be it because of imperfect conversion routines or because of deliberate exceptions. 

CHAT is the format associated with the \href{http://childes.psy.cmu.edu/}{CHILDES online archive} of child language acquisition corpora \citep{MacWhinney2000a}. The full specification can be found at \url{http://childes.psy.cmu.edu/manuals/CHAT.pdf}. 
The most important characteristics of CHAT are as follows. 

One corpus file corresponds to one recording session (or sometimes to a smaller stretch corresponding to the length of a tape). Each file contains the metadata for the session and all speakers in its head and the primary data (transcriptions and all annotations) in its body. Corpus-level metadata are stored in separate text-based files with the extension cdc. The body part of corpus files is divided into utterance blocks, where each utterance block in turn consists of one or several lines corresponding to different tiers. The first line in an utterance block is the main transcription tier and all following lines are annotations associated with it. An example for the first few lines of a CHAT corpus file is shown in the screenshot in \autoref{fig:CHAT example}. The file was opened in CLAN, the editor associated with the format. 

\begin{figure}[ht!]
	\centering
	\includegraphics[scale=0.55]{pics/chat-screenshot.png}
	\caption{The first lines of a typical CHAT file, opened in CLAN}
	\label{fig:CHAT example}
\end{figure}

A peculiarity of CHAT, which makes it as difficult to keep it consistent as it is to parse it, is that logical tiers are often not kept separate in the syntax (i.e.\ information belonging to different tiers may be coded on a single line) and that a multitude of special characters in various combinations is used to accommodate such “dislocated” annotations. For instance, error coding, coding for action accompanying speech, comments on the language and register of individual words, prosodic and/or pragmatic markers, and free comments may all be inserted on the main transcription tier using various kinds of brackets, asterisks, equal signs, at signs combined with single-letter codes, and various combinations of punctuation markers. 

Take the string “dashiyo(o)@n [= dasoo] [*] ka ?” (taken from Japanese MiiPro, als19990706.cha) as an example. Here, what the child said is \emph{dashiyo ka}. “(o)” means the transcriber assumes this form has been shortened (the target form would have had a long [oː]), “@n” indicates that the same word is a neologism, “[= dasoo]” gives the adult target form, “[*]” marks \emph{dashiyo} as an error, and “?” marks the whole utterance as a question. 

% more fun stuff in case we need it: 
% Turkish_KULLDD/cha/can07_22aug01_01-00-03.cha:*FAT:	hadi yeo:vah@i [!] dıgıdık_dıgıdık@o [x 6] [= meaning the voice of running horses] .
% Turkish_KULLDD/cha/tugce37_09sep03_02-03-28.cha:*MOT:	ama git <oraya [= in front of MOT] git> [x 2] bak ben sana [=! makes a bubble] .

This mismatch between corpus syntax and semantics also was the reason why CHAT was not accepted as an input format for the ACQDIV Corpus. The three corpora that initially were only available as CHAT were cleaned and converted to TalkBank XML as described in \autoref{cha:Generating the corpus}. Nevertheless, CHAT is present in fragments in almost all input data, especially in the morphology tier \texttt{\%mor:}, whose syntax is so complicated and inconsistent that it turned out to be easier to take it over without changes into TalkBank XML and parse it from there than to first make it conform to CHAT and then convert it. 

The following tools are associated with CHAT: 

\begin{itemize*}
	\item CLAN can be used for editing and validating CHAT and can perform basic statistics. It can be downloaded from \url{http://childes.psy.cmu.edu/clan/}.
		Documentation can be found at \url{http://childes.psy.cmu.edu/manuals/clan.pdf}. 
	\item Chatter is a parser that can transform CHAT to TalkBank XML. It can be downloaded from \url{http://talkbank.org/software/chatter.html}. 
		There is no comprehensive documentation available. 
\end{itemize*}


\subsection{TalkBank XML}
\label{subsec:TalkBank XML}

TalkBank XML is an XML format closely associated with CHILDES and CHAT (and a few other formats and archives, see \url{http://talkbank.org/}). There is a parser from CHAT to TalkBank XML (\href{http://talkbank.org/software/chatter.html}{Chatter}) that can deal with all standard CHAT constructs. Currently there are two different yet closely related schemas describing the structure of TalkBank XML:

\begin{itemize*}
	\item \url{http://talkbank.org/talkbank.xsd}
	\item \url{https://talkbank.org/software/talkbank.xsd}
\end{itemize*}

TalkBank XML keeps the basic structure of CHAT, coding all data and most metadata associated with a session in a single XML file with a head and a body section. Within the body section, the utterance and word levels are marked by the nested tags \texttt{<u>} and \texttt{<w>}, respectively. While the CHAT primary transcription tier is split up into words in TalkBank XML, all other tiers are taken over \emph{en bloc} and appear directly under \texttt{<u>} as \texttt{<a>} with various attributes marking the tier type.

The frequent mismatches between corpus syntax and semantics that are characteristic of CHAT carry over to TalkBank XML, where they are variously coded as attributes of words or groups of words (\texttt{<g>} between \texttt{<u>} and \texttt{<w>}), as tags nested in \texttt{<w>}, or as tags on the same level as \texttt{<w>} (grouped together with it via \texttt{<g>}).

Only one of the XML corpora in the ACQDIV Corpus (Japanese Miyata) has explicit XML coding for morphemes. In the other corpora, morphology is coded in less explicit, often very dense and idiosyncratic formats and is structurally located on the utterance level.

% Notes by SM:

% The CHAT/Talkbank format is limited (this is not a criticism, its formatting was very forward thinking at the time of its creation and inception). Technology however has advanced considerably. In 2001 a full definition of the CHAT format was developed by Romeo Anghelache (from the formal CHAT specifications) and released under the GNU Public License, 2001. Development was continued by Franklin Chen.\footnote{}
%
% In 2001, Anghelache and Chen's work resulted in a formalized XML schema (XSD) of the CHAT formalization.\footnote{URL}
%
% The XSD uses the Java Architecture for XML Binding (JAXB). JAXB allows Java developers to map Java classes to XML representations.
%
% It is presumably used to validate XML output from CHATTER, a GUI program that allows users to convert between CHAT format and ChatXML format.
%
% % https://en.wikipedia.org/wiki/Java_Architecture_for_XML_Binding
% "The tool "xjc" can be used to convert XML Schema and other schema file types (as of Java 1.6, RELAX NG, XML DTD, and WSDL are supported experimentally) to class representations."
%
% % Namespace in the XML schema? Pain in the arse?
% % <Element '{http://www.talkbank.org/ns/talkbank}w' at 0x100c989a8>

This is also the main reason why TalkBank XML is not trivial to process. The most important problematic constructs are briefly described below. All of them have the following properties in common:

\begin{itemize*}
	\item They feature a contrast between an actual and a target form (cf.\ \autoref{subsec:Actual and target fields}).
	\item They are coded by a complicated syntax that makes it hard to process them (especially when nested).
	\item Words containing them are frequently not glossed, giving rise to alignment problems between the word and morpheme levels.
\end{itemize*}

\autoref{tab:TalkBank shit} shows an overview of the existing constructs and their distribution in the ACQDIV original data. “-” indicates absence, “+” presence; where a construction is present, “+g” indicates that it is normally glossed, “+n” that it is normally not glossed.

\begin{table}
	\centering
	\begin{tabular}{lccccccc}
		\toprule
		& \textbf{CCLAS} & \textbf{AIC} & \textbf{MPJC} & \textbf{MYJC} & \textbf{DSC} & \textbf{KULLDD} & \textbf{PYC} \\
								  & \textbf{crl} & \textbf{ike} & \textbf{jpn} & \textbf{jpn} & \textbf{sot} & \textbf{tur} & \textbf{yua} \\
		\midrule
				   untranscribed 	& +g 		& +g 			& +n 			& +n 			& +g 			& +n 			& +n \\
				   fragments		& - 		& +n 			& +n 			& +n 			& - 			& - 			& - \\
				   omissions		& +? 		& - 			& - 			& - 			& - 			& - 			& - \\
				   replacements 	& - 		& - 			& +g 			& +g 			& - 			& +g 			& - \\
				   shortenings 		& +?		& +g 			& +g 			& +g 	 		& +g			& +g 			& +g \\
				   repetitions 		& - 		& +g 			& +n 			& - 			& - 			& +n 			& - \\
				   retracings 		& - 		& +g 			& +n 			& +n 			& - 			& +g 			& - \\
		\bottomrule
	\end{tabular}
	\caption{Distribution of actual/target constructs in the ACQDIV original data}
	\label{tab:TalkBank shit}
\end{table}

\subsubsection*{Untranscribed words}

Words can be untranscribed for various reasons but mostly are because they are unintelligible. This is coded in TalkBank XML as \texttt{<w untranscribed="unintelligible">xxx </w>}. In the ACQDIV Corpus, the actual and target form for such words is \texttt{NULL/NA} (= ‘unknown’).

\subsubsection*{Fragments}

An actual word with no clear target is called a fragment. TalkBank XML codes this as \texttt{<w type="fra} \texttt{gment">...</w>}, where the “...” part marks the transcribed actual word. The target is set to \texttt{NULL/NA} in the ACQDIV Corpus.

\subsubsection*{Omissions}

Rarely, no actual word is present but the target syntax suggests there should have been. In this case the \texttt{<w>} tag contains the target form and is marked as an omission by the type attribute: \texttt{<w type="omission">...</w>}. Omissions are completely ignored in the ACQDIV Corpus (i.e.\ the omitted target word is not represented at all) because they are rare and considered to be speculative.

\subsubsection*{Replacements}

Children often replace a target word by another, similar actual word. TalkBank XML codes this in a rather complicated manner: all involved words appear within the group tag \texttt{<g>}. The actual form is the text of an ordinary \texttt{<w>} tag, which is the parent of a \texttt{<replacement>} tag that can itself contain an arbitrary number of \texttt{<w>} tags having the target form(s) as their text. The schema in short is \texttt{<g><w>...<replacement><w>...</w></replacement></w></g>}. The actual/target contrast is taken over into the ACQDIV Corpus. Where a single actual word corresponds to several target words, new empty actual words are inserted, creating a 1:1 correspondence between the two groups of words.

\subsubsection*{Shortenings}

Shortenings feature a contrast between a full target string and a contracted actual string. The target string may once more consist of several words, so the group tag \texttt{<g>} is used: \texttt{<g><w>...<shortenin} \texttt{g>...</shortening>...</w></g>}. The substring that was omitted is the text of the nested \texttt{<shortening>} tag. The full target string is the text of the parent \texttt{<w>} before \emph{and} after \texttt{<shortening>}. Shortenings are mapped similarly to replacements.

In some corpora, shortenings may appear within replacements, causing particularly convoluted constructions. For instance, Japanese MiiPro (aprm19990722.xml) contains the following XML string: \texttt{<w>kitenee<replacement><w>kite</w><w><shortening>i</shortening>nai</w>\\</replacement></w>}. This basically assumes two layers of target strings: the ultimate target \emph{kite inai} (two words) first gets shortened to \emph{kite nai}, which is then “replaced” by \emph{kitenee} (one word).

\subsubsection*{Repetitions}

Repetitions are specially marked when they diverge from the target syntax (where repetitions are sometimes expected, e.g.\ where full reduplication is a means of grammatical marking). TalkBank XML does this by embedding a special tag \texttt{<r>} with an attribute specifying the number of repetitions under \texttt{<w>}: \texttt{<g><w>...</w><r times="..."></g>}. In the ACQDIV Corpus repetitions are simply spelt out (n times the same word). The corresponding glosses are repeated, too.

\subsubsection*{Retracings}

Retracing in conversation analysis is the action of canceling an utterance at a given point in order to restart it or switch to a different utterance. TalkBank XML groups the complete group of words assumed to be part of a canceled utterance by \texttt{<g>} and uses the special tag \texttt{<k>} with a type attribute to mark what is happening: \texttt{<g><w>...</w><w>...</w><k type="retracing"/> </g>}. Retracings are not marked specially in the ACQDIV Corpus. However, when there are no glosses for the cancelled portion, the parser tries to suggest a gloss from similar parts of the rest of the utterance.


\subsection{Toolbox}
\label{subsec:Toolbox}

Toolbox is a textual format that is associated with the software of the same name and has been developed by SIL international. General documentation and links to downloads can be found at \url{http://www-01.sil.org/computing/toolbox/}. 

Typical Toolbox corpus files code sessions as trees where the three central levels are utterance, word, and morpheme, very much as in CHAT and TalkBank XML. However, differently from these, the syntactic coding of this structure is highly implicit. The syntactic unit corresponding to the utterance level is the record. Records are delimited by a record ID at the top and a double linebreak at the end. Each record may have several tiers consisting of a so-called field marker, which starts with a backslash and indicates the type of content (e.g.\ “\textbackslash ps” for parts of speech), and of the content itself (e.g.\ “adj”). The association of annotations with the three levels (utterance, word, morpheme) is not explicitly coded.

All elements on a tier (words or morphemes) are separated by spaces. Alignment across tiers works via indices: the first element on one tier (e.g.\ a segment) is associated with the first element on another (e.g.\ a gloss), the second with the second, and so on. The various other fields listed above are all on separate tiers in Toolbox.

Dependent morphemes are marked by morpheme separators on one side (e.g.\ “un-” for prefixes, “-able” for suffixes). These separators make it possible to reconstruct word boundaries on a tier focussing on morphemes. Sequences of the types stem-stem, stem-prefix, and suffix-stem can be inferred to belong to different words,  whereas stem-suffix, suffix-suffix, prefix-stem, and prefix-prefix must belong to the same word. A “floating separator” (morpheme separator with spaces on both sides) can be used to indicate that two stems belong to the same word (e.g.\ in the case of compounds: “apple - tree -s”). 

\autoref{fig:Toolbox example} shows an example for one record in a typical Toolbox file. 

\begin{figure}[ht!]
	\centering
	\includegraphics[scale=0.45]{pics/toolbox-screenshot.png}
	\caption{The first lines of a typical Toolbox file, opened in the Toolbox program}
	\label{fig:Toolbox example}
\end{figure}


\section{Chintang}
\label{sec:Chintang}

\subsection{Publication, accessibility, documentation}
The Chintang Language Corpus \citep{Stoll_etal2015b} was compiled between 2004 and 2015 in the course of several research projects now summarized as the \href{http://www.clrp.uzh.ch}{Chintang Language Research Program} (CLRP). It is documented in the \href{http://spwarran.uzh.ch/chintangwiki/index.php/Conventions_for_the_linguistic_analysis_of_Chintang}{Conventions for the linguistic analysis of Chintang} \citep{Schikowski2015a}. The standard citation for the language acquisition subcorpus, which is the portion included in the ACQDIV Corpus, is as follows: 

\begin{quote}
	Stoll, Sabine, Elena Lieven, Goma Banjade, Toya Nath Bhatta, Martin Gaenszle, Netra P.\ Paudyal, Manoj Rai, Novel Kishor Rai, 
	Ichchha P.\ Rai, Taras Zakharko, Robert Schikowski \& Balthasar Bickel. 2015. \emph{Audiovisual corpus on the acquisition of Chintang by six children.} % Stoll_etal2015b
\end{quote}

An older version of the corpus was published in the \href{http://dobes.mpi.nl/}{DoBeS archive} at the MPI Nijmegen. This version is now outdated and the publication guidelines are under revision. 


\subsection{Recording scheme}

\begin{table}[ht]
	\centering
	\begin{tabular}{ll}
		\toprule
		number of children 	& 7 (one canceled early) \\
		age ranges 			& 0;7.23-0;7.25, 0;6.30-1;11.12, 0;6.12-1;9.20, \\
							& 2;1.9-3;5.25, 2;0.29-3;5.13, 3;0.14-4;4.25, 2;11.2-4;3.14 \\
		recording rhythm 	& 4h per month (taken during several sessions within a single week) \\
		recording environment & mainly outside, close to home \\
		other speakers		& relatives, other children, passers-by \\
		other languages		& Nepali, Bantawa \\
		\bottomrule		
	\end{tabular}
	\caption{Recording scheme for the Chintang corpus}
	\label{tab:Chintang recording scheme}
\end{table}

% cf. text from mixing paper: 

% The climate in Chintang is warm and thus recordings were made mainly outdoors, either on the veranda of the houses or in the gardens and fields, where children spend most of their time. Children in Chintang usually play or roam around with other children. Starting from around 8 months of age they are carried around by other children and are part of larger groups of children (see \citealp{Lievenetal2013Early} for children's socio-communicative environment in Chintang).
%
% For the recordings, no instructions were given to the families other than that we were interested in the daily activities of children and their language development. Further, no restrictions were made concerning conversational partners or people present during recordings. As a result, we captured free play among a large number of children and natural conversational exchanges of many different people of varying ages.
%
% Each recording was conducted by a Nepalese research assistant familiar with the community, together with a native speaker assistant of Chintang who was part of the community and familiar with the children recorded. The assistants were instructed to intervene as little as possible in the activities recorded. The recordings were done with a video camera equipped with a fish-eye lens fixed on a tripod so as to interfere as little as possible by moving the camera. An external microphone was used to improve recording quality. The recorded interactions of adults within this corpus were mostly conversations on various household topics, chatting, gossip, politics, etc.

\subsection{File system and formats}

All files are located in a single folder. Files in the language acquisition subcorpus follow a naming scheme that is best understood on the base of examples such as “CLDLCh2R03S10” and “CLLDCh1\-R10S01”. The detailed rules are as follows: 

\begin{itemize*}
	\item “CL” (“Child Language”) is prefixed to all file names. 
	\item “DL” (“Days in the Life of”) is prefixed to baby sessions (range 0;6-2;0), “LD” (“Linguistic Development”) to sessions with older target children. 
	\item “Ch” combined with the following number indicates the speaker code of the target child.
	\item “R” and “S” (each with following numbers) indicate the recording cycle (= the number of the month, “01” being the first month in which recordings
		were taken for a child) and the number of the session within that month. 
\end{itemize*}

All corpus files are encoded as UTF-8 text. Tiers containing Chintang words frequently feature the special characters ⟨ŋ⟩, ⟨ɨ⟩, ⟨ʔ⟩, ⟨ṽ⟩ (Tilde on vowels, U+0303). Nepali translations contain Devanagari letters and punctuation. 

\subsection{Corpus format}

The input format used for the ACQDIV Corpus is \hyperref[subsec:Toolbox]{Toolbox}. Table \autoref{tab:Chintang tiers} shows how the fields in the ACQDIV Corpus are related to tiers in the input.

\begin{table}[ht!]
	\centering
	\begin{tabular}{lll}
		\toprule
			\textbf{target table} & \textbf{target field} & \textbf{source} \\
		\midrule 
			sessions 	& session\und id\und fk & file name \\
			utterances 	& utterance\und id		& \bks ref \\
			utterances 	& start					& \bks ELANBegin \\
			utterances 	& end					& \bks ELANEnd \\
			utterances 	& speaker\und label		& \bks ELANParticipant \\
			utterances 	& addressee				& \bks tos \\
			utterances 	& childdirected			& \bks tos \\
			utterances 	& sentence\und type		& utterance delimiter on \bks nep \\
			% utterances 	& phonetic				& \bks tx \\
			utterances 	& utterance\und raw		& \bks gw \\
			utterances 	& translation			& \bks eng \\
			% utterance & nepali				& \bks nep \\
			utterances 	& comment				& \bks comment \\ % and former variants, should now be corrected in source files: \com, \comm, \Comment etc.
			% disabled tiers
			% utterance	& argument\und coding & \bks anno \\
			% utterance & error\und coding	& - \\
			% utterance	& gestures			& \bks point and \bks object\und pointed\und at \\
			% utterance	& information\und structure & - \\
			% utterance	& phonetic\und target & - \\
			% utterance	& syntax\und coding	& \bks anno
			words	 	& word			& \bks gw \\
			% disabled tiers
			morphemes	& morpheme				& \bks mph \\ % actually segments_target, but that tier has been disabled 
			morphemes	& gloss\und raw			& \bks mgl \\ % actually glosses_target, but that tier has been disabled 
			morphemes	& pos\und raw			& \bks ps \\ % actually pos_target, but that tier has been disabled 
			morphemes	& morpheme\und language	& \bks lg \\
			% disabled tiers
			% morpheme 	& segments\und target & \bks mph \\
			% morpheme	& glosses\und target & \bks mgl \\
			% morpheme	& pos\und target	& \bks pos \\
		\bottomrule
	\end{tabular}
	\caption{Chintang tiers}
	\label{tab:Chintang tiers}
\end{table}

% Chintang has a lot of additional tiers that are ignored by the parser, either because they are outside our scope or because they are typos with frequencies < 100:
%
% * meaningful tiers:
% 	* \dt (date of last edit)
% 	* \ed (last editor)
% 	* \age + \agegroup (should be in IMDI)
% 	* \conv (is like \tx, just different transcription conventions)
% 	* \WordBegin + \WordEnd (irrelevant)
% 	* \rt (rough translation, \eng is always better)
% 	* \anno (referential annotations, irrelevant)
% 	* \object_pointed_at (irrelevant), \point (part of CSS coding, irrelevant)
% 	* \ELANMediaURL + \ELANMediaMIME + \ELANMediaExtracted (irrelevant)
% 	* \erg (outdated, now also in \mgl)
% 	* \grandfather (irrelevant)
% 	* \questions_for_Ravi (irrelevant)
% * garbage: \sg (unclear), \check, \intention, \word_in_utt, \type_of_sentence, \questions, \commen, \type_of_point, \DLCH3, \type_of_sentenceLNR, \type_of_oint, \txLDCh4, \gwLDCh4, \ctx, \ano, \Sections, \nepUNKNOWN, \convUnknown2, \u, \txUnknown2, \nepSaphal, \grandfather2, \UNKNOWN1, \Turns, \unknown, \type_of_pont, \tcode, \rop, \qst, \ne, \kond_sh, \grandfather3, \ge, \ety, \engl, \engToday, \engThat, \engIt, \engAnd, \eng...only, \eng...(he), \eng(We), \connected_pointingsLNR, \annohã, \anni, \ann, \Sangita, \EUDICOt1, \ELANMediaOrigin, \Dala

Morphology in the Chintang corpus is coded in the regular Toolbox format. 

% EOF Chintang


\section{Cree}
\label{sec:Cree}

\subsection{Publication, accessibility, documentation}
% projects, citation form, online at, license, docs

The Cree corpus \citep{Brittain2015a} is associated with the \href{http://www.mun.ca/cclas/}{Chisasibi Child Language Acquisition Study} (CCLAS), which started in 2004 and will continue until 2018. It should be cited as: 

\begin{quote}
Brittain, Julie. Corpus of the Chisasibi Child Language Acquisition Study (CCLAS). \url{http://childes.psy.cmu.edu/}.
\end{quote}

A fully anonymized version of a small subcorpus is freely available from CHILDES. This is also the subcorpus incorporated into the ACQDIV Corpus. 

Some documentation for the CCLAS corpus can be found in the Cree Auto-Parser Guide \citep{Acton2013a}.

\subsection{Recording scheme}

The following information holds for the subcorpus included in the ACQDIV Corpus (“Ani corpus”): 

\begin{table}[ht]
	\centering
	\begin{tabular}{ll}
		\toprule
		number of children 	& 1 \\
		age ranges 			& 2;1.14-3;8.24 \\
		recording rhythm 	& 30-40 min every 2-3 weeks \\
		recording environment & indoors at home \\
		other speakers		& mainly mother \\
		other languages		& English \\
		\bottomrule
	\end{tabular}
	\caption{Recording scheme for the Cree corpus}
	\label{tab:Cree recording scheme}
\end{table}

\subsection{File system and formats}

Cree file names are composed of an ascending number (for files within one subcorpus), a code for the target child, and the recording date in the format YYYY-MM-DD, e.g.\ “09-A1-2005-10-17”. Some files have an undocumented suffix “ms” behind the date, e.g.\ “10-A1-2005-11-21ms”. 

All files are encoded as UTF-8 text. Tiers containing Cree words frequently feature vowels from the ASCII set with an additional circumflex, e.g.\ ⟨â⟩ (U+00E2). There are phonetic transcriptions which feature a bigger set of IPA characters. 


\subsection{Corpus format}

The input format used for the ACQDIV Corpus is TalkBank XML (converted from \href{https://www.phon.ca/phontrac/wiki/Downloads}{Phon} by the Cree team). Table \autoref{tab:Cree tiers} shows how the tiers in the ACQDIV Corpus are related to tiers in the input.

\begin{table}[ht!]
	\centering
	\begin{tabular}{lll}
		\toprule
			\textbf{target table} & \textbf{target field} & \textbf{source tier} \\
		\midrule
			sessions 	& session\und id\und fk 	& /CHAT@Id \\
			utterances 	& utterance\und id	& //u@uID \\
			utterances 	& start\und raw		& //media@start \\
			utterances 	& end\und raw		& //media@end \\
			utterances 	& speaker\und label	& //u@who \\
			utterances 	& addressee			& - \\
			utterances 	& sentence\und type	& //u/t \\
			% utterances 	& phonetic			& //u/actual \\
			utterances 	& utterance\und raw	& //u//w \\
			utterances 	& translation		& //u/a[@type="english translation"] \\
			utterances 	& comments			& //u/a[@type="comments"] \\
			% disabled tiers
			% utterance	& argument\und coding & - \\ % might be present in selected files
			% utterance & error\und coding	& - \\ 
			% utterance	& gestures			& - \\
			% utterance	& information\und structure & - \\
			% utterance	& phonetic\und target & //u/model - \\
			% utterance	& syntax\und coding	& -  \\ % might be present in selected files (esp. annotation of passives)
			words	 	& word				& //u//w \\
			% disabled tiers
			morphemes	& morpheme			& //u/a[@type="extension" and @flavor="actmor"] \\
			morphemes	& gloss\und raw		& //u/a[@type="extension" and @flavor="mormea"] \\
			morphemes	& pos\und raw		& //u/a[@type="extension" and @flavor="mortyp"] \\
			morphemes	& morpheme\und language		& //u/a[@type="extension" and @flavor="mormea"], special gloss \texttt{Eng} \\
			% disabled tiers
			% morpheme 	& segments\und target & //u/a[@type="extension" and @flavor="tarmor"] \\
			% morpheme	& glosses\und target & - \\
			% morpheme	& pos\und target	& - \\
		\bottomrule
	\end{tabular}
	\caption{Cree tiers}
	\label{tab:Cree tiers}
\end{table}

\noindent The following peculiarities exist in the Cree input:

\begin{itemize}
	\item The phonetic transcription is taken from \texttt{//u/actual}. However, this node always has children \texttt{ph} and \texttt{ss}, 
		 each of which contains a single segment or suprasegmental, respectively. These single values are concatenated for the representation
		 in the ACQDIV Corpus. 
	\item \texttt{//u/w} may contain the string \texttt{missingortho} when it is empty. This is redundant and therefore removed. 
	\item \texttt{//u/w} contains “\und” as a morpheme separator. Since this is neither part of the orthography used in this tier  
		nor required for parsing the morphology, it is removed. 
	\item All tiers containing transcriptions (including morphology tiers) may contain semantically redundant square brackets at their edges \textendash\
		 these are removed.
\end{itemize}

\noindent The Cree morphology tiers are structured as follows: 

\begin{itemize}
	\item Words are separated by spaces, morphemes are separated by “\textasciitilde”. % the documentation says "-" marks unfinished words and some morpheme boundaries, but in the actual data "-" is used synonymously with “=”. JB says that -h [OBV] is motivated by the orthography: /-ch-h/ is conventionally spelt <ch-h>, not <chh>; therefore the morphology tiers also contain <ch-h> instead of <ch=h>). 
	% \item “_” marks morpheme boundaries within orthographical words in <w>
	\item “\%\%\%” indicates untranscribed words, “\#” is for unglossed elements. Both are replaced by \texttt{NULL/NA} (in isolation) or “???” (within strings).
	\item “?” is used instead of a gloss when the meaning of a morpheme is not clear. It may be isolated (e.g.\ “=?”, unclear suffix) or follow a form 
		(e.g.\ “=h?”, might be suffix \emph{-h}). This, too, is replaced by \texttt{NULL/NA}.
	\item “*” marks an element on the actmor or tarmor tier that does not correspond to an element on the other tier. It is replaced by \texttt{NULL/NA}. % Note that this is not used very consistently; often the tiers do not perfectly align but there is no *, or there is a * but the other tier doesn’t exist at all
	\item “.” and “+” connect two glosses to one. “,” adds an additional specification to a gloss, e.g.\ “p,quest” (question particle). “+” and “,” 
		 are replaced by the more standard “.”. % In the initial Ani corpus (10 files), "+" is only used as dem+G (“gestural feature”) and 
		 % vai+o (verb which is morphologically intransitive but syntactically transitive). "." is used in all other cases, i.e. anywhere 
		 % except with “G” and “o”).
		 % Similarly, “,” is only used as p,... (particle of a certain type), pro,... (pronoun), pvb,... (preverb). 
	\item Transitive agreement in the gloss tiers is marked by numbers connected by spaces and a greater than sign, e.g.\ “2 > 1”. The spaces are removed.
	\item Brackets indicate covert grammatical categories in the mortyp tier. In tarmor, they are used around abstract morphemes with no overt 
		morphological shape in order to make morpheme numbers match across tiers. The meaning of the individual abstract morphemes is not clear. 
		They are uppercased by the parser in order to emphasize their grammatical status. % , which might stand for ‘vowel’; the meaning of all other labels is completely unclear - apparently some student assistant inserted these labels long time ago. 
	\item “/” seems to mark semantic underspecification, e.g.\ “yellow/green”. % 1/2, foot/body
	\item “Eng” stands for any English word in the gloss tiers. The gloss is replaced by the English word itself. 
	% \item How to find the stem in Cree: stems can be discontinuous. The various parts of stems are labelled as “initial”, “medial” (rare), and “final”
	% 	in the actmor tier. Stems in the other tiers can only be assessed by comparison with this tier. No morpheme separator is added to affixes below
	% 	since this would potentially be confusing in a language with discontinuous stems.
\end{itemize}


\section{Indonesian}
\label{sec:Indonesian}

\subsection{Publication, accessibility, documentation}
The Indonesian corpus \citep{Gil_etal2007a} was collected at the \href{http://lingweb.eva.mpg.de/jakarta/acquisition.php}{Jakarta Field Station} of the Max Planck Institute for Evolutionary Anthropology between 1999 and 2004. It is officially cited as:

\begin{quote}
Gil, David \& Uri Tadmor. 2007. \emph{The MPI-EVA Jakarta Child Language Database. A joint project of the Department of Linguistics, Max Planck Institute for Evolutionary Anthropology and the Center for Language and Culture Studies, Atma Jaya Catholic University.} \url{https://jakarta.shh.mpg.de/acquisition.php}.
\end{quote}	

An earlier release of the full corpus is freely available at CHILDES. However, the most recent data can now be downloaded from the website of the MPI Jakarta Field Station, \url{https://jakarta.shh.mpg.de/acquisition.php}. Documentation is available on the same website and also in the \href{http://childes.psy.cmu.edu/manuals/10eastasian.pdf}{CHILDES manual for East Asian languages}. 

\subsection{Recording scheme}
% how many children, which age to which age, how many sessions per period, recording environment and instructions

\begin{table}[ht]
	\centering
	\begin{tabular}{ll}
		\toprule
		number of children 	& 10 \\
		age ranges 			& 1;6.15-4;11.29, 1;8.14-5;11.14, 1;9.15-6;1.5, 2;0.11-3;10.29, \\
							& 2;10.20-6;4.9, 2;7.4-6;0.20, 3;4.20-6;5.27, 4;6.0-8;9.29 \\
		recording rhythm 	& 45-60 min every 7-10 days \\
		recording environment & mostly indoors at home, sometimes outdoors \\
		other speakers 		& variety of children and adults \\
		other languages		& Javanese, Traditional Betawi, Toba Batak, others \\
		\bottomrule
	\end{tabular}
	\caption{Recording scheme for the Indonesian corpus}
	\label{tab:Indonesian recording scheme}
\end{table}

\subsection{File system and formats}
% e.g. COD-YYYY-MM-DD, differences between media and transcripts, doublets...

The Indonesian corpus has several subfolders containing subcorpora for each target child and named after their code. Within the folders, Toolbox files are named as “COD-DDMMYY” (where “COD” represents the speaker code of the target child) and XML files are named as “YYYY-MM-DD” with no indication of the target child. Note that this makes XML file names potentially ambiguous when taken out of their folders. 

Indonesian files are encoded as UTF-8 text and do not contain any non-ASCII characters. 

\subsection{Corpus format}

The corpus data for this corpus are taken from the Toolbox version while the metadata are based on the corresponding TalkBank XML files. Both formats have been converted from CHAT by the Indonesian team. Table \autoref{tab:Indonesian tiers} shows how the tiers in the ACQDIV Corpus are related to tiers in the input.

\begin{table}[ht!]
	\centering
	\begin{tabular}{lll}
		\toprule
			\textbf{target table} & \textbf{target field} & \textbf{source tier} \\
		\midrule
			sessions 	& session\und id\und fk 	& file name \\
			utterances 	& utterance\und id	& \bks ref \\
			utterances 	& start\und raw		& \bks begin \\
			utterances 	& end\und raw		& - \\
			utterances 	& speaker\und label	& \bks sp \\
			utterances 	& addressee			& - \\
			utterances 	& sentence\und type	& utterance delimiter on \bks tx \\
			% utterances 	& phonetic			& \bks pho \\
			utterances 	& utterance\und raw	& \bks tx \\
			utterances 	& translation		& \bks ft \\
			utterances 	& comment			& \bks nt \\
			% disabled tiers
			% utterance	& argument\und coding & - \\
			% utterance & error\und coding	& - \\
			% utterance	& gestures			& - \\
			% utterance	& information\und structure & - \\
			% utterance	& phonetic\und target & - \\
			% utterance	& syntax\und coding	& - \\

			words	 	& word		& \bks tx \\
			% disabled tiers
			morphemes	& morpheme			& \bks mb \\
			morphemes	& gloss\und raw		& \bks ge \\ % The tier \gj was ignored because it seems to basically contain the same information as \ge but without spaces between morphemes (so stems are harder to recognise).
			morphemes	& pos\und raw		& - \\
			% disabled tiers
			% morpheme 	& segments\und target & - \\
			% morpheme	& glosses\und target & - \\
			% morpheme	& pos\und target	& - \\

		\bottomrule
	\end{tabular}
	\caption{Indonesian tiers}
	\label{tab:Indonesian tiers}
\end{table}

\noindent Some peculiarities should be noted in the Indonesian input: 

\begin{itemize}
	\item The Indonesian Toolbox data contain CHAT constructs in the transcriptions (e.g.\ truncations as “(ba)nana”), 
		which are dealt with as the corresponding structures in the TalkBank XML corpora (see \autoref{subsec:TalkBank XML}).
		
		% overview of CHAT in Indonesian:
		%
		% | form 		| function  					| appears where 							| what to do |
		% |:----------|:------------------------------|:------------------------------------------|:--------------|
		% | . 		| utterance type “statement” 	| end of \tx, sometimes also end of \pho 	| set “utterance_type”, then delete |
		% | ? 		| utterance type “question”  	| end of \tx 								| set “utterance_type”, then delete |
		% | ! 		| utterance type “imperative” (different from CHAT, where this signifies “exclamation”) | end of \tx | set “utterance_type”, then delete |
		% | , 		| syntactic and/or intonational break | anywhere in \tx 					| delete |
		% | ...\s 	| interruption 					| anywhere in \tx 							| delete |
		% | ... 		| interruption/?hesitation 		| anywhere in \tx 							| delete |
		% | xx, xxx 	| unanalysable element. Documentation says “xx” is treated as word and “xxx” is not, but no difference is visible between the two in the actual corpus data. | single element on any tier | convert to “???” |
		% | “” 		| titles of songs, books etc.	| anywhere in \tx 							| delete |
		% | ‘’ 		| documentation says “quotations, roleplay” | does not appear in the actual corpus data | ignore |
		% | 0 		| action without vocalisation 	| fills \tx completely, always combined with comment in \nt | delete \tx, keep \nt |
		% | w(ord), wor(d) | fragments and shortenings | anywhere in \tx							| set “full_word” and “full_word_target” as appropriate |
		
	\item The first two records of many Toolbox file contain metadata imported from CHAT and dummy markers which do not convey any
		information. These contents have been put on ordinary Toolbox tiers, which have a different meaning in the body of Toolbox files 
		(see \autoref{tab:Indonesian tiers} above). Where this is the case these tiers are ignored. The following tiers may be affected: 
		
		\begin{table}[h!]
			\centering
			\begin{tabular}{lp{.55\textwidth}}
				\toprule
					\textbf{tier}	& \textbf{divergent content} \\
				\midrule
					\bks sp 	&  dummy marker @PAR (first record) or @Begin (second record) \\
					\bks tx 	&  speaker codes (CHAT @Participants, first record) or dummy marker @Begin (second record)\\
					\bks pho 	&  associated media file (CHAT @Filename) \\
					\bks ft 	&  duration of media file (CHAT @Duration) \\
					\bks nt 	&  comments on recording situation (CHAT @Situation) \\
				\bottomrule
			\end{tabular}
			\caption{Indonesian tiers with differing contents in the first two Toolbox records}
			\label{tab:Indonesian tiers with different contents in the first two Toolbox records}
		\end{table}
		
	\item Two different formats are used for speaker codes. The base format is found in CHAT and XML and consists of three uppercase letters. 
		Codes in this format are not unique \textendash\ for instance, “CHI” can stand for any of the eight target children and “MOT” for any
		of their mothers. In order to make codes unique, there is another, extended format where either a code for the target child is suffixed
		to the base code (e.g.\ “CHIHIZ”, “MOTHIZ” = child and mother in the subcorpus for the target child HIZ) or, in the case of researchers, 
		“EXP” is prefixed to the base code (e.g.\ “EXPBET”). This format is used in Toolbox and in a flat Excel metadata table. 
		The ACQDIV Corpus uses the extended format throughout. 
	\item Indonesian is the only corpus without any part-of-speech annotation. 
\end{itemize}

\noindent Morphology in the Indonesian corpus is coded in the regular Toolbox format (see \autoref{subsec:Toolbox}). 
% The main difference between \gj and \ge in Indonesian seems to be that \ge has the canonical spacing just sketched whereas \gj does not have spaces between morphemes. We use \ge.


\section{Inuktitut}
\label{sec:Inuktitut}

\subsection{Publication, accessibility, documentation}
The Inuktitut Corpus \citep{Allen2015a} was compiled for the work in \citet{Allen1996a}, which also contains some documentation. The corpus itself has not been published and is cited as: 

\begin{quote}
	Allen, Shanley. Unpublished. Allen Inuktitut Child Language Corpus.
\end{quote}

\subsection{Recording scheme}

\begin{table}[ht]
	\centering
	\begin{tabular}{ll}
		\toprule
		number of children 	& 4 \\
		age ranges 			& 2;6.6-3;3.2, 2;0.11-2;9.5, 2;6.2-3;2.26, 2;9.16-3;6.12 \\
		recording rhythm 	& 4h every month \\
		recording environment & indoors at home \\
		other speakers 		& relatives, friends \\
		other languages		& (little) English \\
		\bottomrule
	\end{tabular}
	\caption{Recording scheme for the Inuktitut corpus}
	\label{tab:Inuktitut recording scheme}
\end{table}

\subsection{File system and formats}
% txt, doc etc.

Recording sessions in the Inuktitut corpus may correspond to a single file or to a folder containing several transcripts associated with successive tape portions. Folders are named according to the scheme “COD”+“DDMMM” (where “COD” is the speaker code of the target child and “MMM” are three-letter month abbreviations, hence e.g.\ “ALI2APR”). Files within folders are named as “COD”+“DD”, an ascending number and/or letter for tape portions, and the suffix “TF”, e.g.\ “ALI71TF” in the folder “ALI7SEP”. Single files not placed in folders also start with “COD” but apart from that do not have clear naming conventions (e.g.\ “SUP11WM”). 

The original Inuktitut files come with a variety of file extensions (.XXS, .XXX, .NAC) which, however, amount to plain text (structured as CHAT). They are mostly encoded as ISO-Latin and contain a number of unexpected special characters (Inuktitut itself does not use non-ASCII characters, nor should any of the annotations). 

\subsection{Corpus format}

The input format for this corpus is TalkBank XML (cleaned and converted from CHAT by the ACQDIV core team). Table \autoref{tab:Inuktitut tiers} shows how the tiers in the ACQDIV Corpus are related to tiers in the input. 

\begin{table}[ht!]
	\centering
	\begin{tabular}{lll}
		\toprule
			\textbf{target table} & \textbf{target field} & \textbf{source tier} \\
		\midrule
			sessions 	& session\und id\und fk 	& /CHAT@Id \\
			utterances 	& utterance\und id	& //u@uID \\
			utterances 	& start\und raw		& //u/a[@type="timestamp"] \\
			utterances 	& end\und raw		& - \\
			utterances 	& speaker\und label	& //u@who \\
			utterances 	& addressee			& //u/a[@type="addressee"] \\
			utterances 	& childdirected		& //u/a[@type="addressee"] \\
			utterances 	& sentence\und type	& //u/t, //u/e \\
			% utterances 	& phonetic			& - \\
			utterances 	& utterance\und raw	& //u//w, //u/ga[@type="alternative"] \\
			utterances 	& translation		& //u/a[@type="english translation"] \\
			utterances 	& comment			& //u/a[@type="comments"], //u/a[@type="situation"] \\
			% disabled tiers
			% utterance	& argument\und coding & //u/a[@type="extension" and @flavor="arg"] \\
			% utterance & error\und coding	& //u/a[@type="errcoding"] \\
			% utterance	& gestures			& - \\
			% utterance	& information\und structure & - \\
			% utterance	& phonetic\und target & - \\
			% utterance	& syntax\und coding	& //u/a[@type="coding"] \\

			words	 	& word				& //u//w, //u/ga[@type="alternative"] \\
			% disabled tiers
			morphemes	& morpheme			& //u/a[@type="extension" and @flavor="mor"] \\
			morphemes	& gloss\und raw		& //u/a[@type="extension" and @flavor="mor"] \\
			morphemes	& pos\und raw		& //u/a[@type="extension" and @flavor="mor"] \\
			% disabled tiers
			% morpheme 	& segments & - \\
			% morpheme	& glosses & - \\
			% morpheme	& pos	& - \\

		\bottomrule
	\end{tabular}
	\caption{Inuktitut tiers}
	\label{tab:Inuktitut tiers}
\end{table}

\noindent Several things should be noted for the Inuktitut input:

\begin{itemize}
	\item While actual words are normally found in \texttt{//u/w}, occasionally \texttt{w} occurs under \texttt{//u/g} together with another
		 tag \texttt{ga} with the attribute \texttt{type="alternative"}. When this is done, \texttt{ga} contains the actual word 
		 and \texttt{w} contains the target word. 
	\item The corresponding construction on the morphology tier is an actual word followed by “[=? ...]”, where the brackets contain the guessed 
		target form. Occasionally there may be several target forms and several corresponding glosses, e.g.\ \texttt{<g><w>qaungatillaunga</w><ga type=“alternative”>maungatillaunga</ga><ga type=“alternative”>paungatillaunga</ga></g>}. In this case only the first target form is used.
	\item There are some pseudo-compounds where one element is always “xxx” e.g.\ \texttt{<w>ski-doo<wk type="cmp"/>xxx</w>}, gloss 		\texttt{VR|sikituuq\^{}ride\und snowmobile+xxx}. This also includes pseudo-compounds of the type “cli” (clitic), which is likewise meaningless.
		The “xxx” stands for an unidentified morpheme. This type of compound is treated like a normal word without any internal boundaries on the word
		level; on the morpheme level, “xxx” is treated as an unknown morpheme.
	\item Inuktitut has a special sentence type “broken for coding”, which indicates that there is no gloss but does not say anything 
		about the sentence type. A warning “not glossed” is inserted into the corpus object and the sentence type is set to “default”.
\end{itemize}

% * Ignored tiers: <ga type=“paralinguistics”>, <a type=“errcoding”>, <comment type=“...”> (session-level metadata), </action> and <happening> (seem to appear withing <e>, apparently without adding meaning)

\noindent The Inuktitut morphology tier (taken over from CHAT) has the following internal structure: 

\begin{itemize}
	\item Words are separated by spaces, morphemes by “+”. % Unlike in other XML corpora, there are no prefixes (#) or stem glosses (=) in Inuktitut.
	\item Each morpheme consists of three components (identical for lexical and grammatical morphemes):
		\begin{itemize*}
			\item The core element is a phonological form. 
			\item A POS tag is prefixed to this form, using “|” as the separator. Sometimes there are several POS tags, all separated by “|” 
				(e.g.\ “NN|DIM|apik”). Labels further to the right are interpreted as subcategories.
			\item A gloss is suffixed to the form, using “\^{}” as the separator.
		\end{itemize*}
	\item The following special characters are found within glosses:
		\begin{itemize*}
			\item “\und” connects several words that form a single gloss (e.g.\ “look\und for”). This remains unchanged. 
			\item “\&amp;” (sic) connects a stem gloss with a grammatical gloss (e.g.\ “here\&amp;SG\und ST”). This is replaced by more standard “.”.
			\item “@e” marks English words and is deleted. 
			\item Utterance terminators such as “.” or “?” are redundant on the morphology tier and therefore deleted. 
			\item “\&lt;”, “\&gt;” (sic) mark annotation groups in CHAT. They are ignored by the parser together with any associated annotations.
			% \item ’ = part of glosses (e.g.\ “I_don‘t_know”) - make sure this doesn’t break parser % this doesn't seem to be a problem anymore, I didn't find any examples
		\end{itemize*}
	\item Codes in square brackets are often found at the end of the morphology tier. Some of these are generic CHAT, others are specific to
		Inuktitut and have been documented in \citet{Allen1996a}. All of these codes are removed because they do not directly affect the
		interpretation of the morphology tier. The only exception is “[?]”, which indicates insecure glosses and is converted to a warning. 	
		% \begin{itemize*}
		% 	\item [+ SR] = exact repetition % mostly imperatives and questions
		% 	\item [+ PI] = partially intelligible (i.e.\ only a part of the utterance can be parsed)
		% 	\item [+ UI] = utterance not intelligible
		% 	\item [+ EX] = exclamation
		% 	\item [+ RO] = routine (e.g.\ song, alphabet) % mostly appears with singing and never is combined with any glosses
		% 	\item [+ TO] = incomplete utterance / interruption
		% 	\item [+ CQ] = coding questionable
		% 	\item [+ IM] = imitation
		% 	\item [?] = gloss insecure -> warning
		% 	\item [*] = speech error produced by child
		% 	\item [/], [//] = repetitions (already spelt out, no need to repeat!)
		% 	\item [=! ...] = comments on action accompanying speech -> delete
		% 	\item [=? ...] = comments on possible target form. When this is found the last gloss set needs to be reset to “actual” and its “target” will become the bracketed set.
		% \end{itemize*}
	\item The glossed form normally is associated with the target form, although glosses of the actual form are also found. Additional information on the relation between actual and target form is given in <a type=“errcoding”>, but the format is inconsistent, so it is impossible to exploit this tier.
	\item Untranscribed words are found as “xxx” on the morphology tier. 
\end{itemize}


\section{Japanese MiiPro}
\label{sec:Japanese MiiPro}

\subsection{Publication, accessibility, documentation}
The Japanese MiiPro Corpus \citep{Miyata_etal2009a, Nisisawa_etal2009a, Miyata_etal2010a, Nisisawa_etal2010a, Miyata2012a} was compiled between 1997 and 2010. The four subcorpora are cited as:

\begin{quote}
	% Miyata, Susanne \& Hiro Yuki Nisisawa. 2009-2010. \emph{MiiPro \textendash\ Corpus.} Pittsburgh, PA: TalkBank. \\
	Miyata, Susanne \& Hiro Yuki Nisisawa. 2009. \emph{MiiPro \textendash\ Asato Corpus.} Pittsburgh, PA: TalkBank. \\ % 1-59642-474-5.
	Miyata, Susanne \& Hiro Yuki Nisisawa. 2010. \emph{MiiPro \textendash\ Tomito Corpus.} Pittsburgh, PA: TalkBank. \\ % 1-59642-472-9.
	Nisisawa, Hiro Yuki \& Susanne Miyata. 2009. \emph{MiiPro \textendash\ Nanami Corpus.} Pittsburgh, PA: TalkBank. \\ % 1-59642-473-7.
	Nisisawa, Hiro Yuki \& Susanne Miyata. 2010. \emph{MiiPro \textendash\ ArikaM Corpus.} Pittsburgh, PA: TalkBank. \\ % 1-59642-475-3.
	Miyata, Susanne. 2012. \emph{Japanese CHILDES: The 2012 CHILDES manual for Japanese.} Available online at \url{http://www2.aasa.ac.jp/people/smiyata/CHILDESmanual/chapter01.html}.
\end{quote}

It is comprehensively documented in the \href{http://www2.aasa.ac.jp/people/smiyata/CHILDESmanual/chapter01.html}{CHILDES manual for Japanese} (in Japanese) and the \href{http://childes.psy.cmu.edu/manuals/10eastasian.pdf}{CHILDES manual for East Asian languages} (in English). 

\subsection{Recording scheme}
% how many children, which age to which age, how many sessions per period, recording environment and instructions

\begin{table}[ht]
	\centering
	\begin{tabular}{ll}
		\toprule
		number of children 	& 4 \\
		age ranges 			& 2;11.27-5;1.23, 2;11.28-5;0.17 (×2), 3;0.1-5;0.27 \\
		recording rhythm 	& 70 min per session, every week from 1;2 to 3;0, later every 1 or 2 months \\
		recording environment & indoors at home in limited area \\
		other speakers 		& mainly mother \\
		other languages		& none \\
		\bottomrule
	\end{tabular}
	\caption{Recording scheme for the Japanese MiiPro corpus}
	\label{tab:Japanese MiiPro recording scheme}
\end{table}

\subsection{File system and formats}
MiiPro files are composed of the code of the target child and the recording date as “YYYY MM DD” but without any separators, e.g.\ “aprm19990515”. The files are located in folders named after the target children.

All files are encoded as UTF-8 text. The orthography tiers contain CJK characters but are not taken over into the ACQDIV Corpus. All tiers included in the ACQDIV Corpus only contain ASCII characters. 

\subsection{Corpus format}
The input format for the ACQDIV Corpus is TalkBank XML (converted from CHAT by the MiiPro team). Table \autoref{tab:Japanese MiiPro tiers} shows how the tiers in the ACQDIV Corpus are related to tiers in the input.

\begin{table}[ht!]
	\centering
	\begin{tabular}{lll}
		\toprule
			\textbf{target table} & \textbf{target field} & \textbf{source tier} \\
		\midrule
			session 	& session\und id\und fk 	& /CHAT@Id \\
			utterance 	& utterance\und id	& //u@uID \\
			utterance 	& start\und raw		& //u/a[type="time stamp"], //u/media@start \\
			utterance 	& end\und raw		& //u/media@end \\
			utterance 	& speaker\und label	& //u@who \\
			utterance 	& addressee			& //u/a[@type="addressee"] \\
			utterance 	& childdirected		& //u/a[@type="addressee"] \\
			utterance 	& sentence\und type	& //u/t, //u/e \\
			% utterance 	& phonetic			& - \\
			utterance 	& utterance\und raw	& //u/w \\ % note that the Japanese corpora are the only ones where an orthography tier different from the set of words makes sense: the tier //u/a[@type="orthography"] contains Kanji-Kana script
			utterance 	& translation		& - \\
			utterance 	& comment			& //u/a[@type="situation"], //u/a[@type="actions"], //u/a[@type="comments"] \\
			% disabled tiers
			% utterance	& argument\und coding & - \\
			% utterance & error\und coding	& - \\
			% utterance	& gestures			& - \\
			% utterance	& information\und structure & //u/a[@type="coding"] \\
			% utterance	& phonetic\und target & - \\
			% utterance	& syntax\und coding	& - \\

			word	 	& word		& //u//w \\
			% disabled tiers
			morpheme	& morpheme			& //u/a[@type="extension" and @flavor="trn"] \\
			morpheme	& gloss\und raw		& //u/a[@type="extension" and @flavor="trn"] \\
			morpheme	& pos\und raw		& //u/a[@type="extension" and @flavor="trn"] \\
			morpheme	& morpheme\und language	& //u/w/langs \\
			% disabled tiers
			% morpheme 	& segments & - \\
			% morpheme	& glosses & - \\
			% morpheme	& pos	& - \\

		\bottomrule
	\end{tabular}
	\caption{Japanese MiiPro tiers}
	\label{tab:Japanese MiiPro tiers}
\end{table}

The MiiPro morphology tier has been taken over from CHAT without changes and has the following structure in the input:

\begin{itemize}
	\item Words are separated by spaces. There are no unique morpheme separators but various types of boundary markers. 
	\item If there are prefixes, they are always on the left edge of a word and separated from it by a “\#”. The prefix string consists of the phonological shape
		of the prefix without a gloss. 
	\item If there is a gloss for the stem, it is always on the right edge of the word and separated from it by a “=”.
	\item Apart from these special markers, words consist of one or (in the case of compounds) several blocks separated by “+”. 
	\item Each block in turn consists of a POS tag, a stem (phonological shape only), and optional suffixes (gloss only, no phonological shape). 
	\item An example for a minimal gloss is “v|mi-PST”, which is a verb with the stem shape \emph{mi} and a suffix with 
		the function ‘past’. In standard glossing the word form would be \emph{mi-ta} and the glosses would be “see-PST”. 
		Since the MiiPro corpus leaves the meaning of the stem (and prefixes) and the shape of suffixes open, the value \texttt{NULL/NA} is filled 
		in in the corresponding columns. 
	% \item example for a maximal MiiPro word: o#n|+v:c|fum+v:c|kir-SGER=crossing
	% \item equivalent in more standard notation (elements not retrievable from this type of gloss in brackets): segments o- fum (-i) =kir (-i), glosses (HON-) (step) (-LNK) (=cut) -NMLZ, POS pfx- v -sfx =v -sfx. ‘crossing’ is the meaning of the compound.
	\item Compounds may have an additional POS tag for the complete compound. In this case, the POS is prefixed in the usual form (xxx|) but there is no stem that follows.
\end{itemize}


\section{Japanese Miyata}
\label{sec:Japanese Miyata}

\subsection{Publication, accessibility, documentation}
% projects, citation form, online at, license, docs

The Japanese Miyata Corpus \citep{Miyata2004a, Miyata2004b, Miyata2004c, Miyata2012a} was collected between 1986 and 2004. The three subcorpora are cited as:

\begin{quote}
	% Miyata, Susanne \& Hiro Yuki Nisisawa. 2004. \emph{Miyata \textendash\ Corpus.} Pittsburgh, PA: TalkBank. \\
	Miyata, Susanne. 2004. \emph{Aki Corpus.} Pittsburgh, PA: TalkBank. 1-59642-055-3. \\
	Miyata, Susanne. 2004. \emph{Ryo Corpus.} Pittsburgh, PA: TalkBank. 1-59642-056-1. \\
	Miyata, Susanne. 2004. \emph{Tai Corpus.} Pittsburgh, PA: TalkBank. 1-59642-057-X. \\
	Miyata, Susanne. 2012. \emph{Japanese CHILDES: The 2012 CHILDES manual for Japanese.} Available online at \url{http://www2.aasa.ac.jp/people/smiyata/CHILDESmanual/chapter01.html}.
\end{quote}

Contentwise this corpus is closely related to the Japanese MiiPro Corpus. It is documented in the same resources, the \href{http://www2.aasa.ac.jp/people/smiyata/CHILDESmanual/chapter01.html}{CHILDES manual for Japanese} (in Japanese) and the \href{http://childes.psy.cmu.edu/manuals/10eastasian.pdf}{CHILDES manual for East Asian languages} (in English). 


\subsection{Recording scheme}
% how many children, which age to which age, how many sessions per period, recording environment and instructions

\begin{table}[ht]
	\centering
	\begin{tabular}{ll}
		\toprule
		number of children 	& 3 \\
		age ranges 			& 1;5.7-3;0.0, 1;4.3-3;0.30, 1;5.20-3;1.29 \\
		recording rhythm 	& 40-60 min every week \\
		recording environment & indoors at home \\
		other speakers 		& mainly mother \\
		other languages		& none \\
		\bottomrule
	\end{tabular}
	\caption{Recording scheme for the Japanese Miyata corpus}
	\label{tab:Japanese Miyata recording scheme}
\end{table}

\subsection{File system and formats}
\label{subsec:Miyata file system and formats}
% e.g. COD-YYYY-MM-DD, differences between media and transcripts, doublets...
% txt, doc etc.

The Miyata corpus as published on CHILDES contains several files for every session. These files code the same content and are therefore doublets (or triplets), which seem to represent different workflow stages. The files come in three folders named after the target children and with partially diverging file naming conventions: 

\begin{itemize*}
	\item The folder “Aki” contains three files per session. Series 1 is named as “aki” and the age of the child at the time of recording in the format “YMMDD”, 
		e.g.\ “aki10507” (= 1 year, 5 months, 7 days). Series 2 is named as “aki” combined with ascending numbers (“aki01” to “aki56”). Series 3 combines 
		ascending numbers and age but does not contain the code of the child, e.g.\ “50\und 21020”. 
	\item The folder “Ryo” contains four files per session. The names for all files contain the age of Ryo in the same format as for Aki. Series 1 has the 
		prefix “ryo” (“ryo10303”), series 2 has “yo”, series 3 has “r”, and series 4 has no prefix at all. 
	\item The folder “Tai” contains four files per session. The file names of series 1 and 2 are composed of “tai” and “t”, respectively, and the recording date 
		as “YYMMDD” (“tai931125”, “t931125”). Series 3 has “tai” combined with the age in the format already described (“tai21114”), and series 4 has 
		ascending numbers combined with age (“36\und 20220”). 
\end{itemize*}

All files are encoded as UTF-8 text. The orthography tiers contain CJK characters but are not taken over into the ACQDIV Corpus. All tiers included in the ACQDIV Corpus only contain ASCII characters. 


\subsection{Corpus format}

The Japanese Miyata corpus is semantically very similar to the \hyperref[sec:Japanese MiiPro]{MiiPro corpus}, with which it shares the author. However, the Miyata corpus does not contain any traces of CHAT but is completely formatted as TalkBank XML, which also serves as the input format for the ACQDIV Corpus. It is documented in the \href{http://childes.psy.cmu.edu/manuals/10eastasian.pdf}{CHILDES manual for East Asian languages}. Table \autoref{tab:Japanese Miyata tiers} shows the associations between tiers in the input and in the ACQDIV Corpus.

\begin{table}[ht!]
	\centering
	\begin{tabular}{lll}
		\toprule
			\textbf{target table} & \textbf{target field} & \textbf{source tier} \\
		\midrule
			sessions 	& session\und id\und fk 	& /CHAT@Id \\
			utterances 	& utterance\und id	& //u@uID \\
			utterances 	& start\und raw		& //u/a[type="time stamp"], //u/media@start \\
			utterances 	& end\und raw		& //u/media@end \\
			utterances 	& speaker\und label	& //u@who \\
			utterances 	& addressee			& //u/a[@type="addressee"] \\
			utterances 	& childdirected		& //u/a[@type="addressee"] \\
			utterances 	& sentence\und type	& //u/t, //u/e \\
			% utterances 	& phonetic			& - \\
			utterances 	& utterance\und raw	& //u/w \\ % note that the Japanese corpora are the only ones where an orthography tier different from the set of words makes sense: the tier //u/a[@type="orthography"] contains Kanji-Kana script
			utterances 	& translation		& - \\
			utterances 	& comment			& //u/a[@type="situation"], //u/a[@type="actions"], //u/a[@type="comments"], //u/a[@type="explanation"] \\
			% disabled tiers
			% utterance	& argument\und coding & - \\
			% utterance & error\und coding	& //u/a[@type="errcoding"] \\
			% utterance	& gestures			& //u/a[@type="gesture"] \\
			% utterance	& information\und structure & //u/a[@type="coding"] \\
			% utterance	& phonetic\und target & - \\
			% utterance	& syntax\und coding	& - \\

			words	 	& word		& //u/w \\
			% disabled tiers
			morphemes	& morpheme			& //u/w/mor//mpfx, //u/w/mor//stem, //u/w/mor//mk \\ % actually segments_target, but that tier has been disabled
			morphemes	& gloss\und raw		& //u/w/mor//menx \\ % actually glosses_target, but that tier has been disabled
			morphemes	& pos\und raw		& //u/w/mor//pos/c, //u/w/mor//pos/s \\ % actually pos_target, but that tier has been disabled
			morpheme	& morpheme\und language			& //u/w/langs \\
			% disabled tiers
			% morpheme 	& segments\und target & //u/w/mor//mpfx, //u/w/mor//stem, //u/w/mor//mk \\
			% morpheme	& glosses\und target & //u/w/mor//menx \\
			% morpheme	& pos\und target	& //u/w/mor//pos/c, //u/w/mor//pos/s \\

		\bottomrule
	\end{tabular}
	\caption{Japanese Miyata tiers}
	\label{tab:Japanese Miyata tiers}
\end{table}

The transcription tier in the Japanese Miyata Corpus is incomplete in that utterances of the mother have often been omitted. These omissions are not marked, so the Miyata data are not suitable for studying child-surrounding speech or adult language in general. 

The Miyata input also has some peculiarities in its morphology coding: 

\begin{itemize}
	\item Different morphological components have their own tags: prefixes are coded by \texttt{<mpfx>} (under the morphological word \texttt{<mw>}) or
		under the compound group \texttt{<mwc>}), stems by \texttt{<stem>} (under \texttt{<mw>}), and suffixes by \texttt{<mk>} (under \texttt{<mw>}). 
		Glosses are only given for stems and are coded by \texttt{<menx>} (under \texttt{<mw>} or \texttt{<mwc>}). 
	\item Some clitics (e.g.\ honorifics) are regularly glossed, but the glosses appear in \texttt{<menx>} rather than \texttt{<mk>}. These glosses
		are moved to the right place by the parser. 
	\item Some suffixes have a type attribute “fused”. These are suffixes with no clear phonological shape which are fused with their stem. 
		The glosses of such suffixes are joined to that of the preceding stem using the conventional separator “.”. 
	\item Part-of-speech tags are not given directly in \texttt{<pos>} but in the child nodes \texttt{<c>} “category” and \texttt{<s>} “subcategory”.
	\item Compounds are coded for on the morphology tier. When there is a compound, the node \texttt{<mwc>} appears directly under \texttt{<mor>}
		with its own part-of-speech group. Prefixes and morphological words are also under \texttt{<mwc>} in this case. 
		The ACQDIV Corpus ignores compounding, so the stems are concatenated using “=”. The POS tag is taken from the top level rather than from the 
		individual words. 
	\item The glosses for words containing replacements are given \emph{within} the \texttt{<replacement>} tag. 
\end{itemize}

% Struc of fully expanded mor tier looks like this:
%
% ├── <w>
% 	 └── <mor>
% 		└──  <mw> = morphological word(s)
% 			└──  <pos> = POS for word
% 				└──  <c>, <s> (concatenate!) = main POS and subcategories
% 			├──  <mpfx> = prefixes
% 			├──  <stem> = stem of word
% 			├──  <mk> = suffixes of different types
% 			└──  <menx> = meaning of word (present when there is only one <mw>)
%
% Or like this in the case of compounds:
%
% ├── <w>
% 	 └── <mor>
% 	     └── <mwc> = group of compound elements
% 			└──  <pos> = POS for compound
% 				└──  <c>, <s> = main POS and subcategories
% 	 		└── <mpfx> = prefixes (may precede <pos>!)
%  			└──  <mw> = morphological word(s)
%  				├──  <stem> = stem of word
%  				└──  <mk> = suffixes of different types
% 			└──  <menx> = meaning of compound (when there are several <mw> under a <mwc>)


\section{Nungon}
\label{sec:Nungon}

\subsection{Publication, accessibility, documentation}
The Sarvasy Nungon Corpus \citep{Sarvasy2017a, Sarvasy2017b} was compiled between 2015 and 2017. Two resources should be cited: 

\begin{quote}
	Sarvasy, Hannah. Sarvasy Nungon Corpus. Available online at \href{http://childes.talkbank.org/access/Other/Nungon/Sarvasy.html}{http://childes.talkbank.org}. \\ % Sarvasy2017a
	Sarvasy, Hannah. 2017. \emph{A Grammar of Nungon: A Papuan Language of Northeast New Guinea}. Leiden: Brill. % Sarvasy2017b
\end{quote}

Some documentation is available on the \href{http://childes.talkbank.org/access/Other/Nungon/Sarvasy.html}{CHILDES website}. 

\subsection{Recording scheme}
% how many children, which age to which age, how many sessions per period, recording environment and instructions

\begin{table}[ht]
	\centering
	\begin{tabular}{ll}
		\toprule
		number of children 	& 5 \\
		age ranges 			& 2;1-4;1, 2;10-4;10, 3;5-5;5, 3;8-5;8, 1;2-2;3 \\
		recording rhythm 	& 1 continuous hour per month \\
		recording environment & natural environment \\
		other speakers 		& various \\
		other languages		& Tok Pisin \\
		\bottomrule
	\end{tabular}
	\caption{Recording scheme for the Nungon corpus}
	\label{tab:Nungon recording scheme}
\end{table}

\subsection{File system and formats}
So far files are not yet systematically named, but the schema “code-age” is emerging where both code and age refer to the target child, e.g.\ “TowetOe-020310”. The folder structure is not final yet. 

All files are encoded as UTF-8 text. There is a single but frequent non-ASCII character ⟨ö⟩. 

\subsection{Corpus format}
% TODO
The input format for the ACQDIV Corpus is TalkBank XML (converted from CHAT). Table \autoref{tab:Nungon tiers} shows how the tiers in the ACQDIV Corpus are related to tiers in the input.

\begin{table}[ht!]
	\centering
	\begin{tabular}{lll}
		\toprule
			\textbf{target table} & \textbf{target field} & \textbf{source tier} \\
		\midrule
			session 	& session\und id\und fk 	& file name \\
			utterance 	& utterance\und id			& //u@uID \\
			utterance 	& start\und raw				& //u/media@start \\
			utterance 	& end\und raw				& //u/media@end \\
			utterance 	& speaker\und label			& //u@who \\
			utterance 	& sentence\und type			& //u/t \\
			utterance 	& utterance\und raw			& //u/w \\ 
			utterance 	& translation				& //u/a[@type="english translation"] \\
			utterance 	& comment					& //u/a[@type="comments"] \\
			word	 	& word						& //u//w \\
			morpheme	& morpheme					& //u/a[@type="target gloss"] \\
			morpheme	& gloss\und raw				& //u/a[@type="extension" and @flavor="cod"] \\
			morpheme	& pos\und raw				& //u/a[@type="extension" and @flavor="cod"] \\
			morpheme	& morpheme\und language		& //u/a[@type="extension" and @flavor="cod"] \\
		\bottomrule
	\end{tabular}
	\caption{Nungon tiers}
	\label{tab:Nungon tiers}
\end{table}

The morphology tiers in the Sesotho input are structured as follows: 

\begin{itemize}
	\item Words on the target gloss tier are separated by spaces, morphemes are separated by hyphens. Since prefixes and suffixes have the same separators
		and there are no spaces between them and stems, stem boundaries can only be reconstructed from comparison with the cod(ing) tier. 
		Clitics are separated by “=” and correspond to independent words in \texttt{//u/w}. They are therefore split off in parsing. 
	\item On the coding tier the same rules apply. Within complex words, the stem is the morpheme to which the POS tag is prefixed 
		(e.g.\ “n\^{}”, “v\^{}”). 
	\item Untranscribed words are coded as “xxx” on the word tier and are not morphologically coded. 
\end{itemize}


\section{Russian}
\label{sec:Russian}

\subsection{Publication, accessibility, documentation}

The Russian Corpus \citep{Stoll_etal2008a} was compiled for the work in \citet{Stoll2001a} but was only finished later. The corpus itself has not been published and is cited as:

\begin{quote}
	Stoll, Sabine \& Roland Meyer. 2008. Audio-visual longitudinal corpus on the acquisition of Russian by 5 children.
\end{quote}

\noindent The corpus is also known as the “Stoll Russian Corpus” (hence the acronym StRuC used in this document). There is no official documentation available. 


\subsection{Recording scheme}

\begin{table}[ht!]
	\centering
	\begin{tabular}{ll}
		\toprule
		number of children 	& 5 \\
		age ranges 			& 1;3.26-4;11.0, 1;4.22-5;6.26, 1;6.10-5;4.18, 1;11.28-4;3.14, 3;1.8-6;8.12 \\
		recording rhythm 	& 1h every week \\
		recording environment & indoors at home \\
		other speakers 		& mother and relatives \\
		other languages		& none \\
		\bottomrule
	\end{tabular}
	\caption{Recording scheme for the Russian corpus}
	\label{tab:Russian recording scheme}
\end{table}


\subsection{File system and formats}

The Russian corpus consists of several parallel versions in separate numbered folders. While most of the folders build on each other (for instance, “4a\und tbx\und lemma\und separated\und timecodes\und lgr” takes over all information from “4\und tbx\und lemma\und separated\und timecodes\und lgr” but adds glosses modified according to the Leipzig Glossing Rules), some also contain conflicting information (for instance, “6\und elan\und coded\und pointing” contains annotations which are missing from the mentioned folders but at the same time does not have LGR glosses). As indicated by the folder names, the versions are distinguished by varying formats and annotation layers. 

Within that folder, files are named as “code + session number + age”, where code is the first letter of the target child code, session number is a three-digit ascending number, and age is the age of the target child given as “YMMDD”, e.g.\ “A03120419”. 

Most files were encoded as UTF-8 text with some exceptions in ISO-Latin. For the ACQDIV Corpus the original data were all reencoded to UTF-8. The Russian data do not contain any non-ASCII characters. 


\subsection{Corpus format}

The input format of the Russian corpus is hybrid Toolbox/CHAT (converted from CHAT by the Russian team). All files contain CHAT-style metadata headers and Toolbox-like bodies with frequent traces of CHAT on the transcription tier and elsewhere. Table \autoref{tab:Russian tiers} shows how the tiers in the ACQDIV Corpus are related to tiers in the input.

\begin{table}[ht!]
	\centering
	\begin{tabular}{lll}
		\toprule
			\textbf{target table} & \textbf{target field} & \textbf{source tier} \\
		\midrule
			sessions 	& session\und id\und fk 	& file name \\
			utterances 	& utterance\und id	& \bks ref \\
			utterances 	& start\und raw		& \bks ELANBegin \\
			utterances 	& end\und raw		& \bks ELANEnd \\
			utterances 	& speaker\und label	& \bks EUDICOp \\
			utterances 	& addressee			& \bks add \\
			utterances 	& utterance\und raw & \bks text \\
			utterances 	& sentence\und type	& utterance delimiter on \bks text \\
			utterances 	& comment			& \bks act, \bks com, \bks ct, \bks err, \bks sit \\
			% disabled tiers
			% utterance	& argument\und coding & - \\
			% utterance & error\und coding	& \bks err \\ % this is presently merged with \com in the comment tier
			% utterance	& gestures			& \\
			% utterance	& information\und structure & \\
			% utterance	& phonetic\und target & \\
			% utterance	& syntax\und coding	& \\

			words	 	& word				& \bks text \\
			% disabled tiers
			morphemes	& morpheme			& \bks lem \\
			morphemes	& gloss\und raw		& \bks mor \\
			morphemes	& pos\und raw		& \bks mor \\
			morphemes	& morpheme\und language			& \bks mor, special gloss FOREIGN \\
			% disabled tiers
			% morpheme 	& segments\und target & - \\
			% morpheme	& glosses\und target & - \\
			% morpheme	& pos\und target	& - \\

		\bottomrule
	\end{tabular}
	\caption{Russian tiers}
	\label{tab:Russian tiers}
\end{table}

Several points to be noted concern the morphology tiers in the input: 

\begin{itemize}
	\item There is no segmentation. Instead, words are analyzed on \texttt{\bks mor} using long strings of concatenated glosses. 
		Presently the lemmatization tier \texttt{\bks lem} is interpreted as if it contained segments for the sake of uniformity across
		the ACQDIV subcorpora. 
	\item The elements on \texttt{\bks mor} are separated by spaces and contain both glosses and POS, which are in turn separated by “-” or “:” 
		according to the following rules:
		\begin{itemize*}
			\item Sub-POS are always separated by “-” (e.g.\ \texttt{PRO-DEM-NOUN}), subglosses are always separated by “:” (e.g.\ \texttt{PST:SG:F}). 
				What varies is the character that separates POS from glosses in the word string.
			\item If the POS is \texttt{V} (’verb‘) or \texttt{ADJ} (’adjective‘), the glosses start behind the first “-”, 
				e.g.\ \texttt{V-PST:SG:F:IRREFL:IPFV} → POS \texttt{V}, gloss \texttt{PST.SG.F.IRREFL.IPFV}.
			\item For all other POS, the glosses start behind the first “:”, e.g.\ \texttt{PRO-DEM-NOUN:NOM:SG} → 
				POS \texttt{PRO.DEM.NOUN}, gloss \texttt{NOM.SG}.
			\item If there is no “:” in a word string, gloss and POS are identical (most frequently the case with \texttt{PCL} ’particle‘).
		\end{itemize*}
\end{itemize}

Also note that overlaps are regularly transcribed twice in the Russian corpus (once in the interrupted utterance, then once again in a separate record with the
right speaker). This could not be corrected in the ACQDIV representation. % Overlaps are found on \text and have the form \[[a-z].*?\]

% The following tiers were ignored:
%
% * \age (but note: only the ages for children are given in the header, the rest can only be inferred from the CHAT roles, e.g.\ Aunt = adult)
% * \tim % because time stamps only occur occasionally and because the format is completely unclear


\section{Sesotho}
\label{sec:Sesotho}

\subsection{Publication, accessibility, documentation}
% projects, citation form, online at, license, docs

The Sesotho corpus \citep{Demuth1992b, Demuth2015a} was compiled between 1980 and 1990. Citations should mention the corpus and one following paper: 

\begin{quote}
	Demuth, Katherine. Demuth Sesotho Corpus. \href{http://childes.talkbank.org/access/Other/Sesotho/Demuth.html}{http://childes.psy.cmu.edu/}. \\
	Demuth, Katherine. 1992. Acquisition of Sesotho. In Dan Slobin (ed.), \emph{The Cross-Linguistic Study of Language Acquisition}, vol.\ 3, 557-638. Hillsdale, N.J.: Lawrence Erlbaum Associates.
\end{quote}

\subsection{Recording scheme}

\begin{table}[ht]
	\centering
	\begin{tabular}{ll}
		\toprule
		number of children 	& 4 \\
		age ranges 			& 2;1-3;0, 2;1-3;2, 2;4-3;3, 3;8-4;7 \\
		recording rhythm 	& 3-4 hours every month \\
		recording environment & home and neighborhood \\
		other speakers 		& relatives, other children, passers-by \\
		other languages		& none \\
		\bottomrule
	\end{tabular}
	\caption{Recording scheme for the Sesotho corpus}
	\label{tab:Sesotho recording scheme}
\end{table}

\subsection{File system and formats}
% e.g. COD-YYYY-MM-DD, differences between media and transcripts, doublets...
% txt, doc etc.

Examples for file names in the published corpus are “hiib” and “tvie”. These names are composed of three elements: 

\begin{itemize*}
	\item the first letter of the target child
	\item an ascending roman number (counting sessions within that child)
	\item an ascending lowercase letter indicating several recording sessions which come from the same period of intensive recording 
		 within a month but may or may not be adjacent. Sessions of this type also correspond to separate media files. 
\end{itemize*}

All files are encoded as UTF-8 text and only contain ASCII characters. 


\subsection{Corpus format}

The input format of the Sesotho corpus is TalkBank XML (converted from CHAT by the Sesotho team). Table \autoref{tab:Sesotho tiers} shows how the tiers in the ACQDIV Corpus are related to tiers in the input.

\begin{table}[ht!]
	\centering
	\begin{tabular}{lll}
		\toprule
			\textbf{target table} & \textbf{target field} & \textbf{source tier} \\
		\midrule
			sessions 	& session\und id\und fk 	& /CHAT@Id \\
			utterances 	& utterance\und id	& //u@uID \\
			utterances 	& start\und raw		& //u/media@start \\
			utterances 	& end\und raw		& //u/media@end \\
			utterances 	& speaker\und label	& //u@who \\
			utterances 	& addressee			& - \\
			utterances 	& sentence\und type	& //u/t \\
			% utterances 	& phonetic			& - \\
			utterances 	& utterance\und raw	& //u/w \\
			utterances 	& translation		& //u/a[@type="translation"] \\
			utterances 	& comment			& //u/a[@type="situation"] \\
			% disabled tiers
			% utterance	& argument\und coding & - \\
			% utterance & error\und coding	& - \\
			% utterance	& gestures			& - \\
			% utterance	& information\und structure & - \\
			% utterance	& phonetic\und target & - \\
			% utterance	& syntax\und coding	& - \\

			words	 	& word		& //u/w \\
			% disabled tiers
			morphemes	& morpheme			& //u/a[@type="target gloss"] \\
			morphemes	& gloss\und raw		& //u/a[@type="coding"] \\
			morphemes	& pos\und raw		& //u/a[@type="coding"] \\
			% disabled tiers
			% morpheme 	& segments & - \\
			% morpheme	& glosses & - \\
			% morpheme	& pos & - \\

		\bottomrule
	\end{tabular}
	\caption{Sesotho tiers}
	\label{tab:Sesotho tiers}
\end{table}

The morphology tiers in the Sesotho input are structured as follows: 

\begin{itemize}
	\item Words on the target gloss tier are separated by spaces, morphemes are separated by hyphens. Since prefixes and suffixes have the same separators
		and there are no spaces between them and stems, stem boundaries can only be reconstructed from comparison with the coding tier. 
	\item On the coding tier the same rules apply; however, many glosses (esp.\ for noun classes) contain brackets within which spaces do not count
		as word separators. Any orthographic word with only one morpheme is a stem. Within complex words, the stem is the morpheme starting with
		“n\^{}”, “v\^{}”, or “id\^{}”, or ending with “aj”, “nm”, or “ps” and a sequence of digits. 
	\item “\und” connects two glosses to one, e.g.\ “come\und out”, “t\^{}...\und v\^{}”. % = 'covert tense + verb form'
	\item Brackets after a noun most frequently indicate noun classes. There are always two noun classes (possibly corresponding to singular
		and plural) separated by a comma, and sometimes several such pairs may appear separated by semicolons. 
		Noun classes are kept with their brackets, but spaces within the brackets are deleted. 
	\item Contracted forms which do not leave any traces at the surface also appear in brackets \textendash\ these are completely removed with their contents. 
	\item Finally, morphemes may occur in brackets without a documented meaning. In this last case only the brackets are removed and the content is kept. 
	\item Parts of speech are incorporated into the coding tier in various ways. Verbs and ideophones have prefixes “v\^{}” and “id\^{}”, respectively. 
		Nouns can be recognized by their noun class brackets. For all other parts of speech the gloss itself is the part of speech and a true
		gloss reflecting the semantics is missing. % This is the case for aj ADJ, ps\d+ POSS, pn\d+ PRON, d\d+ DEM, nm NUM, cp\d* COP, (cm|ht|ng|cd|loc) PTCL, cj CONJ, pr PREP, av ADV, ij INTJ, wh ITRG
	\item Noun class prefixes are given in the form “n\^{}” followed by a sequence of digits. % * [nN]\^\d[ab]?- = noun class of noun; occasional mistake: n^\d[ab]?word without hyphen
	\item Proper nouns are given as “n\^{}” followed by “name”, “place”, “game”, or “song”. 
	\item Untranscribed words are found as “xxx” on the morphology tier. 
\end{itemize}

% both tiers contain punctuation [.?!]
% words in "coding" may contain <‘>
% * [tf]^(p|pf|f\d|\w+) = tense -> Ø
% * m^(in|pt|x|\w+) = mood -> Ø


\section{Turkish}
\label{sec:Turkish}

\subsection{Publication, accessibility, documentation}
% projects, citation form, online at, license, docs

The Turkish corpus \citep{Kuntay2015a} has not been published. It should be cited as 

\begin{quote}
	Küntay, Aylin Copty, Dilara Koçbaş, Süleyman Sabri Taşçı. Unpublished. Koç University Longitudinal Language Development Database on language acquisition of 8 children from 8 to 36 months of age. 
\end{quote}

\noindent There is no official documentation available. 

\subsection{Recording scheme}
% how many children, which age to which age, how many sessions per period, recording environment and instructions

\begin{table}[ht]
	\centering
	\begin{tabular}{ll}
		\toprule
		number of children 	& 8 \\
		age ranges 			& 1;0.2-3;0.3, 0;7.28-3;0.24, 0;8.6-3;0.14, 0;8.1-1;9.28, \\
							& 0;8.0-2;4.20, 0;8.2-3;0.14, 0;8.30-3;0.20, 0;9.27-2;9.13 \\
		recording rhythm 	& 1h every 2 weeks \\
		recording environment & indoors at home \\
		other speakers 		& variety of children and adults \\
		other languages		& none \\
		\bottomrule
	\end{tabular}
	\caption{Recording scheme for the Turkish corpus}
	\label{tab:Turkish recording scheme}
\end{table}


\subsection{File system and formats}

File names consist of the code of the target child, an ascending number for counting sessions within that child, the recording date (DDMMMYY), and the age at that time (YY-MM-DD), e.g.\ “burcu45\und 10apr04\und 02-06-20”. Files are located in folders named after the target children.

The original Turkish CHAT files come with mixed encodings, most prominently UTF-8 and ISO-Latin, and contain a plethora of unintended special characters. The only special characters that are well-formed by the criteria of the Turkish orthography are ⟨ç⟩, ⟨ğ⟩, ⟨ö⟩, ⟨ş⟩, ⟨ü⟩.


\subsection{Corpus format}

The input format of the Turkish KULLDD corpus is TalkBank XML (converted from CHAT by the ACQDIV core team and the KULLDD team). Table \autoref{tab:Turkish tiers} shows how the tiers in the ACQDIV Corpus are related to tiers in the input.

\begin{table}[ht!]
	\centering
	\begin{tabular}{lll}
		\toprule
			\textbf{target table} & \textbf{target field} & \textbf{source tier} \\
		\midrule
			sessions 	& session\und id\und fk 	& /CHAT@Id \\
			utterances 	& utterance\und id	& //u@uID \\
			utterances 	& start\und raw		& //u/a[@type="time stamp"] \\
			utterances	& end\und raw		& - \\
			utterances 	& speaker\und label	& //u@speaker \\
			utterances 	& addressee			& //u/a[@type="addressee"] \\
			utterances 	& childdirected		& //u/a[@type="addressee"] \\
			utterances 	& sentence\und type	& //u/t, //u/e \\
			% utterance 	& phonetic			& - \\
			utterances 	& utterance\und raw	& //u/w \\
			utterances 	& translation		& //u/a[@type="english translation"] \\
			utterances 	& comment			& //u/a[@type="explanation"], //u/a[@type="actions"] \\
			% disabled tiers
			% utterance	& argument\und coding & \\
			% utterance & error\und coding	& //u/a[@type="errcoding"] \\ % not used consistently
			% utterance	& gestures			& //u/a[@type="gesture"] \\
			% utterance	& information\und structure & \\
			% utterance	& phonetic\und target & - \\
			% utterance	& syntax\und coding	& \\

			words	 	& word		& //u/w \\
			% disabled tiers
			morphemes	& morpheme			& //u/a[@type="extension" and @flavor="mor"] \\ % not really coded consistently in KULLDD - these may be both actual or target segments 
			morphemes	& gloss\und raw		& //u/a[@type="extension" and @flavor="mor"] \\ % ditto
			morphemes	& pos\und raw		& //u/a[@type="extension" and @flavor="mor"] \\ % ditto
			morphemes	& morpheme\und language			& //u/w/langs \\
			% disabled tiers
			% morpheme 	& segments & //u/a[@type="extension" and @flavor="mor"] \\
			% morpheme	& glosses & //u/a[@type="extension" and @flavor="mor"] \\
			% morpheme	& pos & //u/a[@type="extension" and @flavor="mor"] \\

		\bottomrule
	\end{tabular}
	\caption{Turkish tiers}
	\label{tab:Turkish tiers}
\end{table}

\noindent The following peculiarities should be noted in the Turkish KULLDD input:

\begin{itemize}
	\item Many constructions are present on the word level but have no corresponding elements on the morphology tier. 
		This concerns words with the formtype attributes “interjection”, “onomatopoeia”, “family-specific” and the 
		letter construction \texttt{<g><w>...</w><ga type="explan} \texttt{ation"}\texttt{>letter</ga></g>}. Repetitions
		are mostly glossed, retracings mostly not.
	\item Replacements may occur in \texttt{//u/w} or in \texttt{//u/g} with a \texttt{<w>} sibling.
		Shortenings are in \texttt{//u/w} or in \texttt{//u/g/w}.
\end{itemize}

% * Turkish KULLDD is inconsistent with respect to repetitions and retracings. Repetitions are more often glossed than not, so Turkish should be treated like Inuktitut (and differently from MiiPro). However, there are also a lot of non-glossed repetitions, so we’ll get alignment problems that can‘t be helped (it’s the corpus that needs to be corrected here). Retracings, on the other hand, are usually NOT glossed, so there the treatment should be parallel to MiiPro. Fragments don‘t occur in Turkish_KULLDD.
% * replacement: replacements are mostly on same line with <w>, e.g.\ <w>tan<replacement><w>tane</w></replacement></w>, where “tan” is the full_word and “tane” the full_word_target. But there are also replacements in <g><w...</g> tags that cause alignment problems. Those are very inconsistent.
% * shortening: shortened form is full_word, unshortened form is full_word_target. Attention: shortenings are mostly on <w> level (but sometimes also on <g> level), e.g.\ (<w> level): <w>kanyo<shortening>r</shortening></w> (<g> level): <g><w>bak<shortening>a</shortening>ym</w><k type=“retracing”/></g>, this also causes alignment problems.

\noindent The input morphology tier is structured as follows: 

\begin{itemize}
	\item Words are separated by spaces; morphemes are separated by “-”. 
	\item There are no prefixes, so the first morpheme is always the stem. The stem is preceded by a POS tag, separated from it by “|”. 
		Sub-POS can be given using the colon, e.g.\ “PRO:DEM|bu” (replaced by period).
	\item For lexical elements, only the phonological form of the stem is given. By contrast for grammatical elements, only the function of the morpheme is given. 
		This results in “glosses” such as “V|getir-FUT-1S” (= verb with stem “getir”, first suffix = future marker, second suffix = 1st person singular), 
		which in standard interlinearization would be \emph{getir-eceğ-im} for the form and “bring-FUT-1SG” for the gloss. In the ACQDIV Corpus, the unknown
		form of the suffixes and function of the stem are given as \texttt{NULL/NA}.
	\item Grammatical information contained in the stem and subglosses for suffixes are also indicated by “:” (replaced by period). 
	\item Some difficulties are connected to the use of “+” and “\und”, both of which indicate mismatches between word boundaries as indicated by orthography
		and morphology and are completely interchangeable (e.g.\ \emph{bir şey} is considered a single morphological word (\emph{bir+şey}/\emph{bir\und şey})
		meaning ‘something’ but is spelt apart in standard orthography). The corresponding words on the orthographic tier may or may not be joined by
		an underscore. \\
		When a complex containing these characters is indeed treated as a single morphological word (i.e.\ the complex shares a single POS tag and suffix chain), 
		the corresponding orthographic words are joined by “\und” (if they aren’t already). When a complex is treated as two words (i.e.\ they have separate
		POS tags and/or suffixes), the corresponding orthographic words are split (if they aren’t already separate). 	
	% \item The glossed form normally is associated with the target form, although glosses of the actual form are also found. Additional information on the relation between actual and target form may be given in <a type=“errcoding”>, but the format is inconsistent, so it might not be possible to exploit this tier.
\end{itemize}


\section{Yucatec}
\label{sec:Yucatec}

\subsection{Publication, accessibility, documentation}
% projects, citation form, online at, license, docs

The Yucatec corpus \citep{Pfeiler2015a} has not been published. It should be cited as 

\begin{quote}
	Pfeiler, Barbara. Unpublished. Pfeiler Yucatec Child Language Corpus.
\end{quote}

\noindent There is no official documentation available. 


\subsection{Recording scheme}

\begin{table}[ht]
	\centering
	\begin{tabular}{ll}
		\toprule
		number of children 	& 3 \\
		age ranges 			& 1;11.9-3;5.4, 2;0.1-3;0.29, 2;1.5-3;3.11 \\
		recording rhythm 	& 30-90 min every 2 weeks \\
		recording environment & indoors and outdoors at home \\
		other speakers 		& relatives \\
		other languages		& Spanish \\
		\bottomrule
	\end{tabular}
	\caption{Recording scheme for the Yucatec corpus}
	\label{tab:Yucatec recording scheme}
\end{table}


\subsection{File system and formats}
\label{subsec: Yucatec file system and formats}

There are no principled file naming conventions for the Yucatec corpus, although almost all files include the recording date as “MMDDYY” and the code of target children (full or abbreviated) is an additional frequent element. Files are located in a complex folder structure motivated by target children, recording cycles, and steps in the workflow (transcription, glossing). About one third of all files are doublets or triplets.

The original Yucatec CHAT files are formatted as text (structured as CHAT or unstructured) or doc (MS Word). They have highly heterogeneous encodings and a long list of unintended special characters apparently produced by multiple incomplete reencodings. The only characters that naturally appear in Spanish or Yucatec orthography are vowels with acute accents, ⟨ñ⟩, and ⟨ʼ⟩ (= modifier letter apostrophe, U+02BC).

\subsection{Corpus format}

The input format of the Yucatec corpus is TalkBank XML (converted from CHAT by the ACQDIV core team). Table \autoref{tab:Yucatec tiers} shows how the tiers in the ACQDIV Corpus are related to tiers in the input.

\begin{table}[ht!]
	\centering
	\begin{tabular}{lll}
		\toprule
			\textbf{target table} & \textbf{target field} & \textbf{source tier} \\
		\midrule
			sessions 	& session\und id\und fk 	& /CHAT@Id \\
			utterances 	& utterance\und id	& //u@uID \\
			utterances 	& start\und raw		& - \\
			utterances 	& end\und raw		& - \\
			utterances 	& speaker\und label	& //u@who \\
			utterances 	& addressee			& - \\
			utterances 	& sentence\und type	& //u/t \\
			% utterance 	& phonetic			& //u/a[@type="extension" and @flavor="pho"] \\
			utterances 	& utterance\und raw	& //u/w \\
			utterances 	& translation		& //u/a[@type="english translation"] \\
			utterance 	& comment			& //u/a[@type="explanation"], //u/a[@type="comments"] \\
			% disabled tiers
			% utterance	& argument\und coding & - \\
			% utterance & error\und coding	& - \\
			% utterance	& gestures			& - \\
			% utterance	& information\und structure & - \\
			% utterance	& phonetic\und target & - \\
			% utterance	& syntax\und coding	& - \\
			word	 	& word		& //u//w \\ % actually more like full_word_target, but that tier has been disabled
			% disabled tiers
			morpheme	& morpheme			& //u/a[@type="extension" and @flavor="mor"] \\ % actually segments_target, but that tier is disabled
			morpheme	& gloss\und raw		& //u/a[@type="extension" and @flavor="mor"] \\ % actually glosses_target, but that tier is disabled
			morpheme	& pos\und raw		& //u/a[@type="extension" and @flavor="mor"] \\ % actually pos_target, but that tier is disabled
			% disabled tiers
			% morpheme 	& segments\und target & //u/a[@type="extension" and @flavor="mor"] \\ 
			% morpheme	& glosses\und target & //u/a[@type="extension" and @flavor="mor"] \\
			% morpheme	& pos\und target	& //u/a[@type="extension" and @flavor="mor"] \\

		\bottomrule
	\end{tabular}
	\caption{Yucatec tiers}
	\label{tab:Yucatec tiers}
\end{table}

\noindent The Yucatec morphology tier is structured as follows in the input: 

\begin{itemize}
	\item Words are separated by spaces, morphemes by “\#” (prefixes) or “:” (suffixes). 
	\item The morpheme tier may also contain the symbols “\&” and “+”, both of which mark clitics. Since these are most often treated as 
		separate orthographic tiers in \texttt{<w>}, these symbols are treated like spaces (i.e.\ as word separators) in the
		morphology tier. 
	\item Every morpheme block consists of a gloss and a morpheme form, separated by “|”. The gloss of stems is a part of speech rather than a 
		functional label. The form of suffixes is preceded by a redundant “-”. An example for a word with both prefixes and suffixes is 
		“3ERG|u\#VN|hoʼol:POS|-il” (standard interlinearization: form \emph{u-hoʼol-il}, gloss “3ERG-VN-POS”). In the ACQDIV Corpus, \texttt{NULL/NA}
		is inserted when the function of a stem is not known. 
	\item In many glosses “:” is also used to separate one or several subglosses, e.g.\ “IMP:ABS:SG”. This use can be distinguished 
		from the morpheme-separating use by checking the strings to the left and right of the “:” \textendash\ when they consist 
		of nothing but uppercase letters and digits, they are subglosses; otherwise they belong to different morphemes.
	\item Sometimes words do not contain any “\#” or “:” but do contain “-”. In this case “-” represents a morpheme separator. 
		Words with “-” as the morpheme separator only contain morpheme forms but no glosses. 
\end{itemize}

% earlier stuff < corpus_parser_documentation.md, written by Danica: 

% * The segmentation on the **morpheme level** is done as follows:
%
% - Split “mor” tier into words on white space.
%   1. Get *pronouns* and *proper names*:
%     - `pron = re.search(’^\d?.*?:.*?\|.*?$‘, w)`
%     - example: `<a type=“extension” flavor=“mor”>N:PROP|Koncha N:PROP|Koncha .</a>`
%   2. Get words that contain *neither prefixes nor suffixes*:
%     - These are the ones neither containing “:” nor “#”
%     - example: `INT|baʔax`
%   3. Get the prefixes and suffixes from the words (Go through words on morpheme level and split on ’#‘ or ’:‘, preserving the delimiter! Preserving the delimiter is needed in order to be able to distinguish prefixes from suffixes in the steps that follow)
%     1. Get the *prefixes* (this had to be done in a separate step in order to catch the prefixes before the suffixes): split on “#”, preserving the delimiter.
%       - `check_pfx = re.search(’(.*)\|(.+#)‘,w)
%       - `prefixes = ’#,‘.join(w.split(’#‘)).split(’,‘)`
%     2. Get *stem* of words that contain prefix(es) and suffix(es)
%       1. Get stem of words that have prefix(es) *and* suffix(es)
%         - `stem_marker = re.search(’(.*)?#(.*)\|(.*?):‘, w)`
%         - example: `PFV|0#VI|bin:PFV|-0`
%       2. Get stem of words that *only have prefix(es)*
%         - `stem_marker2 = re.search(’(\|.*)?#(.*)\|(.*?)$‘, w)`
%           - example: `2POS|a#S|nak
%         - `stem_marker3 = re.search(’^(.*)\|(.*?)[:\|]‘, w)` (this is for prefixes with wrongly annotated “:” instead of “#”)
%   4. Get the *suffixes* (split on “:”, preserving the delimiter)
%     - `sfx_marker = re.search(’:‘, w)`
%     - `suffixes = ’,:‘.join(w.split(’:‘)).split(’,‘)
% * Most alignment problems occur because of inconsistent annotation on the mor tier level (e.g.\ due to white spaces before delimiters or the additional “-” before the form tag of the suffixes).

% EOF Subcorpora and original data


\chapter{Generating the corpus}
\label{cha:Generating the corpus}
% @RS: we should decide what we are calling this "thing" in general -- ACQDIV database or ACQDIV corpus (or my fav ACQDIV matrix)?
% @SM: Do we already have red and blue pills? 
% I'll vote for "corpus". "Database" sounds a bit intimidating and also is more vague (most linguistic databases probably do not look like ours). 

% @SM: discuss with SM: this chapter rather reflects the "functional" side what we did while ignoring the technical side. For instance, character replacements, file name corrections and CHAT corrections are treated in three different places even though they were all performed by code.py (right?).

The ACQDIV Corpus is dynamically generated from the original data described in \autoref{cha:Data sources}. A new version is generated every time the original data or their interpretation change. 

The original data are extremely heterogeneous and often have greater or smaller internal problems. Therefore, creating a single user-friendly corpus from them requires several processing steps: 

\begin{itemize*}
	\item clean files of formal issues that hinder automatic processing (e.g.\ problematic encodings and file formats)
	\item ensure compliance with applicable corpus standards as far as necessary
	\item parse the data and metadata, i.e.\ read the information contained in them and store it in a temporary unified structure
	\item build the database and map the unified structure into it
	\item postprocess the data in the database for greater semantic homogeneity (e.g.\ with regard to glosses or timestamps)
\end{itemize*}

Note that the first two steps, which also involved manual cleaning, were only carried out once during the initial phase of the project. The remaining steps are fully automatised and are repeated every time the corpus is generated. The following section gives an overview of what happens during all steps in which corpora. It reflects the conceptual rather than the technical functioning of corpus generation. For details on the technical side see \autoref{cha:Information for developers} and the documentation that comes with the individual scripts involved in each step.


\section{Cleaning of file formats}
\label{sec:Cleaning of file formats}

The first cleaning step deals with general issues that make files hard to process automatically. Our goal was for every corpus to have a flat collection of text files that had UTF-8 encoding and the intended character set. In addition, one file should correspond to one recording session (in the sense of a contiguous stretch of time) and vice versa. Both files and their names were required to be unique within one corpus. The sections below describe how this goal was achieved. 

\subsection{Non-textual formats}
Most corpora were already formatted as structured text (e.g.\ XML or Toolbox) at the time the ACQDIV project started working on them. However, there were a few exceptions that were dealt with as follows: 

\begin{itemize*}
	\item The Yucatec and to a lesser degree the Turkish corpus contained many doc files. The text was extracted using \texttt{doc2txt} and saved with the extension txt. 
	\item All files in the Inuktitut corpus were text but had undocumented file extensions (XXS, XXX, NAC). These were converted to txt. 
	\item Some of the Inuktitut corpus documentation had Word Perfect formats (REP, SUB, IKT). These were converted to pdf. 
\end{itemize*}

\subsection{Encodings}

All corpora were required to be encoded in UTF-8. This was not the case for most Inuktitut and Yucatec as well as for some Russian and Turkish files, where ISO-Latin and ASCII were found among other, less common encodings. Encodings were determined using the Python library \texttt{chardet} and converted to UTF-8.  

\subsection{Character sets}

The same three corpora (Inuktitut, Turkish, Yucatec) also had problems with unintended special characters such as letters with accents, letters from foreign alphabets, and non-textual characters such as suns or alien heads. Problems of this kind were least prominent in Inuktitut, whose orthography does not feature any special characters, but very widespread in Turkish (special characters ⟨ç⟩, ⟨ğ⟩, ⟨ö⟩, ⟨ş⟩, ⟨ü⟩) and Yucatec (vowels with acute accents, special characters ⟨ñ⟩, ⟨ʼ⟩ (= modifier letter apostrophe, U+02BC)). Character lists were automatically extracted from all files of these corpora, replacement lists were compiled for all corrupted characters, and automatic replacements were made wherever such a corrupted character uniformly corresponded to a well-formed character. 

\subsection{Folder systems and file names}

Many corpora initially had deeply nested folder systems which often obscured the actual structure of the corpus or made it possible for a corpus to contain two or more files with the same name. The following steps were undertaken to overcome these problems:

\begin{itemize}
	\item Whenever a corpus was available in different formats (e.g.\ CHAT vs.\ TalkBank XML), only the strictly required formats were taken over. 
	\item The Indonesian corpus originally had several subfolders containing subcorpora for each target child and named after their code. Within the folders, Toolbox files were originally named as “COD-DDMMYY” (where “COD” represents the speaker code of the target child) and XML files were more simply named as “YYYY-MM-DD” with no indication of the target child. Both formats were unified to “COD-YYYY-MM-DD” in the input data for the ACQDIV Corpus to achieve consistency and unique session names across all folders. All files were put on the same level. 
	\item In the Inuktitut corpus, recording sessions often corresponded to several files within one subfolder. All such files were fused to a single file, keeping the shared string in the beginning of the file name and replacing everything else by “All” (e.g.\ “JUP21All” instead of “JUP21ATF”, “JUP21BTF”, “JUP21CTF” etc.). The merged files were put on the same level as the pre-existing other files, thus creating a flat structure. 
	\item The Japanese MiiPro files were originally located in folders named after the target children but were all put on the same level for input in the ACQDIV Corpus. 
	\item The Japanese Miyata subcorpus is a particularly complicated case. Every session is represented at least twice and maximally four times by files with largely identical contents but different file names \textendash\ see the \hyperref[subsec:Miyata file system and formats]{description of the original data} for details. Only the most recent series of files was used for each child (Aki 3, Ryo 4, Tai 4). Since these series did not indicate the name of the target child in the file name and were thus potentially ambiguous, the codes were prefixed to the file name with an underscore (“aki\und 34\und 20629”) before putting all files on the same level. 
	\item The Russian corpus consists of several parallel versions with different annotations in separate folders. For the ACQDIV Corpus the folder was used that contained most of all recent annotations and glosses based on the Leipzig Glossing Rules (“4a\und tbx\und lemma\und separated\und timecodes\und\ lgr”). 
	\item The Turkish corpus contained subfolders for target children. This structure was flattened as in the other corpora. 
	\item Yucatec is another corpus with many doublets and triplets, which in this case becomes possible by a complicated folder structure and competing naming conventions (see the \hyperref[subsec: Yucatec file system and formats]{description of the original data} for details). Doublets were detected by checking all file names and especially the string of digits contained in them, which turned out to be most indicative of session identity. In the next step, the most recent version in every doublet set was determined based on file size and annotation layers. Older versions were discarded and all files were renamed according to the scheme “COD-YYYY-MM-DD” and put on a single level. Where all versions represented the same level of analysis the version kept was the one which was easiest to process (e.g.\ because of encodings). 
\end{itemize}


\section{Cleaning of corpus formats}
\label{sec:Cleaning of corpus formats}

The two corpus formats that were accepted as input for the ACQDIV Corpus are TalkBank XML and Toolbox. CHAT was deliberately excluded from this list because a good CHAT to XML parser is available with \href{http://talkbank.org/software/chatter.html}{Chatter}. However, three corpora \textendash\ Inuktitut, Turkish and Yucatec \textendash\ were delivered in broken CHAT that could initially not be parsed by Chatter. Given the choice to either write a new CHAT parser or clean up the CHAT in order to make it convertible, we decided that it would be both easier and more sustainable to go for the latter option. 

One of the most frequent parsing problems were broken headers. The header of a CHAT file is an obligatory section at the head of the file that lists all session-level and speaker-level metadata associated with it. Some examples for problems with headers are to missing information, corrupted tier names, or whitespace characters in the wrong places. Cleaning headers required the following steps: 

\begin{itemize}
	\item Specify a set of non-discardable metadata. For the session level these were the recording date (CHAT tier \texttt{@Date:}), the recording situation (\texttt{@Situation:}), and the name of the associated media file
		(\texttt{@Media}). For the speaker level the non-discardable data were code, name, age, sex, and spoken languages (all coded on the CHAT tier \texttt{@ID:}) as well as role (\texttt{@Participants:}). 
	\item For each corpus create a table containing all existing information from all metadata tiers. Our collaborators were requested to go through these tables fill in any gaps in the data. In addition, 
		tiers that had been found in the corpus but did not form part of the above-mentioned list were to be deleted by default. The collaborators were asked to go through these tiers and to transfer any contents
		that they wanted to be retained to another tier from the standardized set. For instance, one frequent pseudo-tier that was not accepted by Chatter was \texttt{@Age of CHI:}. The contents of this tier
		could easily be transferred to a modified or newly created \texttt{@ID:} tier for the target child. 
	\item Then clean the finished tables of any remaining clutter and convert them to two simple CSV files per corpus: \texttt{ids.csv} for speaker-level metadata and \texttt{sessions.csv} for session-level metadata. 
		These files were then used by the pyacqdiv package (cf.\ \autoref{cha:Information for developers}) to create a new metadata header for every file in each corpus. All pre-existing information that had not been 
		captured in the metadata tables was overwritten in this process. 
\end{itemize}

The body data presented different problems and were therefore dealt with separately. While these did not contain any non-systematic gaps, there were many more formatting problems, many of them having to do with whitespace characters and special characters, which are only allowed in specific places in CHAT. These problems were dealt with in the following way: 

\begin{itemize} 
	\item Chatter was systematically run over the data. Error message produced by the parser were collected and sorted by frequency in order to be able to deal with the most frequent problems first. 
	\item For all types of problems with a token frequency roughly higher than 50, replacement rules were collected in a file called \texttt{code.py} (one per corpus). This served again as input for
		the pyacqdiv package, which applied the replacement rules to all files in each corpus. 
	\item All problems with lower frequencies were corrected manually, either by the ACQDIV team or by the collaborators responsible for the particular subcorpus. This was mainly done in \href{http://childes.psy.cmu.edu/clan/}{CLAN}, although
		certain differences between the standards applied in CLAN and Chatter sometimes brought to light additional problems when attempting to parse corrected files in Chatter. 
\end{itemize}

Although all corpora apart from the ones mentioned above superficially did not contain problematic formats, most of them do on a deeper level. The reason for this is always that the corpora were converted from CHAT and still contain traces of it. This applies to the following cases: 

\begin{itemize}
	\item Japanese MiiPro and Sesotho are available as TalkBank XML. However, the morphology tier has no explicit internal structure but has been directly taken over from CHAT as a single string. This CHAT pocket 
		has to be parsed differently from the rest of the files (see \autoref{sec:Parsing the corpus data} below for details). 
	\item Similarly, Inuktitut, Turkish and Yucatec had morphology tiers with rather different conventions (see the description of corpus-specific conventions in \autoref{cha:Data sources} for details). 
		These differences seem to reflect changes in the CHAT standard during time. Because of this and also because the morphology tier is by far the most complex CHAT tier with the most possibilities for formal mistakes, 
		we decided not to clean it but to transfer it directly to XML and deal with it in the parser, as for Japanese MiiPro and Sesotho. 
	\item The Indonesian Toolbox files contain metadata in CHAT format that have been superficially “masked” as Toolbox (see again \autoref{cha:Data sources} for details).
	\item Russian contains CHAT codes on some of its Toolbox tiers, especially on the main transcription tier. 
\end{itemize}

All of these problems do not affect the convertibility of files, so they were not dealt with in the cleaning block but directly in the parser. 


\section{Parsing the corpus data}
\label{sec:Parsing the corpus data}

Parsing in a narrow sense concerns the process that transforms one of the accepted input formats (TalkBank XML, Toolbox) into an output format (e.g.\ a SQLite database, cf.\ the \hyperref[sec:Structure of the corpus]{specifications}). This is the most complex process in the corpus pipeline and can only be roughly sketched here. For more details refer to the documentation of the relevant scripts. 

\subsection{TalkBank XML}

Parsing a TalkBank XML file involves the following conceptual steps: 

\begin{itemize}
	\item The XML tree is read using a Python library such as ElementTree. 
	\item The content of unproblematic tiers (e.g.\ translations, timestamps) is transferred as a whole to the relevant target field. 
	\item The problematic tiers are on the word level (represented by \texttt{<w>}) and the morpheme level (segmentation, glosses, and POS, mostly on a single tier, sometimes spread over several). 
		These have to be split (especially the morpheme tier, which has no explicit internal structure) and aligned with each other (so that one \texttt{<w>} word corresponds to one morphological word
		and each segment has matching glosses and POS tags). 
	\item The contents of \texttt{<w>} are cleaned of any remnants of CHAT or other idiosyncratic conventions. The contrast between actual and target form is built based on the constructs described 
		in \autoref{subsec:TalkBank XML}. For instance, the construct \texttt{<g><w><shortening>com</shortening>puter</w> </g>} would give rise to an actual word \emph{puter} and a target word \emph{computer}. 
	\item The same constructs also influence the interpretation of the morphology tier. For instance, fragments are not glossed in many corpora, i.e.\ there is no element on the 
		morphology tier that corresponds to the word coded by \texttt{<w>}. After encountering such a word in \texttt{<w>}, the parser adjusts the indices for aligning morphological words. 
	\item The morphology tier(s) are cleaned of small-scale inconsistencies and components which are redundant or not compatible with the other corpora. Morpheme-internal spaces are removed. 
		They are then split into words and morphemes. 
	\item Some corpora occasionally specify distinct actual and target forms for words or morphemes on the morphology tier(s). These are separated and stored before all following actions. 
	\item Segments, glosses, and POS are identified based on the corpus-specific formalism. In some cases the formal coding is in contrast to the real function of the coded element. For instance, 
		labels may sometimes be formally marked as glosses but rather code POS. In this case contents are transferred from one category to another as far as possible. 
	\item The words in \texttt{<w>} are now aligned with the morphological words, taking into account what is known about missing glosses. 
\end{itemize}

\subsection{Toolbox}

Toolbox files are somewhat simpler to parse because they have a flatter structure than XML, thus providing less opportunities for drastic mismatches between coding syntax and semantics. Parsing a Toolbox XML file involves the following conceptual steps: 

\begin{itemize}
	\item Files are split into records based on linebreak characters. Each record contains several lines, which at the same time correspond to its tiers. 
	\item All tiers are cleaned of remnants of CHAT or other idiosyncratic conventions. 
	\item The content of unproblematic tiers (e.g.\ translations, timestamps) is transferred as a whole to the relevant target field. 
	\item Tiers coding words are split into words by spaces. 
	\item Tiers coding morphemes are split into morphemes by spaces. The boundaries of morphological words then have to be reconstructed based on morpheme separators. For instance, 
		given the string \emph{play -ing with} the parser can infer that there is a word boundary between \emph{-ing} and \emph{with} because a suffix cannot be followed by a stem
		(at least given the definition of these terms in Toolbox). Identifying segments, glosses, and POS is trivial because these are given on separate tiers. 
	\item The last step concerns alignment. Orthographic words are aligned with morphological words, and for every morpheme-level element the parser checks if there are corresponding 
		elements on all three tiers (segments, glosses, POS). Note that alignment in Toolbox is always based on corresponding indices (i.e.\ the first element of set 1 corresponds to
		the first element of set 2, etc.).
\end{itemize}

\subsection{Intermediate storage}

The parsed data are stored in a nested Python structure before inserting them into the database. This structure can be schematically depicted as follows: 

\begin{verbatim}

{
   "corpus": {
      "session": {
         "utterance": {
            [
               ["key", "value"], 
               ["key", "value"],
               ["key", "value"]
            ]
         }
      }
   }
}
	
\end{verbatim}

where the \texttt{["key", "value"]} tuples give all information contained in one tier irrespective of its level (utterance, word, morpheme). Multiple words or morphemes are passed on as a single string (e.g.\ \texttt{[["morpheme", "Bäum e"], ["gloss", "tree PL"], ["pos", "n sfx"]]}). 

% corpus1
% ( session1 (ordered dictionary)
% 	( utterance 1 (ordered dictionary)
% 		[(seg, "baum -e"), (glos, "tree -PL"), (pos, "n -sfx"), ...] (tuples)
% 	)
% 	( utterance 2 )
% )
% ( session2 )
%

An earlier version of the parser used JSON for intermediate storage. The main difference to the new structure is that there used to be additional nested levels below the utterance level. One full example for a record mapped to JSON is given below for legacy reasons. 

% TODO delete once no longer relevant

\begin{verbatim}

{
   "session_name": [
      {
         "utterance_id": "child1-session17.386",
         "speaker_id": "CHI1",
         "addressee": "MOT2",
         "starts_at": "00:12:53",
         "ends_at": "00:12:55",
         "orthographic": "Atukunai no?",
         "phonetic": "atsunaino:",
         "phonetic_target": "atsukunai no",
         "english": "Isn't it cold?",
         "nepali": "Garmi chaina?",
         "spanish": "No es caliente?",
         "sentence_type": "question",
         "comments": "Mother has gone out, CHI1 talking to herself.",
         "warnings": "English translation might be wrong",
         "words": [
            {
               "full_word": "atsunai",
               "full_word_target": "atsukunai",
               "warnings": "transcription might be wrong",
               "morphemes": [
                  {
                     "segments": "atsu",
                     "glosses": "hot",
                     "pos": "adj"
                  },
                  {
                     "segments": "na",
                     "segments_target": "kuna",
                     "glosses": "NEG",
                     "glosses_target": "NEG",
                     "pos": "sfx"
                     "pos_target": "sfx"
                  },
                  {
                     "segments": "i",
                     "glosses": "NPST",
                     "pos": "sfx"
                  }
               ]
            },
            {
               "full_word": "no"
               "full_word_target": "no"
               "morphemes": [
                  {
                     "segments": "no",
                     "glosses": "Q",
                     "pos": "ptcl",
                  }
               ]
            }
         ]
      }
   ]
}

\end{verbatim}


\section{Parsing the metadata}
\label{sec:Parsing the metadata}

Metadata are often stored in separate files and/or structured differently from corpus data. They are therefore parsed separately from the latter. 

Metadata are either read from TalkBank XML files, where data and metadata for a session are stored in a single file, or from IMDI XML, a generalized metadata standard that is commonly used in combination with Toolbox data. A special case is presented by Indonesian, where the latest data are formatted as Toolbox but the metadata are still in TalkBank XML, reflecting the origin of the corpus in CHAT. 

Compared to the data, the metadata are relatively easy to parse because their syntax is more robust (and thus much less frequently broken), string operations such as replacements, deletions or splits are not necessary, and elements of fields do not have to be connected or moved (in contrast to the multiple alignments described above for the data). The data are therefore simply read and transferred to the relevant target fields in the database. Note that the metadata tables \texttt{sessions} and \texttt{speakers} are generated independently of the corpus data tables and can only be relinked to them via the relevant keys (\texttt{corpus}, \texttt{session\und id}, \texttt{speaker\und label}). 


\section{Building the database and postprocessing}
\label{sec:Building the database and postprocessing}

Once the corpus data and metadata have been parsed as described above, a database skeleton with the structure described in \autoref{sec:Structure of the corpus} is built and the internal data structures are mapped to it. Numeric IDs are automatically generated for all records. 

After the tables have been filled, any edits and additions that are best performed over all data simultaneously are carried out. This is referred to as postprocessing. Postprocessing adds the following tables and columns (also cf.\ again \autoref{sec:Structure of the corpus}): 

\begin{itemize}
	\item Glosses (\texttt{morphemes.gloss\und raw}) and parts of speech (\texttt{morphemes.pos\und raw}) are unified across corpora based on the lists given in \autoref{subsec:Grammatical glosses} and \autoref{subsec:Part-of-speech tags}. 
		The modified labels are written to \texttt{morphemes.gloss} and \texttt{morphemes.pos}/\texttt{words.pos\und ud}, respectively. 
	\item \texttt{utterances.utterance\und raw} is cleaned of any remaining punctuation and other special markers (cf. the \hyperref[subsec:Transcription conventions]{transcription conventions}); the new content is put into \texttt{utterances.utterance}.
	\item In the \texttt{utterances} table, \texttt{start\und raw} and \texttt{end\und raw} are unified to the format HH:MM:SS. 
	\item The \texttt{language} field is set in various tables tables based on the \texttt{corpus} field.
	\item The contents of \texttt{speakers.age\und raw} are converted to the language acquisition format YY;MM.DD in the \texttt{speakers.age} field.
		An additional representation in days is inserted into \texttt{speakers.age\und in\und days}. 	
	\item The gender values found in \texttt{speakers.gender\und raw} are standardized to the two values \texttt{Female} and \texttt{Male}
		in the column \texttt{speakers.gender}. Where the gender is not known the postprocessor tries to infer it from role values such as \texttt{Mother}. 
	\item The many values of \texttt{speakers.role\und raw} are greatly simplified based on the controlled vocabulary given in \autoref{subsec:Roles and macroroles}. 
	\item Based on age, standardized role, and ID, a new column \texttt{speakers.macrorole} is inserted, which only allows the three values
		\texttt{Child}, \texttt{Target\und child}, \texttt{Adult}. 
	\item The complete \texttt{uniquespeakers} table is generated based on inference from the name, speaker label, and birthdate of speakers as occurring in sessions. 
\end{itemize}

Postprocessing also checks the tables \texttt{utterances}, \texttt{words}, and \texttt{morphemes} for completeness and consistency, editing the field \texttt{warnings} in each of the tables in the following way: \\

\begin{longtable}{lp{.2\linewidth}p{.6\linewidth}}
	\toprule
		\textbf{table} & \textbf{warning} & \textbf{explanation} \\
	\midrule
	\endhead
	
	\bottomrule\\[-0.15cm]
	\caption{Overview of warnings}
	\endfoot
	
	utterances 		& insecure			 		& the transcription for the complete utterance is insecure, \\
					& transcription 			& e.g.\ because the utterance is hard to understand \\
	utterances		& not glossed				& the complete utterance has not been glossed for some reason \\
	utterances		& broken word			 	& the number of words coded on the main transcription tier \\
					& alignment	 				& is bigger or smaller than the one coded on the morphology tier so that it’s not clear which words are associated \\
	words			& broken morpheme 			& the number of morphemes coded on one of the logical \\
					& alignment 				& morphology tiers (morphemes, glosses, parts of speech) is bigger or smaller than the one coded on another of these tiers so it’s not clear which annotations are associated \\
	words			& contains	 				& the marked word seems to contain morpheme boundaries \\
					& unstructured 				& but can’t be split because the formatting is broken \\
					& morpheme 					& \\
	morphemes		& annotation insecure 		& the marked morpheme, gloss, or part-of-speech tag \\
					& for tier X				& is insecure \\[-0.3cm]
	
	\label{tab:Overview of warnings}
\end{longtable}

% EOF Generating the corpus


\chapter{Information for developers}
\label{cha:Information for developers}

The ACQDIV Corpus has been cleaned and is generated with Python 3. For cleaning, a Python package called \texttt{pyacqdiv} has been created that is currently still under development and might incorporate database generation in the future. The database is generated by a suite of scripts, the most important ones being the loader, which reads corpus-specific information from ini files, calls the matching data and metadata parsers, and creates the database, and the postprocessor, which checks the contents of database tables and applies rules for normalization. A testing mechanism based on gold standards ensures that changes in the generating and processing scripts do not alter the structure of the database in undesirable ways. 

This architecture is the result of team work that has been made possible with the help of GitHub. Currently the GitHub repository at \url{https://github.com/uzling/acqdiv} is closed to the public. Developers wishing to get access should contact \href{mailto:steven.moran@uzh.ch}{Steven Moran} (tech lead) or \href{mailto:robert.schikowski@uzh.ch}{Robert Schikowski} (project coordinator). 

% The database creation pipeline described in Section \ref{} performs an Extract, Transform and Load (ETL) process on a set of heterogeneous corpora data formats, each of which is described in Section \ref{}. Basically, our ETL approach extracts the data fields described in detail in Section \ref{} into a set of relational database tables, which can be accessed in several different formats that we export, as described below. First we present an overview and our database design.
%
% Within the ETL process, we go from original data source formats to a relational database. At the moment, our database format is `sqlite3`. We also write to disk plain CSV (comma separated value) documents, where each column in each data table (described above) is separated by a comma. There are several CSV tools on the market to explore the CSV files' contents. Lastly, we use the CSV tables to create an R  data object, so that users can explore the ACQDIV database through the R statistical tool.


\newpage
\bibliographystyle{unified.linguistic} 
\bibliography{references.bib}


\end{document}
